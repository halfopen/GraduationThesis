\documentclass[type=master]{fduthesis}
\usepackage{subfigure}
\usepackage{graphics}
\usepackage{multirow}
\usepackage{booktabs}
\usepackage{float}
\usepackage{enumerate}
\usepackage{ctex}
\usepackage{xcolor}
\usepackage{listings}
% \usepackage{pdfpages}
% \usepackage[AutoFakeBold=true, AutoFakeSlant=true]{xeCJK}

% 引用
\newcommand\myfont[1]{
    \textit{“#1”}
}

% \usepackage{cite}

% 模板选项:
%   type = doctor|master|bachelor  论文类型,默认为本科论文
%   oneside|twoside                论文的单双面模式,默认为 twoside
%   draft = true|false             是否开启草稿模式,默认关闭
% 带选项的用法示例:
%   \documentclass[oneside]{fduthesis}
%   \documentclass[twoside, draft=true]{fduthesis}
%   \documentclass[type=bachelor, twoside, draft=true]{fduthesis}

\fdusetup{
  % 参数设置
  % 允许采用两种方式设置选项:
  %   1. style/... = ...
  %   2. style = { ... = ... }
  % 注意事项:
  %   1. 不要出现空行
  %   2. “=” 两侧的空格会被忽略
  %   3. “/” 两侧的空格不会被忽略
  %   4. 请使用英文逗号 “,” 分隔选项
  %
  % style 类用于设置论文格式
  style = {
    font = lm,
    % 西文字体(包括数学字体)
    % 允许选项:
    %   font = garamond|libertinus|lm|palatino|times|times*|none
    %
    % cjk-font = fandol,
    % 中文字体
    % 允许选项:
    %   cjk-font = adobe|fandol|founder|mac|sinotype|sourcehan|windows|none
    %
    % 注意:
    %   1. 中文字体设置高度依赖于系统。各系统建议方案:
    %        windows:cjk-font = windows
    %        mac:    cjk-font = mac
    %        linux:  cjk-font = fandol(默认值)
    %   2. 除 fandol 和 sourcehan 外,其余字体均为商用字体,请注意版权问题
    %   3. 但 fandol 字体缺字比较严重,而 sourcehan 没有配备楷体和仿宋体
    %   4. 这里中西文字体设置均注释掉了,即使用默认设置:
    %        font     = times
    %        cjk-font = fandol
    %   5. 使用 font = none / cjk-font = none 关闭默认字体设置,需手动进行配置
    %
    font-size = -4,
    % 字号
    % 允许选项:
    %   font-size = -4|5
    %
    fullwidth-stop = false,
    % 是否把全角实心句点 “.” 作为默认的句号形状
    % 允许选项:
    %   fullwidth-stop = catcode|mapping|false
    % 说明:
    %   catcode   显式的 “。” 会被替换为 “.”(e.g. 不包括用宏定义保存的 “。”)
    %   mapping   所有的 “。” 会被替换为 “.”(使用 LuaLaTeX 编译则无效)
    %   false     不进行替换
    %
    % footnote-style = pifont,
    % 脚注编号样式
    % 允许选项:
    %   footnote-style = plain|libertinus|libertinus*|libertinus-sans|
    %                    pifont|pifont*|pifont-sans|pifont-sans*|
    %                    xits|xits-sans|xits-sans*
    %
    hyperlink = none,
    % 超链接样式
    % 允许选项:
    %   hyperlink = border|color|none
    %
    hyperlink-color = default,
    % 超链接颜色
    % 允许选项:
    %   hyperlink-color = default|classic|elegant|fantasy|material|
    %                     business|science|summer|autumn|graylevel|prl
    % 默认与西文字体保持一致
    %
    bib-backend = bibtex,
    % 参考文献支持方式
    % 允许选项:
    %   bib-backend = bibtex|biblatex
    %
    % bib-style = plain,
    % 参考文献样式
    % 允许选项:
    %   bib-style = author-year|numerical|<其他样式>
    % 说明:
    %   author-year  著者—出版年制
    %   numerical    顺序编码制
    %   <其他样式>   使用其他 .bst(bibtex)或 .bbx(biblatex)格式文件
    %
    % cite-style = {},
    % 引用样式
    % 默认为空,即与参考文献样式保持一致
    % 仅适用于 biblatex;如要填写,需保证相应的 .cbx 格式文件能被调用
    %
    bib-resource = {bibliography.bib},
    % 参考文献数据源
    % 可以是单个文件,也可以是用英文逗号 “,” 隔开的一组文件
    % 如果使用 biblatex,则必须明确给出 .bib 后缀名
    %
    % logo = {fudan-name.pdf},
    % 封面中的校名图片
    % 模版已自带,通常不需要额外配置
    %
    % logo-size = {0.5\textwidth},      % 只设置宽度
    % logo-size = {{}, 3cm},            % 只设置高度
    % logo-size = {8cm, 3cm},           % 设置宽度和高度
    % 设置校名图片的大小
    % 通常不需要调整
    %
    auto-make-cover = false
    % 是否自动生成论文封面(封一)、指导小组成员名单(封二)和声明页(封三)
    % 除非特殊需要(e.g. 不要封面),否则不建议设为 false
  },
  %
  % info 类用于录入论文信息
  info = {
    title = {日常健康中面诊交互技术的研究},
    % 中文标题
    % 长标题建议使用 “\\” 命令手动换行(不是指在源文件里输入回车符,当然
    % 源文件里适当的换行可以有助于代码清晰):
    %   title = {最高人民法院、最高人民检察院关于适用\\
    %            犯罪嫌疑人、被告人逃匿、死亡案件违法所得\\
    %            没收程序若干问题的规定},
    %
    title* = {Face Reading Technologies for Everyday Health},
    % 英文标题
    %
    author = {秦贤康},
    % 作者姓名
    %
    % author* = {Your name},
    % 作者姓名(英文 / 拼音)
    % 目前不需要填写
    %
    supervisor = {丁向华\quad 副教授},
    % 导师
    % 姓名与职称之间可以用 \quad 打印一个空格
    %
    major = {计算机应用技术},
    % 专业
    %
    department = {计算机科学与技术},
    % 院系
    %
    student-id = {17210240182},
    % 专硕
    %
    degree = professional,
    % 作者学号
    %
    % date = {2019 年 1 月 1 日},
    % 日期
    % 注释掉表示使用编译日期
    %
    % secret-level = ii,
    % 密级
    % 允许选项:
    %   secret-level = none|i|ii|iii
    % 说明:
	%   none  不显示密级与保密年限
    %   i     秘密
    %   ii    机密
    %   iii   绝密
    %
    % secret-year = {五年},
    % 保密年限
    % secret-level = none 时该选项无效
    %
    instructors = {
      {顾\quad 宁 \quad 教\quad 授},
      {张\quad 亮 \quad 教\quad 授},
      {丁向华 \quad 副教授},
      {卢\quad 暾 \quad 副教授}
    },
    % 指导小组成员
    % 使用英文逗号 “,” 分隔
    % 如有需要,可以用 \quad 手工对齐
    %
    keywords = {人机交互, 面诊系统, 日常健康},
    % 中文关键字
    % 使用英文逗号 “,” 分隔
    %
    keywords* = {HCI, face diagnosis system, everyday health},
    % 英文关键字
    % 使用英文逗号 “,” 分隔
    %
    clc = {TP3}
    % 中图分类号
  }
}

\lstset{
  numbers=left, %设置行号位置
  numberstyle=\tiny, %设置行号大小
  keywordstyle=\color{blue}, %设置关键字颜色
  commentstyle=\color[cmyk]{1,0,1,0}, %设置注释颜色
  frame=shadowbox, %设置边框格式
  escapeinside=``, %逃逸字符(1左面的键),用于显示中文
  %breaklines, %自动折行
  extendedchars=false, %解决代码跨页时,章节标题,页眉等汉字不显示的问题
  xleftmargin=2em,xrightmargin=2em, aboveskip=1em, %设置边距
  tabsize=4, %设置tab空格数
  showspaces=false %不显示空格
}


\begin{document}

% \raggedbottom
% 这个命令用来关闭版心底部强制对齐,可以减少不必要的 underfull \vbox 提示,但会影响排版效果

\frontmatter
% 前置部分包含目录、中英文摘要以及符号表等

\tableofcontents
% 目录

\begin{abstract}



随着经济高速发展、人们生活节奏日渐加快以及养生意识逐步提高,越来越多的人开始关注自己的健康问题。
近些年移动设备日渐普及和人工智能技术的快速发展,出现了大量的支持用户在日常生活中进行自我诊断和健康管理的技术。
本文主要关注基于智能读脸技术的健康技术即面诊技术在日常场景下的应用及相关的交互问题。
目前已有的面诊技术主要还是面向诊所医院等专业环境,在日常生活场景中使用面诊技术进行日常健康诊断和管理,虽然在技术层面已经非常深入,在用户交互方面上还存在诸多挑战;而在人机交互领域虽然已经有大量关于日常场景下健康管理的研究,但主要集中在用户健康行为引导和如何帮助用户更好地进行慢性病管理等方面,关于面诊技术在日常健康情景中的使用和交互问题的研究还比较缺乏。将面诊技术从专业的诊所环境应用到日常健康场景,能在一定程度上缓解当前整体公共医疗资源不足的压力,帮助用户更加健康地生活。

% 本文针对日常环境下的面诊技术如何支持日常情境中的健康诊断和管理的应用和交互问题进行了探索,主要工作如下:

本文探索了如何支持日常场景中的面诊技术的应用及交互问题,主要工作如下:

1.利用技术探针进行日常场景下面诊应用的交互研究。
技术探针是交互领域中适合在真实场景中让用户参与设计过程的设计方法。本文使用诊所环境下的一款面诊应用(云中医)作为技术探针,通过对参与者在日常场景下进行深度访谈,分析转录成文字后的访谈资料,探索了日常场景下面诊应用新的潜在使用场景,发现了面诊技术在日常场景下会遇到长期使用、实用性、敏感性、情绪与信赖等方面的问题。针对这些问题,本文提出了对应的设计思路:支持可持续使用、系统可解释性和日常可用性。

2.解决日常场景下系统可用性问题,设计并实现了可拓展的面诊应用交互设计实验平台。
为了提高后续交互实验的效率,同时解决日常场景下用户设备多样性带来的稳定性的等问题,本文在系统设计层面设计并实现了用于快速开展面诊应用及其交互设计的实验平台。平台实现了系统高可用、模型服务化与客户端分离、用户操作记录管理和第三方问卷支持等功能,这些功能的设计提高了系统的稳定性,能够应对后面大规模的用户交互实验场景并快速地开展新的交互实验。

% 通过调用实验平台接口对跨平台客户端进行实现并通过实验验证设计思路的有效性。模型运行在服务器,解决了用户设备的差异性带来的稳定性和性能的问题。

3.利用实验平台,分别实现并验证了增加系统可解释性和支持可持续使用的设计思路。
本文在实验平台上实现了两个交互设计,分别探索了如何通过交互设计增强系统的日常可用性,以及如何对系统的背后原理以及影响权重进行解释来增加系统可解释性。
通过两个交互实验,分别验证了设计的有效性。
实验的结果和分析表明,本文实现的支持可持续使用的设计能够提高系统日常使用的便利性,能够方便用户在日常健康的场景下使用;
而本文实现的增加系统解释性设计,能够在目前算法的结果不完美的情况下,提高用户对系统的信赖程度和对结果的接受程度。

\end{abstract}

\begin{abstract*}

With the rapid development of economy, people nowadays have a faster life pace and growing concern on health issues. With the increasing advancement and popularity of mobile devices, more and more mobile applications for everyday health are being explored, including health tracking applications based on Face Reading Technologies (FRT). However, so far, while FRT based health tracking applications are technically feasible to be used on mobile devices in everyday settings, so far their explorations are mainly limited to clinic settings. At the same time, although there is extensive work on everyday health management in the field of human-computer interaction (HCI), the work mainly focuses on persuasiveness for user health behavior change and how to help them better manage chronic diseases. It is still underexplored how FRT based health tracking applications should be designed for everyday health.
To address such a gap, this thesis explores how to design FRT based health tracking applications to be integrated in everyday settings. The major work is as follows: 

1.	Qualitative research on face reading technologies in everyday health using technology probes. We conducted a qualitative research to explore the issues of face reading technologies in everyday health and proposed our design implications: Sustainable Everyday Use of Health Technologies, Transparency for Reliability Assessment and Striking a Balance for Everyday Use. 

2.	A platform is designed and implemented for interactive experiment of face reading technologies. To improve the efficiency of subsequent experiments and simultaneously solve the stability problems caused by the diversity of user equipment, an experimental platform was designed and implemented for rapid experiments of face reading technologies. 

3.	Design implications of Transparency and Striking a Balance for Everyday Use using the platform. We introduced new designs of solving Transparency for Reliability Assessment and Striking a Balance for Everyday Use, and verified them by experiments. In the process of implementing interpretable design, we explored how to explain the principle behind the system and the impact of user’s choices on the final result.Results show that our designs can improve the convenience of the system and increase users' acceptance of the 'imperfect' results.


\end{abstract*}

% \begin{notation}
%   $x$                          & 坐标        \\
%   $p$                          & 动量        \\
%   $\psi(x)$                    & 波函数      \\
%   $\langle x |$                & 左矢(bra) \\
%   $| x \rangle$                & 右矢(ket) \\
%   $\langle\alpha|\beta\rangle$ & 内积        \\
% \end{notation}
% 符号表
% 语法与 LaTeX 表格一致:列用 & 区分,行用 \\ 区分
% 如需修改格式,可以使用可选参数:
%   \begin{notation}[ll]
%     $x$ & 坐标 \\
%     $p$ & 动量
%   \end{notation}
% 可选参数与 LaTeX 标准表格的列格式说明语法一致
% 这里的 “ll” 表示两列均为自动宽度,并且左对齐
% \includepdf{cover.pdf}
\mainmatter



% 1.背景介绍
\chapter{绪论}

\section{研究背景}

长期以来,人脸在人与人之间相互识别,情感交流等方面起着重要作用。
随着人工智能,特别是基于深度学习的计算机视觉技术的发展,深度神经网络模型在各大竞赛中已经超越了人类水平。
在这个背景下,基于面部图像分析的读脸技术(Face Reading Technologies)在日常生活中的应用越来越广泛。
例如,通过将人脸图像与人脸库中图像进行比较来识别人脸的身份的人脸识别技术 \cite{Zhang2016Joint}\cite{Schroff2015FaceNet},已广泛应用于考勤管理,访问控制和安全性的所有领域(Apple Face ID)。
此外,还有相关研究探索了使用读脸技术进行情感识别\cite{corneanu2016survey},识别面相 \cite{Li2007Online}\cite{Tempark2012Chinese},甚至可以用于智商检测\cite{Kleisner2014Perceived}。
随着读脸技术不断进步和日益普及,其在日常生活中的应用领域也越来越广泛。

面部除了用于识别和表达情绪之外,也是反映一个人身体健康和精神面貌的重要部位,能够透露出各种健康状况甚至体内疾病的迹象。
在现代医学领域,当医生评估患者的总体体质并得出可能疾病的初步假设时,直接观察面部是医学诊断的一部分\cite{Clifford2006Shortliffe}。例如,患有肝炎和其他肝脏问题的人的脸或眼睛可能带有黄色的色调\cite{Li2008Therapeutic}。
此外,中国文化历史悠久,在长期的积累中,中医方面有着丰富的面部特征和身体状况的面诊理论,看面相在中国也有深厚的文化基础。
中医有经验充足的面诊理论基础,大量学者也为中医面诊信息化做了实证研究。
中医认为,内脏的病变可以反映到体表;相反,通过对外部的诊察,也可以推测内脏的变化。面部是最快表现脏腑病症的部位,通过观察面部的颜色、形状、五官状况等,可以快捷大致地诊断脏腑疾病。
在中医领域也有大量关于面部特征和疾病关系的研究,如面色和乙肝关系的研究\cite{杨宏志2007慢性乙型肝炎肝硬化中医面部五色诊断与临床病理的相关性研究}、基于面部诊断黄疸\cite{艾英1998黄疸病人面部色泽定量实验研究}等。
面部还可以指示一个人的整体心理和情感健康状况,显示出疲劳或疲劳的状态。总体而言,通过观察一个人的脸部可以了解很多关于人的健康的信息。
特别的是,面诊是中医四诊(望闻问切)的重要组成部分。基于中医理论,目前已开发出多种用于临床目的的计算机辅助工具完成健康诊断,包括面诊仪\cite{Liu2014Computerized},舌诊仪\cite{Wang2004An}和脉诊仪\cite{Shu2007Developing}。然而,直到最近,信息技术的进步和移动设备的便携性增强才将其中一些工具变成了日常用户可使用的设备。例如,各类可穿戴和便携式的脉冲读取设备,例如金姆健康推出的指尖脉搏血氧仪———金姆脉诊仪\footnote{http://www.jinmuhealth.com},已商业化用于日常健康监测。

于此同时,移动设备近几年也在迅速发展,无论是手机拍照的清晰度还是硬件的计算能力都有了很大的提高。由于移动操作系统本身的优越性加上硬件技术的迅速发展,移动设备的计算能力和存储能力不断增强,移动设备可以处理越来越复杂的任务,在很多方面逐渐取代传统PC设备。
由此看来,相信不久的将来,手机上用于日常健康管理的成熟的面诊应用将成为可能。

但是,和诊所环境相比,日常健康场景下主要有以下几个特点需要考虑: 

(1) 长期使用。日常健康追踪对于用户来说,是一个长期而持续的一个过程。 在日常场景下,如何吸引用户继续使用、如果提高日常使用场景下交互的便捷性、如何引导用户改变自己的生活方式,提高生活质量等是需要考虑的问题。

(2) 设备的多样性。 医院或者诊所环境下,用户是在专门的设备下进行健康诊断。而在日常场景下,所有操作都在用户的移动设备上完成。用户的移动设备具有多样性,从移动操作系统到硬件计算能力和存储空间的都存在很大的差异性。

(3) 用户的多样性。与医院或者诊所环境下不同,用户在使用过程中是没有专业的医生或者有可询问的技术人员,帮助用户理解诊断过程和诊断结果难度更大。

目前关于面诊技术的研究应用场景主要还是集中在诊所环境下,如何设计系统将面诊技术整合到日常健康实践中方面,它仍然没有得到充分的研究。



\section{本文研究内容与创新}

本文以日常环境下中医面诊应用的交互技术作为研究问题,
使用定性研究的方法设计交互实验,深入探索了目前基于面部诊断应用在日常使用场景下的一系列的交互问题,希望能为将来成熟的面诊应用提供交互方面的指导。

本文的主要研究工作有以下几点:

(1) 对基于面诊养生应用在日常使用的场景下进行了定性研究,选用云中医作为技术探针设计并开展实验,探索了日常使用环境下面诊应用的特点以及待解决的问题,并提出了三条对应的日常场景下面诊应用的设计原则。

(2) 为了支持后续的深入研究,设计并开发了一套支持后续面诊交互研究的实验平台。该实验平台提供了版本管理,日志管理,模型管理,问卷关联等功能,方便接入其他新的人脸模型,快速开展人机交互的实验,并降低了系统对用户设备类型和性能的要求。

(3) 在第二个工作点的基础上,利用实验平台,设计并实现了两个实验用的原型系统,分别对第一个工作点提出的设计原则进行了验证实验。实验的结果和分析表明,本文提出的提高可用性的设计规则的确能够提高系统使用的便利性,能够满足用户日常使用的需求;而本文提出的增加系统透明性的设计原则,的确能够在目前系统算法结果不完美的情况下,提高用户对系统的信赖程度和对结果的接受程度。


\section{论文结构}
本文按照一共组织了六个章节,具体的组织结构如下:

第一章,介绍了本文工作的研究背景,分析了日常使用场景下的特点,最后介绍了本文的主要工作和创新点。

第二章,介绍了和本文工作相关的研究情况,分人机交互领域的日常健康交互技术和诊断技术信息化两个方面分别进行了相关的介绍。

第三章,就本文的研究问题,采用技术探针的方法进行了定性研究,设计了调研问卷和深度访谈。通过对访谈数据的仔细分析,总结出了日常场景下面诊技术的特点,发现了其中存在的问题,并给出日常场景下面诊技术的设计规则。

第四章,为了方便后续的研究工作,介绍了本文设计并实现的用于面诊交互实验的实验平台,该平台提供了日志管理,问卷关联,模型管理,任务管理等功能,并在客户端通过重新设计交互界面,解决了第三章提到的影响用户使用的交互可用性问题。

第五章,在实验平台的基础上,对定性研究提出的结论进行了两个验证实验,验证了第二章工作的有效性。

第六章,总结了文本的研究内容和不足之处,并对未来工作进行了展望。
% 2.相关工作
\chapter{相关技术和方法}
本文已在绪论\ref{subsec:面诊系统}中介绍了当前面诊系统现状,本章则关注于介绍本文用到的相关技术和方法。
在本文研究内容中,首次将技术探针的方法用于研究日常场景中的面诊系统设计问题,因此第一部分介绍技术探针相关知识及数据获取方法;
其次本文提出了通用可拓展的面诊系统设计并进行了实现,因此第二部分将介绍面诊系统相关技术;最后部分介绍系统设计中用到的方法和概念。

\section{技术探针}
% 为什么要用技术探针
日常场景下的面诊系统设计,涉及到用户行为模式分析,日常场景存在环境复杂,难以获取用户真实想法等挑战。

技术探针有一套激发用户参与设计、获取复杂日常场景下用户数据的方法论,本小节将简单介绍本文使用的技术探针及数据获取方法。

\subsection{技术探针简介}

通常,典型的人机交互领域中研究系统设计是访问用户的家庭,然后设计并实现一个系统让用户评价。
在日常场景下,这种设计方法缺点非常明显\cite{Hutchinson2003Technology}:
(1)没有激发用户的思考,可能无法反映用户的实际需求和掩盖在家庭内部之间更加有趣的设计。
(2)在设计和实现系统过程中,没有提供真实的长期使用场景,用户缺乏参与性。

\begin{figure}[h]
    \centering
    \includegraphics[width=6cm]{images/user_study_hard.png}
    \caption{日常场景中用户研究的难点}
    \label{fig:user_study_hard}
\end{figure}

如图\ref{fig:user_study_hard}所示,在日常场景中,用户的个人爱好、性格,家庭环境都不尽相同,家庭成员之间也互相联系。特别是考虑到日常长期使用场景,传统的研究方法容易破坏真实环境,并不能激发用户内心真正的需求和积极参与性。
在复杂的日常和私人环境下,了解用户对于面诊交互技术的需求和态度更加具有挑战性。此外,在研究某类技术新的场景时,如果该技术没有被广泛使用,直接对用户进行访谈无法获取相关数据。

% 介绍什么是技术探针
为了避免上述的问题,Hutchinson等\cite{Hutchinson2003Technology}提出了一种名为技术探针的研究方法:利用快速开发出来的、功能并不完善的简单系统,让用户在实际使用探针的过程中,鼓励用户大声表达自己真实的想法,在草稿上画出自己心中的系统,参与设计过程。
在人机交互中,技术探针是一种探索未知事物的工具。基于技术探针的方法主要用于系统设计的早期阶段\cite{turmo2020training},也经常被用来为日常使用的健康技术的设计提供信息。
目前已有不少成功使用案例,例如通过技术探针的设计方法,Papi等\cite{papi2015knee}利用可穿戴式膝关节技术探针,探索了在家庭环境下如何设计用于骨关节炎患者的康复工具;同样,Singh等\cite{singh2017supporting}通过1-2周的用户研究,利用技术探针,研究了如何设计慢性疼痛患者使用的日常可穿戴设备。

% 技术探针与原型的不同点
\subsection{技术探针特点与要求}
% 目前已有工作的不足之处

原型系统是在软件工程中,为了方便所有参与项目的成员如客户、开发者对系统功能达成共识,从而构建的最小化的系统模型。
而技术探针通过一个功能简单的系统,让用户在真实的场景中使用,鼓励用户大声说出自己的想法,并在纸上纪录下来,从而收集用户真实反馈,同时引导用户参与设计的过程。

为了帮助更好地理解技术探针,此处通过对比一下人机交互领域中的技术探针和计算机领域中的原型系统,以此说明为何以往的基于原型系统的面诊系统研究会在系统设计层面存在不足之处,同时体现技术探针的特点和要求。
与原型系统相比,技术探针的不同之处及要求在于:
\begin{enumerate}
    \item 系统定位。原型系统的目的是实现系统的几个基本功能,给用户介绍系统目前的能力,让用户体验系统。技术探针的定位是一个采集信息的工具,用来帮助确定未来的设计方案。因此对于技术探针来说,研究者可能一开始并不知道该系统应该如何设计,所以在实现一个技术探针时,功能应该尽可能简单,减少模块分层,把设计过程让用户自由发挥,不限制用户的使用方式。

    \item 功能可用性。对于原型系统来说,系统中的几个功能需要保证正常可用,才能算是一个合格的原型系统。但是对于技术探针说来,技术探针在设计的时候,是允许特意出现非常反人类的设计或者功能不符合预期的情况,以提前激发用户发表自己的设计意见。

    \item 与设计方案的关系。原型系统和技术探针,与设计方案的关系不同。技术探针出现在设计阶段之前,用于获取信息指导如何设计。原型系统在设计阶段之后,原型系统根据设计方案进行实现。
\end{enumerate}

此外,技术探针通常需要记录用户的操作日志,用于分析用户的行为,帮助产生新的设计;原型系统虽然也会记录日志,但用户操作日志通常用于记录信息与排查错误。

\subsection{数据获取与分析方法}
在计算机领域,常见的用户数据获取是通过大量统计数据,如大批量结构化的问卷、用户行为日志等,然后在分析过程中,通过提取自变量、因变量做变量的相关性分析得到最终的结论。
在人机交互中,通常采用获取数据的方法是通过资料分析、案例分析、直接观察、用户访谈等方式获取研究数据。
其中,深度访谈是人机交互领域中收集数据最常用的一种方式,通过开放式、半结构化的面对面访谈,能够深度了解调查者内心的客观想法,避免研究者的主观判断。
该方法在心理学与社会学领域被广泛应用,能够发掘出用户更细节的需求和想法。
同时,在挑选了足够的代表性采访对象的前提下,该方法相对地对数据量的要求不高,在连续访谈得到的结果相对稳定后可停止访谈\cite{cleary2014data}。
本文使用两者相结合的方式:在使用技术探针进行实验之后,首先通过分析日志和问卷,得到大致的结论,然后通过面对面深度访谈发掘更深层次的原因,得到最终的结论。

\section{云中医诊断模型}
本文采用技术探针的方法研究面诊技术在日常场景应用问题,因此在模型选取方面更加关注其面诊流程覆盖度以充分获取用户的数据。
云中医是复旦大学与上海中医药大学为社区与诊所环境共同研发的面诊系统,云中医的诊断模型不依赖特定的设备,实现了基本的体质诊断功能,准确率已经满足技术探针可以使用的需求,
且属于校内自己开发的平台,因此本文最终选择在云中医的基础上进行用户研究。

\begin{figure}[htb]
    \centering
    \includegraphics[width=15cm]{images/cloud_med3.png}
    \caption{模型处理流程}
    \label{fig:cloudmed1}
\end{figure}

如图\ref{fig:cloudmed1}所示,技术探针所用模型以面部图片、舌部图片、问诊问题作为输入,通过特征提取后由诊断模型与规则系统得到最终的体质和健康得分。
本文主要工作之一是研究面诊技术在日常场景下如何应用,利用云中医作为技术探针发现了目前面诊技术存在的各方面的问题并提出设计策略。
云中医中诊断模型详细内容由于篇幅较长,同时为明确与本文工作的关系,统一放在附录\ref{ch:appendix}中介绍。

下面简单介绍部分技术探针所用到的相关技术。
\subsection{特征提取}
颜色模型是用数值描述颜色的数学模型,当前主要使用的颜色模型有LAB, RGB、HSV、YCbCr等,
如HSV颜色模型中颜色由色相(H)、饱和度(S)、强度值(V)三个维度组成\cite{dhivakar2015face};而LAB颜色空间中L表示亮度通道,A与B分别代表红绿与蓝黄颜色通道。
LAB颜色空间是一种基于生理特征的颜色空间,常被用于人的肤色特征提取。

Haar级联检测器通常用于提取边缘特征比较明显,具有固定图像特征的兴趣区域,如从背景图中检测出人脸、舌部矩形区域等\cite{2001Rapid, 2007Learning}。
Haar特征可用于描述图像中的边缘、中心以及线性特征,通过级联分类器对图像的Haar特征值窗口扫描完成对应的目标检测。

此外,基于混合高斯模型的方法\cite{Hu2016Robust}也常用于不规则区域的提取。
混合高斯模型将图像抽象为多个高斯模型的线性叠加,可用于建立肤色区域建模。
在完成兴趣区域建模后,通过计算图像中每个像素点数据属于模型分布的概率,可将兴趣区域从背景中提取出来。

\subsection{诊断模型}

早期的面诊系统一般通过诊断规则实现。诊断规则即通过专业的医学知识编写的一系列规则,是通过分析大量的临床医疗数据以及现有的结论,将诊断特征与诊断结果的关系进行量化\cite{牛欣2011中医四诊合参辅助诊断关键技术的数字化、量化研究}。
由人工编写的一些固定规则,结合了人工的经验,能提供基本的诊断能力,但由于严重依赖规则、表达能力有限,目前主要用于减少其他模型的训练成本以及修正最终的结果。
目前各类诊所环境中使用的面诊仪,其主要用到的技术是通过上述的人脸检测、预处理和简单分类模型抽取特定特征,如面色、舌苔形状等,
然后系统通过一定规则提示医护人员相关信息,帮助医护人员判断健康情况。
如基于上文提到的颜色模型结合简单的规则,就能完成对慢性肾炎\cite{周小芳2019慢性肾功能衰竭患者虚兼湿浊证的面部色诊研究}、
冠心病\cite{陈聪2019冠心病痰瘀互结证面诊图像特征参数分析}等基本判断。

SVM\cite{cortes1995support}是统计学习方法中常用的分类模型,应用非常广泛。
SVM利用核函数将原始数据映射到高维空间,通过EM(Expectation-Maximum)算法求解一个超平面,完成对原始数据的分类,能够在小样本、非线性的情况下保持优秀的泛化能力。
将面诊问题抽象为分类问题,可使用分类模型抽取相关特征以及判断用户的体质类型、五脏特征等。
如陈淑华等\cite{chen2016facial}使用了SVM分类器\cite{cortes1995support}用于判断用户是否患有糖尿病、乳腺癌、胃炎等五种疾病,
在其数据集上判断健康与非健康能达到70\%的准确率,五种疾病平均准确率能达到79\%左右。

技术探针所用诊断模型由SVM分类器与诊断规则组成,但在系统设计上需要兼容其他诊断模型,下面做简单补充介绍。
近年来,诊断模型发展迅速,出现了大量基于深度学习的诊断模型,如Liang\cite{liang2020oralcam}在移动设备上实现了OralCam系统:
该系统使用深度卷积神经网络判断用户口腔图片是否有口腔疾病的五个特征,从而帮助用户提高口腔健康意识。
端到端(即直接将面部图片进行输入,不做手动特征提取)的模型设计往往需要大量的样本数据支持,而迁移学习能降低模型对数据量的要求。
Jin\cite{jin2020deep}等根据地中海贫血、甲亢病人脸部数据集,将在人脸识别数据集上预训练的卷积分类模型,迁移到面诊场景用于通过面部图片判断地中海贫血、甲亢病等,
其准确率达到90\%, 优于专业医生进行面诊得到的结果。

面对上述功能各异、更新迅速的诊断模型,本文希望提供一种通用可拓展的面诊系统设计,
在系统设计层面解决日常场景下的交互问题(如提供基于系统设计的可解释性而不依赖具体模型),但又同时能提供对各种诊断模型的兼容与接入,
从而达到诊断模型更新变化并不影响系统设计的目的。因此下一小节将介绍本文在实现一个通用可拓展的面诊系统中用到的技术。

%\subsection{特征提取方法}

% 人脸检测是面诊技术中预处理的基本技术之一,用于在图像中标定人脸的位置和大小。
% 人脸检测可能会遇到人脸过多、奇怪表情、面部朝向姿态、光照、分辨率、肤色等问题,特别是在非标准环境下,从复杂的图象中精确定位到人脸是非常有挑战的事情。
% 总的来说,人脸检测可以分为基于特征和基于图像两类。
%面诊技术中特征提取主要用在人脸检测和兴趣区域提取两个步骤中。
%
%人脸检测可以分为基于特征和基于图像两类。(1)基于特征的方法通过提取图像的特征,然后将特征与人脸特征信息进行匹配,这类方法主要代表有基于颜色模型的人脸检测\cite{dhivakar2015face}、ASM(Active Shape Model)\&PDM(Point Distribution Model)\cite{kumar2019face}、基于Gabor滤波器的人脸检测\cite{sharif2011face}等。
%如基于颜色模型的人脸检测通过将人脸图像映射到三位颜色空间进行简单判断。人的肤色是人脸中的一个非常重要的特征,使用人脸的肤色作为特征处理虽然准确率不高,但是速度快,计算简单。
%当前主要使用的颜色模型有LAB、RGB、HSV、YCbCr等,以HSV颜色模型为例,HSV颜色模型中颜色由色相(H)、饱和度(S)、强度值(V)三个维度组成\cite{dhivakar2015face}。
%通过HSV颜色模型,可使用简单的规则过滤出肤色需要满足的条件。本文使用的特征提取方法在提取面色特征与唇色特征时使用了HSV与LAB颜色模型。
%(2)基于图像的方法人脸检测方法直接将输入图像与训练人脸图像进行匹配,这类方法主要代表有特征脸方法(EigenFace)\cite{mulyono2019performance}、Haar分类器\cite{priadana2019face}、基于人工神经网络\cite{farfade2015multi}的方法等。
%这类方法不事先定义人脸的特征,而是采用训练数据的方式从数据中学习人脸的特征分布,在提供足够训练数据的情况下准确率相对基于特征的方法有了很大的提高。
%云中医所用模型采用了Haar分类器的方法用于人脸检测与舌部区域检测。

% \begin{figure}[h]
%     \centering
%     \includegraphics[width=8cm]{images/face_detcetion.png}
%     \caption{人脸检测 \protect\footnotemark}
%     \label{fig:face_plus }
% \end{figure}
% \footnotetext{https://www.faceplusplus.com.cn/}



% % \subsubsection{基于特征的人脸检测方法}
% 基于特征的方法通过提取图像的特征,然后将特征与人脸特征信息进行匹配,这类方法主要代表有基于颜色模型的人脸检测\cite{dhivakar2015face}、ASM(Active Shape Model)\&PDM(Point Distribution Model)\cite{kumar2019face}、基于Gabor滤波器的人脸检测\cite{sharif2011face}等。

% 基于颜色模型的人脸检测通过将人脸图像映射到三位颜色空间进行简单判断。人的肤色是人脸中的一个非常重要的特征,使用人脸的肤色作为特征处理虽然准确率不高,但是速度快,计算简单。
% 当前主要使用的颜色模型有RGB、HSV、YCbCr等,以HSV颜色模型为例,HSV颜色模型中颜色由色相(H)、饱和度(S)、强度值(V)三个维度组成\cite{dhivakar2015face}。
% 通过HSV颜色模型,使用简单的规则就可以过滤出肤色需要满足的条件:
% $$
% \left\{
% \begin{array}{c}
%     0 <= H <= 0.25 \\
%     0.15 <= S <= 0.9
% \end{array}
% \right.
% $$

% 基于Gabor滤波器的人脸检测则通过将人脸图像映射到特征图像进行对比。在图像处理中,Gabor滤波器常用于作为算子获取图像的边缘和纹理特征\cite{sahib2020hybrid}。
% 由于人脸有着独特的轮廓特征,因此Sharif等\cite{sharif2011face}提出了一个使用Gabor滤波器进行人脸检测的方法,
% 该方法使用40个Gabor滤波器为每个输入提供了40个特征图片,然后根据每个特征图片中强度最大的点与人脸库中对应的点进行距离比较,从而判断是否检测到人脸。

% 由上面两个例子可以看出,基于特征的人脸检测方法思想是尝试将人脸映射到另一个特征维度的以获取人脸的不变特征如眼睛、眉毛、嘴巴等,对特征脸的数据要求也比较小,实现起来简单。
% 但当人脸图像受到光线影响、被部分遮挡时,基于特征的方法准确率将大幅度下降。

% \subsubsection{基于图像的方法人脸检测方法}
% 基于图像的方法人脸检测方法直接将输入图像与训练人脸图像进行匹配,这类方法主要代表有特征脸方法(EigenFace)\cite{mulyono2019performance}、Haar分类器\cite{priadana2019face}、基于人工神经网络\cite{farfade2015multi}的方法等。
% 这类方法不事先定义人脸的特征,而是采用训练数据的方式从数据中学习人脸的特征分布,在提供足够训练数据的情况下准确率相对基于特征的方法有了很大的提高。

% Labeled Faces in the Wild (LFW)是著名的人脸识别测试集,一共有6000组人脸图像,其中包含一半的正样本和一半的负样本。
% 其中传统基于特征的检测方法,在LFW数据集准确率大概是60\%, 基于深度学习的检测方法普遍可以达到99\%以上的准确率,超过了人眼判断的97\%的准确率\cite{sun2015deeply}。


% \subsection{面部兴趣区域提取}
%兴趣区域提取在面诊技术中主要指的是从人脸图像中切分并提取出面部(去除五官)、舌部、唇部等不规则形状区域的技术,
%常见的方法有基于混合高斯模型(Gaussian Mixed Model)\cite{Hu2016Robust}、SegNet\cite{Badrinarayanan2017SegNet}、TCMINet\cite{li2020tcminet}等。
%本文中选用的模型,使用了基于混合高斯模型的兴趣区域提取方法提取唇部区域\cite{Hu2016Robust},该方法将人脸识别后的人脸图像拆分为上下两个部分,在上半部分利用混合高斯模型构建肤色分布,利用肤色分布在下半部分抽取唇部和舌部。
%基于混合高斯模型相比其它方法在准确率相对高的前提下鲁班性强且实现简单。
% \begin{figure}[h]
%     \centering
%     \includegraphics[width=12cm]{images/roi.png}
%     \caption{兴趣区域提取}
%     \label{fig:roi}
% \end{figure}

% 介绍单高斯模型


% 与面部图像预处理不同,特征提取算法主要关于点在于如何从图像数据中获取量化的基础特征,如面色、唇色、舌苔类型等。

%\subsection{诊断模型}
%
%早期的面诊系统一般通过诊断规则实现。诊断规则即通过专业的医学知识编写的一系列规则,是通过分析大量的临床医疗数据以及现有的结论,将诊断特征与诊断结果的关系进行量化\cite{牛欣2011中医四诊合参辅助诊断关键技术的数字化、量化研究}。
%由人工编写的一些固定规则,结合了人工的经验,能提供基本的诊断能力,但由于严重依赖规则、表达能力有限,目前主要用于减少其他模型的训练成本以及修正最终的结果。
%目前各类诊所环境中使用的面诊仪,其主要用到的技术是通过上述的人脸检测、预处理和简单分类模型抽取特定特征,如面色、舌苔形状等,
%然后系统通过一定规则提示医护人员相关信息,帮助医护人员判断健康情况。
%如基于上文提到的颜色模型结合简单的规则,就能完成对慢性肾炎\cite{刘金涛2014基于数字化慢性肾炎湿热证面诊特征研究}、
%冠心病\cite{陈聪2019冠心病痰瘀互结证面诊图像特征参数分析}等基本判断。
%
%随着面诊算法研究的进一步深入,将面诊问题按照期望输出可以简化为分类、回归问题,
%从而利用模式识别及深度学习中的方法解决面诊问题。
%将面诊问题抽象为回归问题,可使用回归模型判断用户的健康指数以及患有某类疾病风险的概率;
%将面诊问题抽象为分类问题,可使用分类模型抽取相关特征以及判断用户的体质类型、五脏特征等,以下简单介绍部分相关技术。
%
%SVM\cite{cortes1995support}是统计学习方法中常用的分类模型,应用非常广泛。
%SVM利用核函数将原始数据映射到高维空间,通过EM(Expectation-Maximum)算法求解一个超平面,完成对原始数据的分类,能够在小样本、非线性的情况下保持优秀的泛化能力。
%陈淑华等\cite{chen2016facial}使用了SVM分类器用于判断用户是否患有糖尿病、乳腺癌、胃炎等五种疾病,
%在其数据集上判断健康与非健康能达到70\%的准确率,五种疾病平均准确率能达到79\%左右;
%
%近年来,诊断模型发展迅速,出现了大量基于深度学习的诊断模型\cite{liang2020oralcam, jin2020deep}。
%面对目前功能各异、更新迅速的面诊算法,本文希望提供一种通用可拓展的面诊系统设计,着重解决日常场景下的交互问题,但又同时能提供对各种诊断模型的兼容与接入,
%从而达到诊断算法更新变化并不影响系统设计的目的。
%
%在本文选用的面诊模型中,并未选取在特定疾病判断上准确率较高的模型,而是选取了更加符合面诊流程的基于诊断规则与SVM分类的方法,具体细节在附录\ref{ch:appendix}中介绍。

\section{系统设计相关技术}
微服务是当前发展迅速的服务架构\cite{jamshidi2018microservices},能有效减少系统耦合性,提高系统拓展性等。
为实现一个通用可拓展的面诊系统,本文并没有照搬整套微服务设计而是根据技术探针发现的问题针对性地做了实现,本小节介绍本文在系统设计上用到的方法和相关技术。

\subsection{读写分离设计}
随着系统的吞吐量增加,数据库的读写往往会成为系统的瓶颈。
读写分离将数据库拆分为读库和写库,读库只负责数据的查询操作,写库处理事务性的增删改操作。
读写分离的主要目的是避免数据在更新过程中收到数据的查询请求而导致的行锁,
以提高系统吞吐量。

在数据库层面实现读写分离能提高系统吞吐量,但由于存在主库到从库的数据同步,会导致数据时效性延迟,且增加从库有额外的计算和存储成本。
不依赖数据库的读写分离设计,系统在设计层面也应该尽可能遵循读写分离的设计,如在模块或者方法层面尽可能避免同时读写数据库同一行记录;
对系统进行拆分,将写和读拆分为两个系统,两个系统之前通过消息的方式进行通信,通过缓存提高读服务性能等。

\subsection{主从模式设计}
主从模式设计在处理可分解问题时非常常见,其主要设计思想是为了解决单计算实例处理问题性能有限的问题。
主从模式将系统设计成多个节点,一般分为一到两个主节点和多个从节点,通过主节点将任务进行拆分,拆分成多个可分开执行的子任务,
然后将子任务分配到从节点中运行,运行结束后再将结果汇总返回。

主从模式的优点在于任务的并行处理,提高整体性能,此外多节点的方式具有容错机制,提高系统稳定性。
在实现上,主从模式关注以下两个方面:(1)容错管理机制。当从节点崩溃或者上线时,其他节点需要感知,目前主要通过心跳包的方式实现节点鉴活,提供服务发现机制或利用现有的服务注册框架如Consul\footnote{https://www.consul.io/}、 Nacos\footnote{https://nacos.io/}等;当主节点崩溃时,其他节点可以完成主节点的选举和更换,目前有大量一致性算法如raft、paxos等。(2)服务间通讯方式。多个节点之前如何高效地完成同步、异步的通讯,目前实现方式有http/rpc调用,提供回调接口,消息队列,中心存储等。

\subsection{容器化技术}
容器化技术是Linux系统基于Namesapce和Cgroup等机制提供的内核轻量级的虚拟化技术,实现了服务间的资源隔离,方便快速构建部署服务。
如图\ref{fig:container}所示,和传统的虚拟机方式相比,容器化技术启动更快,资源占用更少,性能更好。目前主要的容器化解决方案有Docker及K8s。

Docker\footnote{https://www.docker.com/}是当前最流行的容器化解决方案,其有几个重要概念:镜像、容器、仓库。镜像提供了容器运行时需要的依赖、相关文件和环境变量等;容器与镜像的关系相当于进程与程序文件的关系:镜像成功启动后,就是一个容器;仓库为各类镜像提供了统一的管理平台。
基于以上设计,Docker提供了容器运行、镜像构建、镜像管理等基本功能,方便用户将快速部署代码,将系统服务化。

K8s(kubernetes)\footnote{https://kubernetes.io/}是基于Docker(但不局限于)的容器编排工具,提供对容器集群的部署、拓展、管理等功能,
基于K8s可以快速搭建出一个集群服务。

\begin{figure}
    \centering
    \includegraphics[width=10cm]{images/container.png}
    \caption{容器化vs虚拟机}
    \label{fig:container}
\end{figure}

模型服务化是指将算法和模型等打包成可调用的服务,
目前主流的做法就是利用Docker将模型需要的依赖和环境打包到Docker镜像中,以HTTP或者RPC的方式对外提供调用接口。

\section{本章小结}
在本章中,首先介绍了下一章将要用到的技术探针方法,然后介绍了相关人脸检测算法以及本文用到的特征提取算法、诊断模型等,
最后介绍了本文在系统设计中用到的容器化技术和相关系统设计方法。
% 3.前期用户调研
\chapter{基于技术探针的用户设计研究}

% 介绍为什么要做这个
为了研究日常使用场景下面诊技术的应用会遇到的问题,本文首先利用技术探针\cite{Hutchinson2003Technology}的方法进行了定性研究,通过分析访谈数据来探索和总结面诊应用在日常场景下的特点和存在的挑战。


\section{技术探针}

% 为什么要用技术探针
通常,典型的HCI交互设计方法是访问用户的家庭,然后设计并实现一个系统,再测试用户喜欢或者不喜欢。在日常场景下,这种设计方法缺点比较明显\cite{Hutchinson2003Technology}:

\begin{itemize}
    \item 没有激发用户的思考,可能无法反映用户的实际需求和掩盖在家庭内部之间“好玩”的设计;

    \item 在设计和实现系统过程中,没有提供真实的长期使用场景,用户缺乏参与性。  
\end{itemize}



% 介绍什么是技术探针
为了避免上述的问题,Hutchinson等\cite{Hutchinson2003Technology}提出了一种利用快速开发出来的、功能并不完善的原型系统,让用户在实际使用原型系统的过程中,鼓励用户大声表达自己异想天开的想法,在草稿上画出自己心中的系统,参与设计过程的研究方法。

基于技术探针的方法经常被用来为日常使用的健康技术的设计提供信息。例如基于技术探针的设计方法,Papi等人\cite{papi2015knee}利用可穿戴式膝关节原型,探索了在家庭环境下如何设计用于骨关节炎患者的康复工具;同样,Singh等人\cite{singh2017supporting}通过1-2周的用户调研,利用技术探针,研究了如何设计慢性疼痛患者使用的日常可穿戴设备。

\begin{figure}[h]
    \centering
    \subfigure[主界面]{
        \includegraphics[width=4.5cm]{images/main1.jpg}
    }
    \subfigure[问诊]{
        \includegraphics[width=4.5cm]{images/main2.jpg}
    }
    \subfigure[健康报告]{
        \includegraphics[width=4.5cm]{images/main3.jpg}
    }
    \caption{技术探针}
    \label{fig:main}
\end{figure}

为了方便进行快速用户调研,我们选用了云中医\cite{Zhang2018Study}作为技术探针。
云中医是复旦大学计算机学院张文强老师和上海中医药大学合作开发的一款面诊应用,虽然它的主要应用场景是诊所内,且由于准确率的原因,结果只能做参考,但是它实现了面诊,舌诊和问诊的基本功能,符合技术探针的要求。

云中医是一个手机上的健康诊断应用程序,旨在提供一个方便的平台的自我诊断和健康管理。该应用以中医面诊、舌诊、问诊理论为指导,在手机上模拟实现了诊断的过程:
用户需要依次对自己的面部和舌头进行拍照,回答一些与自己健康状况相关的问题,最终会收到一份完整的健康报告和一些健康建议。

云中医app的主要界面及使用流程如图 \ref{fig:main}所示:用户进入诊断页面之后,会看到三个区域:面诊、舌诊、问诊。面诊和舌诊通过拍照或者上传图片完成,问诊是通过依次回答13个问题来完成。
依次完成面诊、舌诊、问诊之后,系统会给出用户的体质和健康分数,并且给出对应的健康建议。

\section{用户实验}
 
% 介绍定性研究
本实验采用的是定性研究方法。定性研究是广泛应用在用户需求调研、人机交互、心理学等领域的研究方法 \cite{崔岩2011统计分析中的定量与定性研究}, 常见的收集数据的方法包括深度访谈、焦点小组访谈、日记、观察法等 \cite{李晓凤2006质性研究方法}。

% 研究方法 -> 研究过程->数据采集->数据分析->得出结论
为了尽可能详细地理解用户在日常健康场景下使用面诊技术的各种行为和态度,我们采用半结构化的深度访谈来收集定性数据\cite{DiciccoThe}。深度访谈是定性研究领域中收集数据最常用的一种方式,通过开放式、半结构化的面对面访谈,能够深度了解调查者内心的客观想法,避免研究者的主观判断。

\subsection{实验过程}
\begin{figure}[h]
    \centering
    \includegraphics[height=15cm]{images/poster.png}
    \caption{招募海报}
    \label{fig:poster}
\end{figure}

% 人员招募

\begin{table}
    \centering
    \begin{tabular}{llll}
          \toprule
          编号 &	性别 &	年龄 &	职业 \\
          \midrule
          参与者1 &	男 &	20多 &	博士生 \\
          参与者2 &	男 &	20多 &	研究生 \\
          参与者3 &	男 &	20多 &	研究生 \\
          参与者4 &	女 &	50多 &	医院护理工 \\
          参与者5 &	男 &	40多 &	办公室职员 \\
          参与者6 &	女 &	40多 &	小学老师 \\
          参与者7 &	女 &	50多 &	办公室职员 \\
          参与者8 &	女 &	19岁 &	本科生 \\
          参与者9 &	男 &	20多 &	办公室职员 \\
          参与者10 &	女 &	40多 &	办公室职员 \\
          \bottomrule
    \end{tabular}
    \caption{参与者}
    \label{tab:part}
  \end{table}
  % 实验过程
我们通过社交媒体发布海报(如图 \ref{fig:poster} 所示)招募志愿者参与实验。通过宣传,我们最终招募到了10位感兴趣的志愿者,志愿者资料如表 \ref{tab:part} 所示。在招募到合适的用户之后,我们会对每一个用户进行一次采访,主要是为了了解用户信息,介绍技术探针。采访是在征得志愿者的同意下,全程录音的情况下进行的。

% 在被试人员对此应用感兴趣,愿意尝试的前提下,尽量使得招募到的人员在年龄、健康状况、教育背景、工作性质等方面具有多样性。不同的年龄层次可能会导向对健康的不同关注角度(中老年人较为关注的养生知识逐渐在青年群体中流行起来)不同的工作性质使得人们重点关注的身体部位不同(白领普遍关注的肩颈,特殊行业的“职业病”)不同的教育背景使得人们对用技术手段监测健康信息的接受程度也有差异。

接下来两周时间,让用户在日常场景下使用应用。在第一星期,我们不做任何干预;到了第二星期,要求用户每天至少使用一次。

在用户使用结束后,我们会进行一次结束后的回访,了解用户在使用中遇到的问题和他们的想法。

我们希望通过此次调研了解到使用面诊应用的用户群体有哪些?用户期望通过面诊应用来获取哪方面的信息?以及在方便用户日常使用上,有哪些待解决的技术难点?为了回答这些问题,我们通过分析采访数据进行探索。

\subsection{数据分析过程}

\begin{table}[h]
    \begin{tabular}{ll}
        \toprule
        编号 & 问题 \\
        \midrule
        1  & 您的年龄?职业背景?   \\
        2  & 您目前健康状况如何?平时感觉身体哪里不舒服吗(慢性病,容易疲劳,关节疼痛等等)?   \\
        3  &  您有使用健康状况评估的技术吗?若有,都有哪些技术?平时如何使用的?使用体验如何?  \\
        4  & 有看多中医吗?是什么原因看的中医?在哪里看的?效果如何?   \\
        5   & 对中医的面诊、舌诊有了解吗?若有,通过什么渠道了解的?   \\
        6  & 平时会留意自己的面色,舌苔吗?会留意其他人的面色,舌苔吗?\\
        7   & 您对这个APP感兴趣的点主要在哪里?\\
        \bottomrule
    \end{tabular}
    \caption{第一次采访问题大纲}
    \label{tab:inteview_questions}
\end{table}

介绍性的问题大纲如表\ref{tab:inteview_questions}所示,我们会根据访谈的深入不断迭代问题大纲。
我们将每个用户对应的采访和回访的录音转录成文字,最后整理后一共有56页的采访数据。由于采访数据量比较大,本文采用主题分析法(Thematic Analysis)进行数据分析工作,大致分为以下几个步骤\cite{SchwandtQualitative}:

\begin{enumerate}

    \item 精简数据。对访谈记录中的数据进行删减,只提取关心的内容。

    \item 展示数据。对保留的内容进行分类,从中提取关键词。然后再根据关键词提取出新的分类,由新分类返回到第一步循环提取相关的内容。反复迭代步骤(1)和步骤(2), 直到无法提取出新的内容和新的关键词为止。

    \item 归纳主题。主题是对关键词所反映的内容模型的抽象化,在访谈数据中重复出现并且具有共通性,通过主题能够将提取出来的关键词联系起来对访谈的内容进行解释。
\end{enumerate}

\section{研究发现}

通过对采访数据进行分析,我们的用户调研发现如下:
\subsection{潜在的使用场景}
通过调研,我们发现面诊类应用在日常使用环境下,用户反馈有两种潜在的使用场景:

\subsubsection{1.了解身体状态}

对于那些已经有健康问题的用户或者日常有养生活动的用户对云中医更加感兴趣,并且知道如何在日常生活中使用它。

参与者1他从小就是一个病骨头,总是生病,所以经常看中医。参与者1说道:\myfont{像我这种生了病或者健康状态不好的人,是可以用云中医应用来日常使用来检测疾病的。}
参与者10对云中医应用也表现出浓厚的兴趣:\myfont{这些诊断结果和建议对我很有用,因为在手机上做诊断很方便,我一有空就会用一用。比如我今天早上有空,我就打开用了一下。}
参与者3是一个对自己健康状态很看重的人,他说道:\myfont{因为它可以进行面诊和舌诊告诉我结果,并且给出了一定穴位按摩的建议,我觉得很有用。}

大多数参与者使用云中医不只是为了了解自己身体的状况,更是为了能够及时调整自己的生活方式。
如参与者3有一次早上感觉非常疲倦,于是他拿出云中医应用对自己进行了诊断,果不其然应用给出的分数是自使用以来最低的65分,他觉得这类应用非常重要:\myfont{作为一名学生,久坐是不可避免的。当我感觉疲倦的时候,我就打开云中医看一下,然后根据结果来决定是不是要去锻炼一下。}
参与者1也提到云中医可以帮助进行健康决策:\myfont{云中医可以帮我了解自己的身体状况,这样我就可以判断自己当天是否可以加班了。}


\subsubsection{2.医患沟通的工具}

正如其他健康跟踪类技术的发现类似,参与者希望云中医能够提供医生和患者沟通的渠道,这样医生可以更好地获取病人的日常健康信息。

参与者8说道:\myfont{这类应用可以和医生合作,这样医生就可以获取患者的健康状况并及时发现问题}, 她进一步建议:\myfont{可以把每次的诊断结果保存到个人档案中,这样医生可以方便地获取到患者的健康信息,这样医生就可以知道患者在就医期间的日常健康状况了。}

现在的知名中医,很多都有自己的或者医院的微信公众号,患者可以私下通过公众号发消息和医生沟通。虽然这是一种不错的尝试,但是这种沟通方式目前限制还是比较明显:

\begin{enumerate}
    \item 微信不是专业的医患沟通的工具,可能会泄露患者和用户的个人信息。

    \item 通过微信沟通,医生需要每次通过微信进行问诊、面诊、舌诊。对于医生来说产生大量的重复性劳动,也无法看到用户近期的甚至更久之前的情况。

\end{enumerate}

面诊类应用可以设计成通过导出用户的诊断记录,同时屏蔽自己的敏感信息,让医生可以通过应用方便地看到自己每次记录下来的面部舌部和回答问题的答案。

\subsection{自适应性}
日常环境和诊所环境不同,日常环境下需要用户长期使用。但是,大部分的参与者在使用云中医一到两次后就不再使用了,其中一个重要的原因就是云中医缺乏自适应性的能力。

比如参与者1在采访中指出:\myfont{无论我什么时候拍照,它都是问我同样的问题,我会感觉很无聊。}
同样,参与者2说:\myfont{每次都重复地检查很枯燥。}参与者6解释了她不继续用云中医的原因:\myfont{一个人的健康状况,在短时间内的变化不会太大,所以我就没一直用了。}
而参与者7虽然在持续使用,但是也不愿意每次看到相同的信息:\myfont{它每次都给我相同的健康结果,我怀疑它的准确性。}

参与者在提出问题的同时,也给了对应提高系统自适应性的建议,主要体现在两个方面:

\subsubsection{1.自适应用户信息}

从调研反馈来看,我们发现因为云中医在问诊的时候,一直问用户同样的问题,而实际情况是短期内来说,大部分问题的答案是不会发生改变的,这样缺乏用户信息的自适应性会让用户使用起来觉得很繁琐。

在这次的用户调研过程中,在诊断应用的场景下,我们发现用户强烈希望系统能够记住他们的用户信息并对下一次诊断进行相应的调整。
比如一些参与者抱怨说每次去回答问诊的问题非常地繁琐, 参与者2说道:\myfont{问诊那一部分设计地不好,每次问题都是固定的,而且太笼统了。},然后他建议这些问题应该更加个性化:\myfont{我认为问诊的问题可以根据之前问的问题更加地具体。}
参与者10则希望系统能够追踪用户的健康状况:\myfont{我希望它有后续的追踪过程,不仅可以评估当天的健康状况,也可以在接下来的几天内,跟踪健康问题并动态地调整建议。}
参与者3也有类似的想法:\myfont{目前系统只有诊断的功能,我希望系统能够接受用户的反馈,比如记录系统给出的健康建议是否有效,这样可以调整健康建议的结果。}

\subsubsection{2.自适应环境信息}

系统给出的养生建议也是固定的,并没有考虑到用户当前的工作状态和环境因素。中医认为,与环境变化和谐相处对于保持健康很重要,特别是在季节变化的时候,即中医中所说的\myfont{天人合一}。

参与者6建议:\myfont{养生建议应该更加丰富一些,比如考虑季节和节气的变化。} 
参与者10希望系统能健康建议和当前天气综合考虑:\myfont{那么热的天气叫我出去运动? 它应该考虑到我个人的健康情况和当前的天气情况。}

\subsection{实用性}
在调研过程中,我们发现并不是所有的参与者都能接受健康报告的结果,主要原因有以下两个方面:

\subsubsection{1.中医术语}

云中医应用是和上海中医药大学一起合作研发出来的,健康报告的文字使用的是标准的中医术语。和平时就医环境不同的是,在日常环境下,用户在使用的时候,周围是没有专业的医生可以询问的,因此通俗易懂的解释中医术语是非常重要的。
在调研过程中我们就发现很多参与者很难理解系统给出的健康报告,特别是在年轻群体中,新一代年轻人对中医的接受程度比较低,这个问题就更加严重。
参与者2是在读研究生对健康报告的里“气”的概念就不是很了解,他提到:\myfont{我对中医知之甚少,它说我气虚,但是什么是气虚我都不知道……}

参与者8也觉得自己对中医知识了解有限,对结果不是很理解:\myfont{分数下面有提到五脏的概念,但是其实我连五脏指的是哪五个器官不是很清楚,可能包括肝脏、心脏和肾脏?}
虽然中医在中国的历史悠久,但是并不是所有的人都清楚中医中的常见概念。 如果系统是设计给普通用户日常使用的话,就需要考虑到用户的中医知识素养了。

解决此类问题的方法之一,是把中医术语和常见的词语进行关联。比如参与者1遇到这样一个例子:\myfont{我今天感冒了,但是系统却没有提示我感冒了,只是说我气血不畅,它为什么不直接告诉我感冒或者风寒呢?}

另一个解决方法则是通过更加生动的方式进行介绍,比如视频图片等。还是参与者1,他说:\myfont{健康建议里的按摩,可能直接给出一个按摩的视频,而不是只有文字和图片,有一个视频跟着做的话会更好。}

参与者还提出可以通过提供学习的机制,让用户了解晦涩的中医术语。
事实上在调研过程中,就有参与者把云中医当作一个学习中医养生的工具,例如参与者10提到:\myfont{我认为系统可以提供更加宽泛的养生知识,而不只是健康建议。我对健康建议中的穴位按摩就很感兴趣。}

\subsubsection{2.养生建议合理性}

不同职业的用户,日常的空余时间是不同的。空余时间是否充裕,是影响用户实践养生建议的关键。因此系统给出的养生建议,特别是需要花费较长时间的按摩熬粥之类大的养生活动,需要合理考虑用户的日常空余时间。

比如参与者2就指出:\myfont{作为一个住在没有厨房的公共宿舍的学生,养生建议中的食疗就是不切实际的。我不会做饭,而且有些食物用的材料太贵了,超出了我的承受范围。}
参与者9是一位刚参加工作的算法工程师,每天8点才下班,他反馈说:\myfont{我基本没有什么时间熬粥或者煎药了。} 参与者7表示:\myfont{每天都进行穴位按摩太费时间了,我坚持不下来。}

\subsection{敏感性}
\subsubsection{1.文化敏感性}

舌诊的时候,需要用户伸出舌头进行拍照,这个过程可能会给用于带来烦恼。
在中国文化中,伸出舌头可以表示厌恶或者无礼,并且大部分中国人是比较传统的。因此在很多情况下,在公众场合伸出自己的舌头被认为是不合适的。大部分参与者表示伸出舌头会让他们感到尴尬,或者是害怕自己因为伸出舌头在无意间冒犯到别人。
如参与者1认为,伸出舌头拍舌头的照片是很不雅的。参与者3表示自己在和别人拍照的时候都会感到紧张,使用云中医时都是私下使用。特别是年轻人来说,参与者8直白地指出伸出舌头这个动作太丑了,她只会在私人空间下使用应用,她甚至调侃说:\myfont{我觉得这个软件是太有意思了,因为它有拍舌头的各种照片,完全就是收集丑照的一大利器。}

不过有趣的是,对于年纪稍大的参与者或者老年人来说,对于伸出舌头进行诊断并不是非常在意。例如参与者10说:\myfont{我并不觉得尴尬,因为这是一种诊断。}, 参与者4对这一现象解释说: \myfont{对于像我这样的年纪的人来说,就不那么在乎了}。 

总而言之,这个发现提醒了我们文化敏感性限制了面诊类应用的日常使用范围。

\subsubsection{2.技术敏感性}

由于模型的计算需要对面部舌部图片进行颜色特征的提取,使其对外部因素如光照和手机摄像头的质量会更加敏感,已经严重影响了部分用户的使用。
参与者7在系统一次又一次提示检测不到照片中的人脸之后选择了放弃:\myfont{照片一直提示重拍,觉得麻烦就不用了。}
参与者8还特别反映了自己用的是vivo手机,拍照会受到相机自带美颜算法的影响: \myfont{但是你需要考虑,我现在用的手机是vivo手机,它的相机自带美颜的功能。你把照片传过去,那准确率不就下降了吗?}

由于目前技术的问题,环境和设备的多样性不仅可能降低最终结果的准确率,甚至会影响用户的正常使用。因此,在离开诊所环境下严格统一的光线和设备条件之后,将面诊应用日常环境下,从技术的层面来讲,需要考虑更多的因素。
或者说,在技术无法改进的前提下,我们应该如何设计系统来屏蔽这些环境因素和设备的差异性,也是一个需要考虑的问题。

\subsubsection{3.社交敏感性}

虽然有几位参与者希望分享它们的健康结果给其他人作为相互学习的机会,但是他们同时也担心因为涉及到面部照片和个人健康信息,可能会导致隐私泄露的问题。
如参与者10讨论了可接受的分享的范围:\myfont{我觉得,比如说有什么症状我可以分享。比如说我今天有哪里不舒服了,得到了哪个专家的帮助,然后经过一段调理,我觉得我有好转了。但是有些可能关系到自己个人隐私的事情,我还是有点不愿意分享的。}

大部分参与者虽然不太愿意分享自己的诊断结果,但是如果将分享的范围限制在家庭或者医生圈子等可信任的人之内的话,他们表示可以接受。不过有趣的是,我们发现和其他社交分享类应用不同的是,对于家人来说,他们更加愿意分享给陌生人,因为分享给家人的话可能会引起家人不必要的担心,如参与者3表示: \myfont{其实我觉得分享的话更倾向的是和一些陌生人分享,如果和亲人分享的话,就觉得反而让他们,但是反而觉得是一种负担,如果和陌生人分享,就会觉得这是一个这一个交流的,就是一个也算是一个社交的一种方式吧,同时也能提高自己的健康知识。}


\subsection{信赖与情绪的问题}
关于用户对于日常面诊技术的态度,我们在本次用户调研过程中也进行了探讨。在调研过程中,我们发现许多参与者对产生的结果持怀疑态度,从而导致他们不信赖系统。
例如当参与者发现系统给出的健康分数较低时,他们不会去反思是不是自己的生活方式不够健康或者其他原因,而是质疑系统的准确性。参与者3认为系统给他的低分数是因为系统的缺陷:\myfont{看到它结果不好的时候,我就第一反应就觉得这个app有问题。就是其实我本身并没有不舒服,但是它打的分很低。我就会想他是不是它面部检测的这个效果不好,还有它这个诊断的这一个整个的正确率也不好。}

在没有其他的评估机制验证结果有效性的前提下,参与者会倾向与根据自己的感觉去判断。一方面,如果结果和自己情况不符合,他们会产生不信任的情绪甚至放弃使用,如参与者2透露:\myfont{也不是很不舒服,我觉得我自己还挺好的,还是就老给我打那么低的分……我就真的我觉得我还没有那么严重,就不想用它了。};另一方面,他们认为结果是可靠地就会促进他们继续使用,如参与者5说:\myfont{基本和我医院看的说法差不多……我觉这个做的很好,符合我的实际情况。}但是我们需要注意的是,调研过程中,参与的的个人感受和看法可能并不能反映他们真实的健康状况。

此外,我们通过分析访谈数据还发现,系统的结果会对使用者的情绪产生影响,特别是分数很低的时候。例如对于参与者8来说,很低的分数对于她来说是不能接受的,因为她已经坚持健康作息很长一段时间了:\myfont{也不是吧,就是感觉受到打击了吧,因为那个时候我还是生活得非常健康的。期末考试那段时间我非常不健康作息的时候都能测到89,现在结果却这么低,真是见了鬼了。} 这也验证了之前的研究\cite{Toscos2013Designing}的结论,健康评测系统会引发用户的消极情绪,特别是当系统给出负面结果的时候需要非常注意。

一些参与者提出,他们需要了解结果背后的原理机制,以便知道如何对结果进行评估。如参与者1认为知道结果是如何得出的很重要:\myfont{它需要告诉我它的数据是怎么来的,否则我不太相信}。而另一位计算机专业的参与者则
希望通过了解背后的算法原理来判断它是否可靠:\myfont{我认为这个背后的模型有些过于简单。通过一些基本的面色、舌苔等指标的规则组合做出对健康的评判,似乎很难让人相信能非常贴切、准确地反映自己的健康状况。在图像方面感觉需要更复杂的模型来对健康状况和面色、舌苔等之间的关系进行发现,这样才使我信服,确实这个系统里的中医有其专业性,能对我的健康提出有益的建议。一个经验丰富的老中医可能需要自己一生的行医积累,才能从表面相同的症状分析出背后不同的病源,有针对性的开方子医治;从计算机专业讲,这可能需要用深度学习的复杂模型才能接近这个过程及达到的效果。} 很明显,将系统背后的原理和数据暴露给用户是帮助用户评估系统可靠性的一种方法。

\section{设计方案}

通过本次用户调研,我们可以发现在日常使用场景下,基于人脸识别技术的健康应用存在着大量待解决的设计方面的问题,而且大多数问题是和使用了面部图片进行面诊相关。为了帮助更好地设计日常面诊应用,我们总结之后认为系统应该遵循以下的设计方案。

\subsection{增加可变性}
在以往鼓励健康生活类交互技术的研究中,用户持续使用一直是一个重要的问题\cite{Clawson2015No} \cite{Epstein2016Beyond}。在本次研究中,也出现过大量的用户不继续使用的情况。
在所有的原因中,最重要的因素是没有出现新的东西,如每次都要拍脸,拍舌头,回答同样的问题,系统给出的也是同样的结果等。

本次实验的环境是要求用户每天使用一次,系统每次都是问同样的问题,然后用户健康状况变化不大的话,也会给出类似同样的结果。本次数据表明,用户很快会失去兴趣。

一方面,基于这个发现,我们建议在设计此类应用的时候,我们对把用户分为两类分别讨论:

\begin{itemize}

    \item 对于健康状况有变化的用户,我们应该允许用户跳过某些步骤,只让用户回答这次不同的地方,或者突出显示出本地和上次记录中出现变化的部分。这样能够减少用户每次诊断的时间,同时用户能追踪自己的健康变化。

    \item 对于健康状况没有变化的用户,可以加入上下文的信息让结果有稍微的变化。例如加入天气和气候的因素,给出用户不同的建议。这样不仅增加了用户对结果的新鲜感,也能让用户学习到天气和气候相关的知识。

\end{itemize}

另一方面,可以将健康知识以推送的方式显示给用户。如果知识库特别大的话,可以一次推送一部分的健康知识,这样也能增加系统的可变性。

\subsection{系统透明性}

和其他成熟的商业健康产品,如血压计,血糖仪不同,基于人脸的健康诊断技术仍是一项新兴的非常不成熟的技术。因此从相对权威的应用场景如诊所,转移到日常环境下时,系统的可靠性会被经常受到质疑。

本次研究数据表明,系统的可靠性或者说信任度也是影响用户继续使用的关键因素。对于本次定性研究使用的技术探针云中医,虽然它是由复旦大学计算机学院和上海中医药大学的专家使用深度学习,利用大量的数据训练而成,但用户在使用的时候,并不知道它背后的技术原理,只能通过自己的直觉来判断系统的准确性。

基于这个发现,我们建议系统在设计的时候,将背后的技术原理和开发背景透明给用户,从而增加用户的理解度,让用户更好地对结果进行评估。

当然,对于当前的人工智能模型,系统不是万能的,总是会存在误判。此类应用的设计目的,并不是为了取代专业的中医进行诊断,而且很多的因素也会影响最终的结果,如拍照时背景光线的强度,设备相机的分辨率,截取图片的区域等对结果影响很大。我们在进行透明化设计的时候,也可以把系统的局限性告知给用户。

\subsection{日常可用性}
基于人脸的诊断应用的特殊性在于,在这个过程中,用户需要拍脸和拍舌头。对于拍脸来说,在中国比较传统的社会环境下,大部分人是比较害羞的,没有拍脸的习惯,特别是在日常的场景下,在公共场合下拍脸;对于拍舌头来说,在别人面前伸出舌头拍照,是一件很不雅观的事情,影响个人形象。

正如本次实验数据所示,大部分参与者在使用的时候,都是在周围没有人的时候,或者在宿舍中进行。也就是说,让用户去拍脸拍舌头从技术上讲是没有实现难度,但是从社会从文化上来说,有些用户是难以接受的。

此外,在目前的互联网环境下,用户的可能会觉得自己的自拍有可能会导致隐私泄露的问题。用户在使用的时候会考虑到自己拍脸导致隐私泄露而拒绝使用。

基于上面的讨论,我们建议在设计的时候,在不影响系统使用的情况下,尽可能把面诊舌诊设计成系统中可选的选项,允许用户跳过某些步骤。

不过,在本次研究中,也有用户发表了完全不同的想法:有的用户愿意分享自己的面诊照片,认为这是一种和类似人群交流学习的机会。

\section{本章小结}


本章介绍了通过定性研究的方法,招募了一群参与者,设计并对参与者进行了深度访谈。
通过分析访谈数据,发现了日常场景下中医面诊应用的潜在使用场景,以及系统的自适应性、实用性、敏感性等问题,
同时从设计的角度给出了增加可变性、系统透明性,日常可用性等设计方案,为日常场景下的面诊应用的设计提供了设计方案。



% 可用性(striking a balance, 不需要拍照),透明性  怎么透明性可以提高

% 为了实现这个实现,我们设计了新的系统

% 可用性方面,我们做了以下的设计与实现

% 设计理念,设计方案:

% 把xxx放到一页,不需要强制顺序
% 为什么跨平台,提高

% 再写具体实现,为了方便研究,满足实验的需要。



% 4.系统实现
\chapter{系统实现}
根据上面的调研结果,可以改进的方面有很多。但是从用户日常使用的角度,影响用户持续使用最大的问题是可用性的问题和理解的问题。

可用性的解决方案主要分为两块:一个是提高系统的可用性;另一个是重新设计用户友好的交互界面,提升交互的体验。增强用户理解则是通过加入透明性。

\section{跨平台系统设计}
\subsection{原始系统设计}
云中医原始版本只有Android版本,核心诊断和打分算法使用c++编写, 分类模型为OpenCV模型。
在具体实现方面,将所有OpenCV格式模型打包到Android安装包中,在Android平台通过Native方法调用动态链接库的方式完成诊断。

这种非跨平台的实现,有一个明显的缺点就是系统的稳定性需要考虑的用户各种设备环境。在用户调研过程中,很多用户的系统反馈无法完成诊断或者闪退。

\subsection{新系统设计}
解决跨平台需要同时实现IOS,Android,以及Web平台应用,目前云中医已经有安卓版本的实现,模型调用是通过Java调用Native动态链接库的方式。
在实现IOS和Web平台应用时,经过考虑有以下方案:

\subsubsection{方案一}
 IOS重新编写一套代码,通过swift语言调用链接库的方式调用模型; Web平台重新编写一套代码,通过接口调用的方式调用模型。

优点: Android 平台核心代码不需要改动,只需要修改交互和界面。模型文件放在客户端,有安全风险。

缺点: 需要重新编写IOS和Web平台代码,需要同时维护三个平台的代码,保持功能界面的一致性。移动端平台模型是本地调用,不方便统一管理。模型本地调用需要考虑客户端的系统平台和设备性能。

\subsubsection{方案二}
采用C/S架构,全部重写。IOS,Android以及Web客户端使用一套H5代码, 处理逻辑在服务端实现,模型独立服务化,通过Http接口的方式调用。

优点: 客户端使用一套H5代码,维护方便。模型部署在服务器,无泄密风险,可以统一管理(统计调用次数,记录错误日志等),方便实现高可用。

缺点: 依赖网络稳定性,重写的工作量稍大, 需要另外实现模型服务化和服务端。

方案一的缺点会导致后续的维护成本过大,本文最终采用方案二,如图\ref{fig:system}所示,系统主要分为三部分:服务端和客户端,算法模型通过容器的方式服务化成模型池。

\begin{figure}[ht]
    \centering
    \includegraphics[width=12cm]{images/system.png}
    \caption{跨平台系统设计}
    \label{fig:system}
\end{figure}

一次诊断的大致流程如下:用户通过客户端,上传图片或者回答问题,客户端则向服务端发起请求。服务端收到请求之后,进行任务分配,对分配到任务的实例,调用模型池中对应的模型完成特征提取或诊断打分,同时将数据持久化到mysql和磁盘中,然后把结果返回给客户端。客户端收到服务端的结果后,根据用户是否透明,进行结果的展示。

\section{模型池}

模型池由多个独立可用的模型服务构成, 通过http 接口调用完成和服务端的交互进行任务处理。

首先,通过Flask框架建立一个Http的服务,把特征提取模型打包成服务。每个特征提取服务,可以接受http的请求,对传过来的图片进行特征提取。模型的计算能力如\ref{tab:face-feature}, \ref{tab:tongue-feature}, \ref{tab:diag-feature}所示。

云中医目前有两类可用的模型,特征提取模型(面部和舌部)和诊断打分模型,分别有对图片进行特征提取和对特征进行打分的能力。

\begin{table}[]
    \centering
    \begin{tabular}{lll}
        \toprule
        特征          & 特征描述     & 特征内容 \\ 
        \midrule
        faceDetectRes & 人脸   & 0:未检测出人脸,1:成功检测出人脸  \\
        faceColor     & 面部颜色 & 0:面白,1:面黑,2:面红,3:面黄,4:面青,5:正常 \\
        faceGloss     & 面部光泽 & 0:有光泽,1:少光泽,2:无光泽\\
        lipDetectRes  & 嘴唇   & 0:未检测出嘴唇,1:成功检测出嘴唇\\
        lipColor      & 嘴唇颜色 & 0:淡白,1:淡红,2:红,3:暗红,4:紫   \\
        \bottomrule
    \end{tabular}
    \caption{脸部特征提取模型输出}
    \label{tab:face-feature}
\end{table}

\subsection{脸部特征提取模型}
如 \ref{tab:face-feature} 所示,脸部特征提取模型,输入为面部特征图片,输出有以下几个维度:

1. 是否检测到人脸:如果没有检测到人脸,则剩余所有维度无效,且取值为0。

2. 是否检测到嘴唇: 如果没有检测到嘴唇,嘴唇颜色的结果无效且取值必定为0。

3. 面部颜色&面部光泽: 面色和光泽是对预处理之后的图片,去除眼口鼻区域的图片进行面色和光泽信息提取的结果。

4. 嘴唇颜色: 嘴唇颜色的结果从浅到深分别为淡白,淡红,红,暗红,紫。

\begin{table}[]
    \centering
    \begin{tabular}{lll}
        \toprule
        特征 & 特征描述 & 特征内容 \\ 
        \midrule
        tongueDetectRes & 舌体 & 0:未检测出舌像,1:成功检测出舌像 \\
        tongueCrack & 舌裂纹 & 0:未检测到裂纹,1:成功检测到裂纹 \\ 
        tongueFatThin & 舌胖瘦 & 0:正常(瘦),1:胖舌 \\
        tongueCoatThickness & 舌苔厚薄 & 0:薄,1:厚 \\
        tongueCoatColor & 舌苔颜色 & 0:苔白,1:苔黄 \\
        tongueNatureColor & 舌质颜色 & 0:舌暗红,1:舌淡白,2:舌淡红,3:舌红,4:舌紫\\
        \bottomrule
    \end{tabular}

    \caption{舌部特征提取模型输出}
    \label{tab:tongue-feature}
\end{table}

\subsection{舌部特征提取模型}
如 \ref{tab:tongue-feature} 所示,舌部特征提取模型,输入为舌头图片,输入有以下几个维度:

1. 是否检测到舌体: 如果没有检测到舌体,则剩余所有维度无效,且取值为0。

2. 是否检测到舌裂纹: 舌裂纹是最终特征,剩余特征是否有效和该标志位独立。

3. 舌头特征: 包括舌胖瘦,舌苔的厚薄,颜色和舌质颜色。

\begin{table}[]
    \begin{center}
        \begin{tabular}{lll}
            \toprule
            特征 & 特征描述 & 特征内容 \\ 
            \midrule
            healthScore & 健康分数 & 0-100 \\
            healthType & 是否包含某种体质 & {[}0, 0, 0, 0, 0, 0, 0{]} \\ 
            questionScore & 各种问题的体质得分 & {[}0, 0, 0, 0, 0, 0, 0{]} \\
            symCount & 各种体质症状个数 & {[}0, 0, 0, 0, 0, 0, 0{]} \\
            symNum & 总体体质症状个数 & 0-13 \\
            baseScore & 基本分数 & 0-100 \\
            phy & 体质结果 & 八种体质中的一种\\
            \bottomrule
        \end{tabular}
    \end{center}
    \caption{诊断打分模型输出}
    \label{tab:diag-feature}
\end{table}

\subsection{诊断打分模型}
如 \ref{tab:diag-feature} 所示,最终诊断打分模型的体质结果输出为 "阳虚","阴虚", "痰湿","瘀滞", "脾虚", "肾虚", "气虚", "健康" 中的一种。

模型服务使用docker进行打包,通过k8s管理多个模型高实现可用。服务端通过k8s提供的集群 ip进行调用。为了减少读写的竞争,模型服务池不进行mysql和redis的读写,任务异步完成之后,通过回调服务端的方式,由服务端对结果进行持久化。

\section{服务端}

服务端有多个实例,共同处理用户发起的诊断任务,使用redis和mysql进行数据的持久化。客户端使用mui框架加上angular及其插件完成,复杂计算和逻辑通过后台调用实现。

\subsection{日志管理}
为了方便后续的数据分析,我们需要采用用户的所有操作日志。日志的数据库主要字段如下:


\begin{table}[]
    \begin{center}
        \begin{tabular}{lll}
            \toprule
            字段 & 类型 & 描述 \\ 
            \midrule
            id & int & 主键 \\
            user, & text & 用户唯一标识 \\ 
            device & text & 所用设备信息 \\
            op & text & 操作 \\
            info & text & 操作信息 \\
            createTime & datetime & 创建时间 \\
            updateTime & datetime & 更新时间\\
            \bottomrule
        \end{tabular}
    \end{center}
    \caption{操作日志表}
    \label{tab:op_log}
\end{table}


其中,user字段用于标识用户,默认使用用户手机号作为唯一标识,要求用户进入系统前需要通过手机验证码进行登录。而在后续的透明性实验环节,为了方便用户跳转完成问卷,不需要用户进行登录,user字段采用的是wjx-问卷星id。

\subsection{服务端高可用}
为了实现服务的稳定性,服务端支持开启多个实例;同时考虑到性能,服务端设计为读写分离的架构:每个实例都可以读取任务列表,处理用户的任务,但只有主节点有任务分配的权限。同时,服务端通过心跳包探活,主节点失效时,通过redis的分布式锁实现master的竞选。
\begin{figure}
    \centering
    \includegraphics[width=10cm]{images/slave-master.png}
    \caption{服务端竞选}
    \label{fig:slave_master}
\end{figure}
通过任务的id进行哈希,按照哈希进行任务的分配。每个服务端实例过一定的间隔时间,就会去读取任务列表,开始执行任务列表里的任务。

\subsection{任务处理}
一次用户诊断,服务端需要完成多个任务:面部特征提取任务,舌部特征提取任务,问诊打分任务。
服务端在收到用户提交的任务之后,会将数据存储到数据库的task表中,task表的主要字段如\ref{tab:task}所示。


\begin{table}[]
    \begin{center}
        \begin{tabular}{llr}
            \toprule
            字段 & 类型 & 描述 \\ 
            \midrule
            id & int & 主键 \\
            type, & int & 任务类型: 面部,舌部,诊断 \\ 
            in & text & 任务输入 \\
            out & text & 任务结果 \\
            handler & text & 分配的服务端 \\
            status & int & 任务状态: 新建,已分配,处理中,失败,完成 \\
            createTime & datetime & 创建时间 \\
            updateTime & datetime & 更新时间\\
            \bottomrule
        \end{tabular}
    \end{center}
    \caption{任务表}
    \label{tab:task}
\end{table}

type一共有三种取值,对应三种人物类型:面部特征提取任务,舌部特征提取任务,和诊断任务。

面部特征提取和舌部特征提取任务需要的输入为图片,通过base64编码序列化为json对象,保存的in字段中。



服务端在获取到分配给自己的任务之后,便会按照任务的类型请求模型池对应的模型完成计算。模型返回结果之后,服务端将结果缓存到redis中,然后写入到mysql数据库中,再返回给客户端。

\section{客户端}
IOS, Android和Web平台客户端都采用同一套代码,采用MUI+AngularJS实现。
MUI是一个高性能前端框架,本身不依赖任何的第三方JavaScript库,实现了接近原生的用户体验,并内置了大量组件。
AngularJS目前是Google公司的JavaScript框架,支持快速构建移动应用。

设计新的自诊界面,解决交互方面易用性的问题,给用户对自己当前状态一种直观的感受。
\begin{figure}[ht]
    \centering
    \includegraphics[height=10cm]{images/diag.png}
    \caption{面诊舌诊问诊新界面}
    \label{fig:diag_new}
\end{figure}

如图\ref{fig:diag_new}所示,根据用户调研的反馈,新界面简化了诊断的流程,面诊舌诊问诊在一个页面显示,并且所有的问题和操作都是可选的,不会出现必须要先面诊然后舌诊然后才能问诊的问题。其次,新界面对面诊和舌诊进行了中间结果的反馈,面诊在用户拍照确认之后,会立即报告本次照片是否合格已经诊断的结果,用户不需要在点击诊断的时候才被提示照片不合格。

在问诊方面,新系统实现了最近一次记录保存,由于问题是可选回答的,所以用户只需要回答和自己上次的结果不一致的即可。同时,我们对不同问题的回答结果进行了颜色的区分。有色实心代表本次回答,有色空心代表上次回答;橙色代表有症状,绿色代表回答的问题表现良好,没有症状;白色没有填充和边框,为黑字,代表未回答。

在方框内部的问题描述设计上,我们把默认的文字描述,显示为问题描述;一旦用户本次或者上次回答过该问题,则直接显示用户回答的结果。这样做的结果是,第二次用户点进来,就能看到上次的回答结果,这样能够对自己的身体情况有个快速的了解。

客户端系统主要有以下几个部分:用户登录,健康诊断(面诊,舌诊,问诊),健康报告,诊断记录。

\subsubsection{用户登录}
\begin{figure}[ht]
    \centering
    \includegraphics[height=10cm]{images/login.png}
    \caption{用户登录}
    \label{fig:login}
\end{figure} 
用户登录时,需要输入正确的手机号才能通过手机格式验证发送验证码到手机上。登录界面可以接受一个ssojump的参数,用于判断是否来自问卷星。如果验证通过,则直接跳过登录进入首页。

\subsubsection{健康诊断}

\begin{figure}[ht]
    \centering
    \includegraphics[height=10cm]{images/diag.png}
    \caption{Caption}
    \label{fig:diag}
\end{figure}
新界面和之前版本的云中医界面不同的是,诊断界面不仅提供了面诊舌诊问诊的入口,同时会将面诊舌诊的照片和中间结果直接显示在当前页面,给用户对于自己当前身体情况一个直观的感受。
如果拍照失败会直接显示,不需要等到用户进行点击诊断之后才知道自己的照片不合格。
	
\subsubsection{照片上传}

\begin{figure}[ht]
    \centering
    \includegraphics[height=10cm]{images/crop.png}
    \caption{Caption}
    \label{fig:crop}
\end{figure}
用户通过点击面诊或者舌诊的圆圈图案,可以通过从相册选择或者通过拍照,上传自己的脸部或者舌头照片。
图片裁剪可以帮助用户定位面部和舌头的位置,提高诊断的精度。用户点击确认裁剪之后,诊断的结果会直接显示在圆圈下面,同时,如果没有识别到人脸或者舌头也会提示用户重新拍照。
点击确认裁剪后,通过对图片进行base64编码上传到服务器的诊断接口,服务器调用模型池对应模型,拿到诊断结果。

\subsubsection{问诊}

\begin{figure}[ht]
    \centering
    \includegraphics[height=10cm]{images/questions.png}
    \caption{问诊}
    \label{fig:questions}
\end{figure}
问诊的问题一共13道,用户根据自己的情况通过勾选回答问题。每次进入问诊界面,系统会尝试加载上一次用户回答问题的历史记录和对应的答案。
同时,回答过的问题,会在界面上进行显示。没有回答过的问题,将是白色黑字没有边框。


用户在完成自己需要回答的问题或者面诊舌诊之后,可以点击蓝色的诊断按钮进行健康诊断,同时,系统会将本地诊断记录上传到后台服务器上,以便查询诊断记录。

\subsubsection{诊断结果}
\begin{figure}[ht]
    \centering
    \includegraphics[height=10cm]{images/report.png}
    \caption{健康报告}
    \label{fig:report}
\end{figure}
系统根据用户个人的情况,会给出面诊结果,舌诊结果和最后的体质以及健康分数。此外,根据体质的不同,会给出和体质对应的健康建议,帮助用户进行针对性的健康调理。

\subsubsection{诊断记录列表}
\begin{figure}[ht]
    \centering
    \includegraphics[height=10cm]{images/history.png}
    \caption{诊断历史}
    \label{fig:history}
\end{figure}
诊断记录页面,能够直观地给出用户近期的健康变化的情况。用户可以点击诊断记录,进入当时的详细诊断结果页面。

%%%%%%%%%%%%%%%%%%%%%%%%%%%%%%%%%%%%%
\section{透明性}
考虑到中医应用的特殊性,普通用户需要提高对应用的理解。因此我们在系统中加入算法的透明性,提升用户的体验。

对于每一个用户,我们通过哈希算法,对用户名计算哈希值,按照哈希值的奇偶性,将奇数用户归类为不透明用户,偶数用户归类为透明用户。
两类用户在进行面诊舌诊时的流程一样,但是透明用户能够看到背后特征提取算法和诊断算法的中间数据,同时系统为给出的诊断结果进行了解释。

\subsection{面诊舌诊的透明性}
普通用户在面诊页面

\subsection{问诊的透明性}

\subsection{诊断结果的透明性}
\begin{figure}[ht]
    \centering
    \includegraphics[height=10cm]{images/report3.png}
    \caption{透明用户看到的诊断报告}
    \label{fig:my_label}
\end{figure}

普通用户在诊断结果页面,可以看到自己的健康分数和体质结果;透明用户可以点击诊断分数,了解这个分数是根据哪些指标,通过哪一个算法计算过来的。如图\ref{fig:report_expalin_score}所示,用户点击诊断页面的分数之后,弹窗里会显示分数相关的问题、雷达图和分数计算公式。

分数相关问题,展示了面诊舌诊对体质分数的影响和问诊对体质分数的影响,无影响的问题则不会显示。其中体质分数的变化分两种,一个是分数的累加,另一种是体质分数的清空。

雷达图对体质分数进行了汇总,给用户展示最终个人的体质倾向的结果。

根据诊断打分模型的内部算法,解释页面的计算公式一共有5种类别,我们使用选项卡的方式,将所有的打分计算公式全部透明给用户,并且默认打开当前计算公式的选项卡。

\begin{figure}[ht]
    \centering
    \subfigure[相关问题]{
        \includegraphics[height=8cm]{images/report7.png}
    }
    \subfigure[雷达图]{
        \includegraphics[height=8cm]{images/report8.png}
    }
    \subfigure[计算公式]{
        \includegraphics[height=8cm]{images/report9.png}
    }
    \caption{分数的解释}
    \label{fig:report_expalin_score}
\end{figure}

点击面诊结果,可以看到自己的面部舌部的对于整个诊断的影响。
\begin{figure}[ht]
    \centering
    \includegraphics[height=10cm]{images/report4.png}
    \caption{面诊的解释}
    \label{fig:my_label}
\end{figure}

点击体质分数,可以看到当次诊断中,面诊舌诊和用户自己回答的问题,哪些影响到了最后体质的判断。


\begin{figure}
    \centering
    \subfigure[概念的解释]{
        \includegraphics[height=10cm]{images/report5.png}
    }
    \subfigure[体质相关问题]{
        \includegraphics[height=10cm]{images/report6.png}
    }
    \caption{体质的解释}
    \label{fig:report_explain_phy_1}
\end{figure}

\section{本章小节}


% 5.交互实验
\chapter{诊断算法解释性实验}

\section{系统可解释性}

已有大量的研究表明,增加系统的可解释性,是可以提高用户对系统的信任程度的。
考虑到中医应用的特殊性,普通用户需要提高对应用的理解。因此我们在系统中加入算法的解释性,提升用户的体验。而根据之前的用户调研来看,由于当前诊断和打分模型存在改进的空间,部分的用户也对结果产生了怀疑或者对结果不理解。
在这个基础上,我们希望通过讲模型的判决过程透明化,并对结果进行解释。

在具体代码实现的时候,系统对于是否过程透明、对结果进行解释是通过用户的类型来进行判断的。

对于每一个用户,我们通过哈希算法,对用户名计算哈希值,按照哈希值的奇偶性,将奇数用户归类为不透明用户,偶数用户归类为透明用户。
两类用户在进行面诊舌诊时的流程一样,但是透明用户能够看到背后特征提取算法和诊断算法的中间数据,同时系统为给出的诊断结果进行了解释。


解释在呈现的时候,大致可以分为两种:

1. 结果中文字结果是有提示可以进行点击。如果用户点击了健康报告的分数,会通过弹窗进行详细的解释。

2. 通过雷达图的方式直接显示对结果的影响,交互性比较强。

\subsection{面诊舌诊过程的解释性}

\begin{figure}
    \centering
    \subfigure[不解释]{\includegraphics[width=4.5cm]{images/face_tongue.png}}
    \subfigure[解释]{\includegraphics[width=4.5cm]{images/exp_face_tongue.png}}
    \subfigure[解释舌诊中间结果]{\includegraphics[width=4.5cm]{images/exp_tongue.png}}
    \caption{面诊舌诊的解释}
    \label{fig:face_diags}
\end{figure}

面诊和舌诊流程比较类似,添加解释主要体现在结果的解释,包括中间结果,和对各种体质倾向的影响。
这样用户查看解释之后,能够大致了解本次拍照是否成功,并且知道目前面诊的结果,会对最终的健康报告造成哪些影响。

如图 \ref{fig:face_diags} 所示:

a) 普通用户只能看到结果。

b) 透明类型的用户,可以通过点击结果,查看对结果的解释。

c) 透明类型的用户,不仅可以看到当面诊断的中间结果,也能看到这次诊断的体质倾向得分。



\subsection{问诊的解释}

\begin{figure}
    \centering
    \subfigure[不解释]{\includegraphics[height=10cm]{images/questions.png}}
    \subfigure[解释]{\includegraphics[height=10cm]{images/questions2.png}}
    \subfigure[解释体质术语]{
        \includegraphics[height=10cm]{images/exp_phy.png}
    }
    \caption{问诊}
    \label{fig:questions}
\end{figure}

如图 \ref{fig:questions} 所示:

a) 普通用户,回答问诊问题之后,没有任何提示或者解释。

b) 透明类型的用户,在进行问诊过程中,可以立即看到每个答案对结果的印象,通过下方的雷达图显示了影响的体质倾向类型和具体的数值。
雷达图的更新是实时根据用户的选择进行更新的,提高了交互性。

c) 对中医术语中,各种体质的解释,对于体质内容的解释文字引用自 《中医体质分类研究》标准。



\subsection{诊断结果的解释}
\begin{figure}[ht]
    \centering
    \subfigure[解释]{
        \includegraphics[height=10cm]{images/report3.png}
    }
    \subfigure[不解释]{
        \includegraphics[height=10cm]{images/report.png}
    }
    \caption{诊断报告}
    \label{fig:my_label}
\end{figure}

普通用户在诊断结果页面,可以看到自己的健康分数和体质结果;透明用户可以点击诊断分数,了解这个分数是根据哪些指标,通过哪一个算法计算过来的。

可以点击查看的结果的解释有:面诊舌诊结果的解释,健康分数的解释,体质的解释。

\subsubsection{面诊舌诊结果的解释}

\subsubsection{健康分数的解释}
问诊结果的解释主要是对用户透明诊断结果是如何计算出来的,以及那些问诊的问题对结果有影响,影响程度多少。

\begin{figure}[h]
    \centering
    \subfigure[相关问题]{
        \includegraphics[height=7cm]{images/report7.png}
    }
    \subfigure[雷达图]{
        \includegraphics[height=7cm]{images/report8.png}
    }
    \subfigure[计算公式]{
        \includegraphics[height=7cm]{images/report9.png}
    }
    \caption{分数的解释}
    \label{fig:report_expalin_score}
\end{figure}

如图\ref{fig:report_expalin_score}所示,用户点击诊断页面的分数之后,弹窗里会显示分数相关的问题、雷达图和分数计算公式。

分数相关问题,展示了面诊舌诊对体质分数的影响和问诊对体质分数的影响,无影响的问题则不会显示。其中体质分数的变化分两种,一个是分数的累加,另一种是体质分数的清空。

雷达图对体质分数进行了汇总,给用户展示最终个人的体质倾向的结果。

根据诊断打分模型的内部算法,解释页面的计算公式一共有5种类别,我们使用选项卡的方式,将所有的打分计算公式全部透明给用户,并且默认打开当前计算公式的选项卡。

点击面诊结果,可以看到自己的面部舌部的对于整个诊断的影响。

点击体质分数,可以看到当次诊断中,面诊舌诊和用户自己回答的问题,哪些影响到了最后体质的判断。

\subsubsection{体质的解释}

\begin{figure}[ht]
    \centering
    \subfigure[面诊的解释]{
        \includegraphics[height=7cm]{images/report4.png}
    }
    \subfigure[概念的解释]{
        \includegraphics[height=7cm]{images/report5.png}
    }
    \subfigure[体质相关问题]{
        \includegraphics[height=7cm]{images/report6.png}
    }
    \caption{诊断结果的解释}
    \label{fig:report_explain_phy_1}
\end{figure}


% \section{反馈调研}
% 在新系统完成之后,我们采访了16名用户。
% 在本次调研过程中,用户试用的是同时有新旧界面的版本,除了第一次介绍使用的时候,我们会让用户两个版本都是使用一下,后续不做限时。用户具体使用的时候可以根据自己的喜好选择。
% 经过回访,新版的界面达到了预期,大部分用户觉得使用起来更加地方便。
% 不过值得注意的是,也有少部分的用户喜欢旧版的将面诊,舌诊,问诊分开为三步进行的方式,因为这样比较符合日常生活的习惯。


\section{实验设计}
我们通过在各大社交平台发布海报招募,如图\ref{fig:poster},以及使用问卷星的样本服务, 经过筛选之后,一共招募了100位左右的用户。
\begin{figure}[htb]
    \centering
    \includegraphics[height=8cm]{images/poster.png}
    \caption{招募海报}
    \label{fig:poster}
\end{figure}
\section{实验流程}

每个用户的实验流程如下:

1. 用户通过扫描二维码,或者通过我们给定的链接,进入问卷星调查问卷。

2. 完成调查问卷之后,自动进入云中医在线app。通过调用问卷星提供的企业用户接口, 同时把问卷星的问卷id通过ssojump传给云中医在线应用。通过ssojump中的问卷id, 完成自动登陆, 登陆的用户id为wjx-{问卷星id}, 透明类别为通过问卷星id哈希得到。

3. 在用户完成一次面诊之后,会在健康诊断页面下,看到一个跳转链接,可以选择填写用后问卷。

这样就把调查问卷信息,云中医应用使用日志和用后问卷数据关联起来了。


\section{实验数据获取}
每个用户在参与实验之后,我们可以得到调查问卷的数据,云中医的使用日志,已经用后问卷的数据。

调查问卷: 序号,提交答卷时间,所用时间,来源,来源详情,IP,个人信息	我相信中医养生,了解中医养生,平时注重养生,经常去看中医,希望自己的生活方式更健康,对学习相关中医养生知识感兴趣,认为中医面诊可以了解健康情况,认为中医舌诊可以了解健康情况,认为智能系统可以自动评估健康情况,随机顺序的中医知识问题

用后问卷: 序号,提交答卷时间,所用时间,来源,来源详情,IP,云中医的诊断结果,结果的信任程度,对结果的理解程度,是否愿意使用类似应用,随机顺序的中医知识问题

用户使用日志: 序号,用户名,设备信息,操作名,操作信息,日期 

个人信息包括性别,年龄段,受教育水平,职业,城市,健康状态等。

\subsubsection{关联规则}
对于同一个用户在一次实验过程中,若调查问卷的序号为ID, 则此次用户使用日志的用户名为wjx-ID, 用户问卷的来源详情为wjx-ID。 
通过这个对应关系,我们就可以利用实验平台提供的问卷关联的功能,把三个表的信息,合并到一个表中,通过django-admin导出下载,最终得到特征如表\ref{tab:exp_data}所示。


\begin{table}[h]
    \centering
    \begin{tabular}{ll}
        \toprule
        字段 & 描述 \\ 
        \midrule
        序号 & 调查问卷序号 \\
        性别 & 男/女/未知 \\
        教育程度 & 本科以下/本科/本科以上 \\
        工作类型 & 计算机相关/计算机不相关 \\
        用户类型 & 透明/不透明 \\
        健康得分 & 云中医应用给出的分数 0-100 \\
        信任 & 对结果的信任程度 1-5 \\
        理解 & 对结果的理解程度 1-5 \\
        前健康知识得分 & 使用云中医之前的健康知识得分 \\
        后健康知识得分 & 使用云中医之后的健康知识得分 \\
        查看了哪些解释 & 使用过程中查看的解释类型 \\
        \bottomrule
    \end{tabular}
    \caption{实验数据汇总}
    \label{tab:exp_data}
\end{table}

表格 \ref{tab:exp_data}的字段说明如下:

工作类型: 根据调查问卷的数据,用户自由填写的职业类型有很多,为了便于分析,我们把 IT经理,IT软件设计, hr, it, 互联网, 技术研发, 技术研发人员, 技术经理, 电脑工程师, 研发, 科研, 程序员, 计算机, 软件工程师, 软件开发工程师, 通信, 通讯等归为it相关。

信任、理解: 这两个字段来自用后问卷调查表,取值范围为1-5的整数。

健康知识得分: 使用云中医应用前后的健康知识得分,使用的是用户回答正确中医知识的个数。

用户类型:透明类型的用户才能看到对结果以及术语的解释。

查看了哪些解释: 通过检索日志中关键字获取,包括面诊过程,舌诊过程,体质术语介绍,诊断报告的:面诊结果,舌诊结果,健康分数,体质,分数计算公式等。

\section{实验结果}

\subsection{用户是否愿意继续使用?}

\subsection{}

同时实验分析,我们可以得出,增加解释可以提高用户对应中医知识的理解。

和看解释有关, 用 二元逻辑回归

把用户进行筛选,看了解释和没有看解释的

有解释的,看了没没看的(为什么看了,为什么没看), 

相关性分析,用于分析自变量

挑选相关性比较低的,放到后续的模型中 , 不加*,是很弱的 0.4-0.6比较强

了解中医养生,平时注重养生,了解中医养生
相信中医,了解中医

有没有看解释和it相关性不高

看了解释的影响:
独立样本t检验,差异性分析,没有显著差异

找最相似的不看解释的用户和看了解释的用户对比: 
对结果的信任和理解程度,两个样本没有显著差异



量化结果

定性实验

\section{结构性访谈}



% 6.总结
\chapter{总结与展望}
\section{总结}
随着移动设备硬件水平的提高,各种日常场景下的健康应用随之出现;而在健康诊断信息化方面,人工智能的发展让许多健康诊断技术也能应用在实际生活中,其中就包括基于面部信息的健康诊断技术。我们针对日常场景下的面诊交互这一场景,发现虽然目前已有大量的日常健康的交互技术和大量的面诊信息化方法,但是主要都是面向专业人士或者在医疗诊所环境下使用,缺乏如何将面诊技术应用到日常场景下的交互研究。本文按照交互研究发现问题、寻求解决方案、确定解决方案、系统实现、探索设计实现并验证的步骤,做了以下的工作:

\begin{enumerate}
	\item 使用云中医作为技术探针,设计了交互实验,通过社交平台发布海报募了参与者参与实验并对在实验过程中及结束后对参与者进行了深度访谈以获取实验数据。
	
	\item 结合实验数据,分析出面诊技术在日常场景下存在着自适应性、实用性和敏感性的问题,特别的是作为健康诊断工具的时候,可能会影响用户的情绪和对系统的信赖程度。针对这些可能存在的问题,本文同时提出了对应:支持持续使用、系统可解释性、日常可用性的设计思路。

	\item 为了方便交互实验,设计并实现了日常场景下可拓展的面诊系统,跨平台的设计屏蔽了用户不同设备类型可能会遇到的问题,统一的模型管理提高了模型的稳定性,方便新模型的接入和替换;任务管理对一次诊断任务进行了分解,提高多用户请求时的并发度;用户操作记录管理和问卷关联的功能,能够更加方便地设计新的问卷和采集用户信息,对用户细粒度的行为进行分析成为可能。
	
	\item 在设计思路的迭代过程中,探索了如何实现可用性的设计、如何对结果影响的权重进行展示、如何对系统进行解释等,最终分别实现了基于可用性和可解释性的设计思路的应用原型。

	\item 对本文提出的可用性设计和可解释性设计在面诊系统上进行了实验,验证了系统功能的同时探索了对应的设计思路的效果。
\end{enumerate}

% 不足之处


% 展望
\section{展望}
本文在后续的实验阶段,通过面诊系统对交互问题中的可用性和可解释性的问题进行了验证和探索,未来可进行的交互研究工作如下:

\begin{enumerate}
	\item 设计如何增加系统的自适应。系统的自适应性包括对用户的自适应和对上下文的自适应。用户自适应性设计的实现需要采集用户信息,并分析出用户信息中和个人健康的相关点,上下文则包括短期的天气节气上下文信息和长期的用户使用记录。如何实现系统对用户数据的自适应性,并且如何在养生建议等步骤中加入上下文信息,还需要进一步地探索和设计实验。

	\item 与用户的日常行为融合。日常健康诊断在某些用户中不是一个日常的行为,用户不一定习惯于每天为专门为了检测自己的健康情况打开相机。而大多数用户会有自拍的习惯,面诊技术的特殊性在于它基于面部信息,如何将自拍和健康诊断结合起来,也是一个非常有意思的研究点。

	\item 探索日常诊断的社交功能。在第二章中,我们讨论了面诊信息的分享的特殊性:一方面有的用户考虑个人照片会泄露隐私,另一方面用户希望通过分享和其他人交流健康的知识。同时,在大量相关研究中,添加社交功能也是促进用户持续使用的常用方法之一。那么如何设计一个在不泄露用户隐私信息的情况下,实现用户之间的面部诊断信息分享和交流,还需要进一步地研究。
	
\end{enumerate}

\backmatter
% 后置部分包含参考文献、声明页(自动生成)等

\small
\bibliographystyle{plain}
\bibliography{bibliography}
% % 打印参考文献列表

\normalsize
% 致谢
% \chapter{致谢}

% 总结这三年的经历
三年飞快就过去了,经历了一波又一波实验室的师兄师姐毕业,转眼就快要到了自己的毕业的时间。 这三年来的生活,和一群优秀的人待在一起,提高了自己的眼界,收获到了很多。
这几年里,复旦大学优秀的学习环境,浓厚的学术氛围也给我留下了深刻的印象。

% 感谢导师
首先感谢我的导师丁向华老师一直以来的包容和工作方面的支持。丁老师的学术功底扎实,科研态度严谨,关心学生, 没有丁老师的帮助我几乎不可能顺利毕业。

% 感谢实验室其他老师
同样感谢实验室顾老师和卢老师三年来在学习和生活上的照顾,让我受益匪浅。

% 感谢实验室其他同学
感谢宋励师兄在我刚来实验室的时候,带领我阅读论文,指导我学习安卓应用开发,一起熟悉实验室的环境。 
感谢刘鹏师兄与蒋特同学,一起完成实验室的项目,在我遇到困难时为我出谋划策。同时感谢210实验室其他同学三年来的陪伴以及在我经济困难是给予的帮助。

% 感谢家人和公司的同事
最后感谢实习公司的领导和同事,在实习期间我了解到了规范的开发流程和成熟的业务解决方案,对论文的系统实现以及后续找工作帮助很大。

\end{document}
