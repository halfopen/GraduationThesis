\begin{abstract}



    随着经济的发展,人们生活节奏的加快以及养生意识的逐步提高,越来越多的人开始关注自己的健康问题,但是紧凑的工作时间让绝大多数人无法单独抽出时间去医院定期检查自己的健康状况。
    近些年移动设备的普及和发展,移动设备的性能得到大大地提高,大量关于日常场景下的健康追踪技术也在不断深入。
    而随着人工智能技术的快速发展,健康诊断算法的能力也在不断地提高,特别是基于面部信息诊断技术也在不断突破,在移动设备上基于面部信息进行日常健康诊断已经成为可能。
    
    目前已有的基于读脸技术的健康诊断技术即面诊技术,主要应用场景还是诊所环境下,而人机交互(HCI)领域虽然有大量关于日常场景下的健康管理技术,但主要集中在用户健康行为引导和如何帮助用户更好地进行慢性病管理等方面。
    在这个背景下,本文对日常环境下的面诊交互技术应该如何设计的问题进行了探索,提出了对应的设计思路并通过实验平台进行了验证。
    本文的主要工作如下:
    
    \begin{enumerate}
    
        \item 利用技术探针进行日常场景下面诊应用的交互研究。
    通过对参与者进行深度访谈,分析转录成文字后的访谈资料,探索出了日常场景下面诊应用会遇到的问题并提出了设计思路:增加可变性、系统可解释性和日常可用性。
    
        \item 设计并实现了可拓展的面诊模型的交互实验平台。
    在开展交互实验之前,为了提高后续实验的效率,同时解决用户设备多样性带来的稳定性的问题,设计并实现了用于快速开展面诊应用相关实验的实验平台:实现了系统高可用,快速迭代,模型任务模块分离,用户操作记录管理,问卷关联等功能,提高了系统的稳定性,能够应对后面大规模的用户实验场景,并且快速地开展新的实验。
    
    % 通过调用实验平台接口对跨平台客户端进行实现并通过实验验证设计思路的有效性。模型运行在服务器,解决了用户设备的差异性带来的稳定性和性能的问题。
    
        \item 利用实验平台,分别验证并分析了增加系统解释性和增加日常可用性的设计思路。
    为了验证设计思路,本文分别实现了解决可用性问题的原型和增加解释性的原型,并通过实地调研的方式验证了设计思路的有效性。
    实验的结果和分析表明,本文提出的提高可用性的设计思路的确能够提高系统使用的便利性,能够方便用户在日常健康的场景下使用;
    而本文提出的增加系统解释性的设计思路,的确能够在目前算法的结果不完美的情况下,提高用户对系统的信赖程度和对结果的接受程度。
    
    
    
    \end{enumerate}
    
    \end{abstract}