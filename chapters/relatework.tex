\chapter{相关工作}

\section{基于中医的健康诊断信息技术}

很多都是在医院里用, 基本还是停留在医疗环境下使用,

中医面诊信息化可以分为两个部分:1)通过信息化技术,定量分析面诊的判决的依据,通过计算机来完整健康诊断;2)使用新的硬件设备,辅助医生进行信息提取,然后由医生来判断
患者的健康状况。
在第一类中,已有大量的工作分析面色和个人健康之间的关系。

而另一方面,四诊仪,脉诊仪目前已经大量在诊所环境下开始应用,提高医生的工作效率。

不足:只在诊所环境,不方便日常使用

\section{日常健康技术}



很多都是日常

没有生病,鼓励健康生活方法

慢性病管理

在人机交互领域,更好地设计健康追踪技术和应用,一直是研究的热点。在以往的研究中,大多数研究者探索了糖尿病,高血压等慢性病追踪技术。




没有涉及到关于日常健康评估的技术

\section{智能系统的透明化}

AI 解释性相关的做法

context-aware
推荐-

考虑到诊断的特殊环境,用户没有专业背景,场景不一样