\chapter{相关工作}
% 对研究现状进行展开,详细介绍
\section{基于中医的健康诊断技术}

目前基于中医的健康诊断信息技术的研究,很多应用场景都是在局限在医院及诊所场景下。目前的医疗信息系统,主要可以分为一下几个方向:
医疗数据采集,医疗数据存储,诊断数据管理如电子病历系统,医护信息系统,电子影像系统等。

按照研究性质也可划分为新的算法研究和硬件应用研究,算法研究如基于肤色的肝病判别方法,硬件研究如面诊仪,舌诊仪等。


当前的基于中医的健康诊断技术研究的局限性在于,大多数系统或者应用只能应用在诊所环境,不方便用户在日常环境下使用。

\section{日常健康技术}

在人机交互领域,如何更好地设计健康相关的技术和应用,也一直是研究的热点。

和诊断技术研究不同,近几年人机交互研究可以大致分类两类: 第一类是在用户没有生病的情况下,鼓励用户进行日常健康的方式, 如更好地锻炼,饮食,睡眠等。健康监测和跟踪系统的主要作用,就是能够让用户得到关于自己健康的反馈。目前这类健康检测系统监测的指标主要是

第二类则是关于慢性病的长期管理和追踪技术,提高用户的生活品质。
如糖尿病\cite{mamykina2008mahi:}, 痴呆\cite{yasuda2009remote}, 多发性硬化\cite{ayobi2017quantifying},
双向情感障碍症\cite{bardram2013designing},以及多种慢性病的管理\cite{nunes2015self-care}。

% 很多都是日常的
        
% 没有生病,鼓励健康生活方法

% 慢性病管理

% 在人机交互领域,更好地设计健康追踪技术和应用,一直是研究的热点。在以往的研究中,大多数研究者探索了糖尿病,高血压等慢性病追踪技术。


但是,目前的研究还没有涉及到给予面部舌部特征进行健康评估的技术。

% \section{智能系统的透明化}

% AI 解释性相关的做法

% context-aware
% 推荐-

% 考虑到诊断的特殊环境,用户没有专业背景,场景不一样


% 类似系统设计

% 基于数字化全科诊断的健康云服务平台关键技术研究 梅仁友 东南大学 (一个用来做数据挖掘的平台?  没有系统对比,只有功能测试)

% 中医多诊信息采集系统的设计与实现 肖赛 华中科技大学

% 健康云平台的建设与实践

