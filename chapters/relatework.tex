\chapter{相关工作}

\section{日常健康技术的交互研究}

在人机交互领域,国内外有大量的关于如何设计日常健康相关的交互研究。从研究的出发点来看,可以分为慢性病管理和鼓励用户健康生活方式的研究两种。

在慢性病管理方面,当前研究主要是关于各种类型的慢性病的长期管理和追踪技术,提高用户在患病状态下的生活品质。
如 Lena Mamykina \cite{mamykina2008mahi:}等人开发了一个名为MAHI的应用,通过蓝牙和传统的血糖仪进行通信,
提供了血糖记录的功能,并且允许用户之间在平台上互相交流等,经过实验发现该系统提供的功能能够大大鼓励用户实现血糖管理目标如控制饮食。
Kiyoshi Yasuda \cite{yasuda2009remote}则研究了如何设计与痴呆病人的远程通讯系统,帮助痴呆病人与家人进行沟通,提高痴呆病人在家保持情绪稳定的时间,改善痴呆病人的日常在家的生活质量。
Amid Ayobi 等\cite{ayobi2017quantifying} 通过对多发性硬化病人的日常检测应用的研究,发现日常健康技术可以帮助提高病人的日常控制意识,而不只是检测疾病相关的指标。
在精神健康方面,Jakob E. Bardram等 \cite{bardram2013designing}探索了如何设计检测双相情感障碍症的日常应用, 研究发现因为移动平台的便携性,患者可以随时携带智能手机,通过系统提醒和实时数据可视化可以提高患者的疾病意识。
同时该系统提供的评估功能相较于纸质表格评估的模式,使用体验大大得到了提高,更加利于用户长期坚持疾病评估。

在鼓励用户健康生活方面,主要是对各类监测系统的研究。监测系统主要是对用户的各种健康比较相关的指标,如睡眠,心态,血压进行检测和记录,能够让用户得到关于自己健康情况的一个反馈。
在用户处于健康或者亚健康的情况下,指导用户更加健康地生活。 
Yuma Inagawa  \cite{Inagawa2013A} 开发了一个营养管理和检索系统,根据用户的喜好给出推荐的菜单,用户也可以给出对应的反馈,以此来培养用户健康的饮食习惯。
Stephen Purpura \cite{purpura2011fit4life} 设计了一个鼓励用户减肥的系统 fit4life,  而Lullaby  \cite{kay2012lullaby} 是Matthew Kay等人利用光线、声音、温度、运动等传感器,用于研究干扰用户睡眠因素的系统。

随着人工智能以及计算机视觉技术的发展,基于面部信息的读脸技术应用也越来越广泛,相关研究也在探索如何将日常健康和面诊技术结合起来。Yasmina Andreu-Cabedo  \cite{andreu2015mirror}则开发了SEMEOTICONS, 一款放在家庭室内环境的镜子,通过采集面部表情信息和体温,检测与心血管疾病相关的户疲劳,压力和焦虑等特征, 让用户能够监测自己的健康情况,并根据量身定制的健康指南来改善用户的生活方式。
 该研究将面部信息和健康管理结合起来,并且将应用场景拓展到了家居环境。
 在后续的用户调研表明\cite{coppini2017user} 他们的原型系统被大多数用户所接受: 虽然测试过程非常耗时,大多数研究中的志愿者仍然很愿意完成和按计划进行实验,部分志愿者考虑了系统给出健康指南,甚至因此改变了自己的生活方式。
该系统通过将诊断和日常的照镜子的行为结合起来,可以很方便与用户的日常生活融合起来,但固定在房间的某个位置使用的方式不够编写,同时也增加了硬件的成本。

总结上述的研究我们发现,当前关于日常健康交互技术的研究有很多,涉及各种疾病已经各种健康评估和检测的方法,但是目前还缺乏关于日常环境下如何设计基于面部信息进行健康评估的研究。

\section{健康诊断信息化技术}

目前的健康诊断相关技术,大多数还是停留在理论阶段或者应用在医疗环境下。

在信息化系统方面,相关的技术有电子病历管理系统\cite{高春芳2013电子病历系统应用现状及前景展望}, 电子影像管理系统\cite{张安平2018医院信息管理系统的电子病历和医学影像系统分析}, 医护信息管理系统\cite{虞正红2018医护合作静脉血栓栓塞管理信息化平台的设计与应用}等。
这些信息管理系统的主要是为了提高医生的工作效率,对医疗信息进行有效的管理。这类管理系统主要是为医护从业人员设计,使用者需要丰富的专业知识和操作培训才能使用。

在医用器械研究方面,面诊仪如道生面诊仪\cite{邸丹2016手持式舌象仪的研制}是目前某些医院用来采集面部信息的设备,需要在中医医师的指导下,进行舌相面色诊断信息采集的设备,供中医医师进行判断;
 同样,舌诊仪\cite{李丹溪2017舌诊仪的发展及其在舌诊客观化研究中的应用现状} 和脉诊仪 \cite{牛婷婷2017脉诊仪}虽然在电脑端会提供一些自动化分析的功能,但主要还是设计为辅助医师进行诊断或者数据采集的工具。
 
 与本文工作相关的是云中医\cite{Zhang2018Study},本文也是使用云中医作为技术探针\cite{Hutchinson2003Technology}, 它是一个放在诊所或者社区,可供用户进行中医面诊的系统,能够让用户在完成面诊舌诊问诊之后,给出健康分数和养生建议。
 云中医虽然功能比较齐全,但是其主要工作是算法和系统实现层面,缺乏深度的用户调研和交互研究。

以上提到的健康诊断技术,主要的实际应用场景还是局限在医疗诊所环境下,没有考虑到普通用户的日常使用。在面诊技术的日常使用这种场景下,面诊技术有哪些独特的社会文化特点,用户会遇到哪些问题,相关的设计应该遵循哪些设计原则,还处于待研究的阶段。

\section{本章小结}
本章按照人机交互领域和诊断信息化两个方面分别介绍了相关的研究内容和成果,并总结了这些工作主要内容和本文工作的不同。