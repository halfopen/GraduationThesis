\chapter{系统实现}
根据上面的调研结果,可以改进的方面有很多。但是从用户日常使用的角度,影响用户持续使用最大的问题是可用性的问题和理解的问题。可用性的解决方案主要分为两块:一个是提高系统的可用性;另一个是重新设计用户友好的交互界面,提升交互的体验。增强用户理解则是通过加入透明性。

如图\ref{fig:system}所示,系统主要分为三部分:算子池、服务端和客户端。

\begin{figure}
    \centering
    \includegraphics[width=12cm]{images/system.png}
    \caption{系统设计}
    \label{fig:system}
\end{figure}

算子是具有基本计算能力的处理单元。算子池中的算子目前有两种,特征提取算子和诊断打分算子,分别有对图片进行特征提取和对特征进行打分的能力。服务端有多个实例,共同处理用户发起的诊断任务,使用redis和mysql进行数据的持久化。客户端使用mui框架加上angular及其插件完成,复杂计算和逻辑通过后台调用实现。

一次诊断的大致流程如下:用户通过客户端,上传图片或者回答问题,客户端则向服务端发起请求。服务端收到请求之后,进行任务分配,对分配到任务的实例,调用对应的算子完成特征提取或诊断打分,同时将数据持久化到redis和mysql中,然后把结果返回给客户端。客户端收到服务端的结果后,根据用户是否透明,进行结果的展示。

\section{算子池}

为了能够满足系统的跨平台使用,我们需要把特征提取算法从客户端剥离到服务器,通过接口调用的方式来实现面诊和舌诊,同时在服务的实现诊断算法。

首先,我们使用容器的方法,通过flask本地调用的方法,把c++实现的特征提取算法打包成特征提取算子。每个特征提取算子,可以接受http的请求,对传过来的图片进行特征提取。算子部署在docker上,通过k8s管理多个算子高实现可用。服务端通过k8s提供的cluster ip进行调用。


\section{服务端}
\subsection{服务端系统设计}



\subsection{服务端高可用}
为了实现高可用,服务端支持开启多个实例;同时考虑到性能,服务端设计为读写分离的架构。每个实例都可以读取任务列表,处理用户的任务,但只有master有任务分配的权限。但是服务端通过心跳包,实现master的竞选。
\begin{figure}
    \centering
    \includegraphics[width=12cm]{images/slave-master.png}
    \caption{服务端竞选}
    \label{fig:my_label}
\end{figure}
通过任务的id进行取余,进行任务的分配。每个服务端实例过一定的间隔时间,就会去读取任务列表,开始执行任务列表里的任务。

\section{客户端}
设计新的自诊界面,解决交互方面易用性的问题,给用户对自己当前状态一种直观的感受。
\begin{figure}
    \centering
    \includegraphics{images/diag.png}
    \caption{面诊舌诊问诊新界面}
    \label{fig:diag_new}
\end{figure}

如图\ref{fig:diag_new}所示,根据用户调研的反馈,新界面简化了诊断的流程,面诊舌诊问诊在一个页面显示,并且所有的问题和操作都是可选的,不会出现必须要先面诊然后舌诊然后才能问诊的问题。其次,新界面对面诊和舌诊进行了中间结果的反馈,面诊在用户拍照确认之后,会立即报告本次照片是否合格已经诊断的结果,用户不需要在点击诊断的时候才被提示照片不合格。

在问诊方面,新系统实现了最近一次记录保存,由于问题是可选回答的,所以用户只需要回答和自己上次的结果不一致的即可。同时,我们对不同问题的回答结果进行了颜色的区分。有色实心代表本次回答,有色空心代表上次回答;橙色代表有症状,绿色代表回答的问题表现良好,没有症状;白色没有填充和边框,为黑字,代表未回答。

在方框内部的问题描述设计上,我们把默认的文字描述,显示为问题描述;一旦用户本次或者上次回答过该问题,则直接显示用户回答的结果。这样做的结果是,第二次用户点进来,就能看到上次的回答结果,这样能够对自己的身体情况有个快速的了解。

客户端系统主要有以下几个部分:用户登录,健康诊断
\subsubsection{用户登录}
\begin{figure}[ht]
    \centering
    \includegraphics{images/login.png}
    \caption{用户登录}
    \label{fig:login}
\end{figure} 
用户登录时,需要输入正确的手机号才能通过手机格式验证发送验证码到手机上。
 
% \subsubsection{系统首页}
%  \begin{figure}[h]
%      \centering
%      \includegraphics{images/main.png}
%      \caption{Caption}
%      \label{fig:my_label}
%  \end{figure}
% 首页简化之后,只保留了新版和旧版界面的入口,方便用户进行对比,同时,用户也可以在首页下面的tab页查看自己之前的诊断记录。

\subsubsection{健康诊断}

\begin{figure}
    \centering
    \includegraphics{images/diag.png}
    \caption{Caption}
    \label{fig:diag}
\end{figure}
新界面和之前版本的云中医界面不同的是,诊断界面不仅提供了面诊舌诊问诊的入口,同时会将面诊舌诊的照片和中间结果直接显示在当前页面,给用户对于自己当前身体情况一个直观的感受。如果拍照失败会直接显示,不需要等到用户进行点击诊断之后才知道自己的照片不合格。
	
\subsubsection{照片上传}

\begin{figure}[h]
    \centering
    \includegraphics{images/crop.png}
    \caption{Caption}
    \label{fig:crop}
\end{figure}
用户通过点击面诊或者舌诊的圆圈图案,可以通过从相册选择或者通过拍照,上传自己的脸部或者舌头照片。图片裁剪可以帮助用户定位面部和舌头的位置,提高诊断的精度。用户点击确认裁剪之后,诊断的结果会直接显示在圆圈下面,同时,如果没有识别到人脸或者舌头也会提示用户重新拍照。点击确认裁剪后,通过对图片进行base64编码上传到服务器的诊断接口,服务器调用对应算子,拿到诊断结果。

\subsubsection{问诊}

\begin{figure}[h]
    \centering
    \includegraphics{images/questions.png}
    \caption{问诊}
    \label{fig:questions}
\end{figure}
问诊的问题一共13道,用户根据自己的情况通过勾选回答问题。每次进入问诊界面,系统会尝试加载上一次用户回答问题的历史记录和对应的答案。同时,回答过的问题,会在界面上进行显示。没有回答过的问题,将是白色黑字没有边框。


用户在完成自己需要回答的问题或者面诊舌诊之后,可以点击蓝色的诊断按钮进行健康诊断,同时,系统会将本地诊断记录上传到后台服务器上,以便查询诊断记录。

\subsubsection{诊断结果}
\begin{figure}
    \centering
    \includegraphics{images/report.png}
    \caption{健康报告}
    \label{fig:report}
\end{figure}
系统根据用户个人的情况,会给出面诊结果,舌诊结果和最后的体质以及健康分数。此外,根据体质的不同,会给出和体质对应的健康建议,帮助用户进行针对性的健康调理。

\subsubsection{诊断记录列表}
\begin{figure}
    \centering
    \includegraphics{images/history.png}
    \caption{诊断历史}
    \label{fig:history}
\end{figure}
诊断记录页面,能够直观地给出用户近期的健康变化的情况。用户可以点击诊断记录,进入当时的详细诊断结果页面。

\section{透明性}
考虑到中医应用的特殊性,普通用户需要提高对应用的理解。因此我们在系统中加入算法的透明性,提升用户的体验。

对于每一个用户,我们通过hash算法,对用户名计算hash值,按照hash值得奇偶性,将奇数用户归类为不透明用户,偶数用户归类为透明用户。两类用户在进行面诊舌诊时的流程一样,但是透明用户能够看到背后特征提取算法和诊断算法的中间数据,同时系统为给出的诊断结果进行了解释。

\subsection{面诊舌诊的透明性}
普通用户在面诊页面

\subsection{问诊的透明性}

\subsection{诊断结果的透明性}
\begin{figure}
    \centering
    \includegraphics{images/report3.png}
    \caption{透明用户看到的诊断报告}
    \label{fig:my_label}
\end{figure}
普通用户在诊断结果页面,可以看到自己的健康分数和体质结果;透明用户可以点击诊断分数,了解这个分数是根据哪些指标,通过哪一个算法计算过来的。

\begin{figure}
    \centering
    \includegraphics{images/report7.png}
    \caption{解释分数相关问题}
    \label{fig:report_expalin_score_1}
\end{figure}

\begin{figure}
    \centering
    \includegraphics{images/report8.png}
    \caption{解释分数雷达图}
    \label{fig:report_expalin_score_2}
\end{figure}

\begin{figure}
    \centering
    \includegraphics{images/report9.png}
    \caption{解释分数计算公式}
    \label{fig:report_expalin_score_3}
\end{figure}

点击面诊结果,可以看到自己的面部舌部的对于整个诊断的影响。
\begin{figure}
    \centering
    \includegraphics{images/report4.png}
    \caption{Caption}
    \label{fig:my_label}
\end{figure}

点击体质分数,可以看到当次诊断中,面诊舌诊和用户自己回答的问题,哪些影响到了最后体质的判断。


\begin{figure}
    \centering
    \includegraphics{images/report5.png}
    \caption{体质的解释}
    \label{fig:report_explain_phy_1}
\end{figure}

\begin{figure}
    \centering
    \includegraphics{images/report6.png}
    \caption{体质相关问题}
    \label{fig:report_explain_phy_2}
\end{figure}

