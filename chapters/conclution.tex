\chapter{总结与展望}
\section{总结}
随着移动设备硬件水平的提高,各种日常场景下的健康应用随之出现;而在健康诊断信息化方面,人工智能的发展让许多健康诊断技术也能应用在实际生活中,其中就包括基于面部信息的健康诊断技术。我们针对日常场景下的面诊交互这一场景,发现虽然目前已有大量的日常健康的交互技术和大量的面诊信息化方法,但是主要都是针对专业人士或者在医疗诊所环境下使用,缺乏如何将面诊技术应用到日常场景下的研究。因此本文就这个研究点进行了以下工作:

\begin{enumerate}
	\item 使用定性研究的方法,利用技术探针招募了志愿者参与实验。结合实验数据,分析出日常场景下面诊技术的特点以及可能存在的问题,并总结出了如何设计日常场景下的面诊应用。

	\item 为了方便交互实验,设计并实现了支持面诊交互实验的实验平台,跨平台的实验屏蔽了用户不同设备类型可能会遇到的问题,统一的模型管理提高了模型的稳定性,方便新模型的接入和替换;任务管理对一次诊断任务进行了分解,提高多用户请求时的并发度;用户操作记录管理和问卷关联的功能,能够更加方便地设计新的问卷和采集用户信息,对用户细粒度的行为进行分析成为可能。

	\item 基于定性研究的设计方案,我们在实验平台上分别实现了两个原型系统,对日常健康下的可用性和透明性的问题在实验平台上进行了实验,验证了实验平台的效果的同时探索了对应的设计方案。
\end{enumerate}


% 不足之处


% 展望
\section{展望}
本文由于时间和精力有限,在实现了实验平台之后,只对交互问题中的可用性和透明性的问题进行了探索,未来可进行的交互研究工作如下:

\begin{enumerate}
	\item 设计如何增加系统的自适应。系统的自适应性包括对用户的自适应和对上下文的自适应。用户自适应性设计的实现需要采集用户信息,并分析出用户信息中和个人健康的相关点,上下文则包括短期的天气节气上下文信息和长期的用户使用记录。如何实现系统对用户数据的自适应性,并且如何在养生建议等步骤中加入上下文信息,还需要进一步地探索和设计实验。

	\item 与用户的日常行为融合。日常健康诊断在某些用户中不是一个日常的行为,用户不一定习惯于每天为专门为了检测自己的健康情况打开相机,而大多数用户会有自拍的习惯。面诊技术的特殊性在于它基于面部信息,如何将自拍和健康诊断结合起来,也是一个非常有意思的研究点。

	\item 探索日常诊断的社交功能。在第二章中,我们讨论了面诊信息的分享的特殊性:一方面有的用户考虑个人照片会泄露隐私,另一方面用户希望通过分享和其他人交流健康的知识。那么如何设计一个在不泄露用户隐私信息的情况下,实现用户之间的面部诊断信息分享和交流,还需要进一步地研究。
\end{enumerate}


% 同时,本文提出的实验平台是为了方便面诊交互实验而设计,因此在用户系统的基础上,添加深度访谈数据管理的功能也是非常必要的。