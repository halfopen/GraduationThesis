\chapter{总结与展望}
\section{总结}
随着移动设备硬件水平的提高,各种日常场景下的健康应用随之出现;而在健康诊断信息化方面,人工智能和移动设备的发展让许多健康诊断技术也能应用在实际生活中,其中就包括基于面部信息的健康诊断技术。
本文针对日常场景下的面诊交互这一场景,发现虽然目前已有大量正在使用的面诊系统,但是主要都是面向专业人士或者在医疗诊所环境下使用,缺乏如何将面诊技术应用到日常场景下的用户研究。
本文按照用户研究发现问题、提出解决方案、系统实现并验证的步骤,做了以下的工作:
\begin{enumerate}
	% todo
	\item 目前已有的面诊系统关于日常场景下的用户研究较少。为了深入探索面诊技术在日常场景中的应用问题,本文通过技术探针并开展了研究。

	\item 结合用户研究的数据,分析出面诊技术在日常场景下存在着自适应性、实用性和敏感性的问题,特别的是作为健康诊断工具的时候,可能会影响用户的情绪和对系统的信赖程度。
	针对本文发现的问题,提出了日常场景下面诊系统对应的设计策略:支持可持续使用、系统可解释性、日常可用性等。

	\item 设计出日常场景下通用可拓展的面诊系统:通过模型池、主从节点、读写分离、诊断任务建模等设计实现系统的通用性和可拓展性;通过加入上下文、诊断任务可选、设计健康看板式界面、加载历史信息等方式提高日常可用性;通过任务分解、文字描述、可视化元素、可交互元素、基于知识图谱的问答系统等方式实现的可解释设计提高用户对系统的理解和对结果的接受程度等。

	\item 对本文提出的系统设计进行实现并通过实验进行验证。
\end{enumerate}

% 不足之处

% 展望
\section{展望}
本文在后续的实验阶段,通过面诊系统对交互问题中的可用性和可解释性的问题进行了验证和探索,未来可进行的相关研究工作如下:
\begin{enumerate}
	\item 设计如何增加系统的自适应。系统的自适应性包括对用户的自适应和对上下文的自适应。用户自适应性设计的实现需要采集用户信息,并分析出用户信息中和个人健康的相关点,上下文则包括短期的天气节气上下文信息和长期的用户使用记录。如何实现系统对用户数据的自适应性,并且如何在养生建议等步骤中加入上下文信息,还需要进一步地探索和设计实验。

	\item 与用户的日常行为融合。日常健康诊断在某些用户的日常生活中不是一个日常的行为,用户不一定习惯于每天为专门为了检测自己的健康情况打开相机。而大多数用户会有自拍的习惯,面诊技术的特殊性在于它基于面部信息,如何将自拍和健康诊断结合起来,也是一个非常有意思的研究点。

	\item 探索日常诊断的社交功能。在第三章中,本文讨论了面诊信息的分享的特殊性:一方面有的用户考虑个人照片会泄露隐私,另一方面用户希望通过分享和其他人交流健康的知识。同时,在大量相关研究中,添加社交功能也是促进用户持续使用的常用方法之一。那么如何设计一个在不泄露用户隐私信息的情况下,实现用户之间的面部诊断信息分享和交流,还需要进一步地研究。
	
\end{enumerate}