\chapter{总结与展望}
\section{总结}
随着移动设备硬件水平的提高,各种日常健康场景下的健康应用随之出现;而在健康诊断信息化方面,人工智能和移动设备的发展让许多健康诊断技术也能应用在实际生活中,其中就包括基于面部信息的健康诊断技术。
本文针对日常健康场景下的面诊交互这一场景,发现虽然目前已有大量的日常健康的交互技术和大量的面诊信息化方法,但是主要都是面向专业人士或者在医疗诊所环境下使用,缺乏如何将面诊技术应用到日常健康场景下的交互研究。
本文按照交互研究发现问题、寻求解决方案、确定解决方案、系统实现、探索设计实现并验证的步骤,做了以下的工作:

\begin{enumerate}
	% todo
	\item 利用技术探针方法揭示面诊技术应用在日常健康场景下存在的交互设计挑战: 通过社交平台发布海报募了参与者参与实验并对在实验过程中及结束后对参与者进行了深度访谈以获取实验数据。
	
	\item 结合实验数据,分析出面诊技术在日常健康场景下存在着自适应性、实用性和敏感性的问题,特别的是作为健康诊断工具的时候,可能会影响用户的情绪和对系统的信赖程度。针对这些可能存在的问题,本文同时提出了对应:支持持续使用、系统可解释性、日常可用性的设计策略。

	\item 设计日常健康场景下可拓展的面诊系统,跨平台的设计屏蔽了用户不同设备类型可能会遇到的问题,统一的模型管理提高了模型的稳定性,方便新模型的接入和替换;任务管理对一次诊断任务进行了分解,提高多用户请求时的并发度。

	\item 对本文提出的系统设计进行了实现和验证。
\end{enumerate}

% 不足之处


% 展望
\section{展望}
本文在后续的实验阶段,通过面诊系统对交互问题中的可用性和可解释性的问题进行了验证和探索,未来可进行的交互研究工作如下:

\begin{enumerate}
	\item 设计如何增加系统的自适应。系统的自适应性包括对用户的自适应和对上下文的自适应。用户自适应性设计的实现需要采集用户信息,并分析出用户信息中和个人健康的相关点,上下文则包括短期的天气节气上下文信息和长期的用户使用记录。如何实现系统对用户数据的自适应性,并且如何在养生建议等步骤中加入上下文信息,还需要进一步地探索和设计实验。

	\item 与用户的日常行为融合。日常健康诊断在某些用户的日常生活中不是一个日常的行为,用户不一定习惯于每天为专门为了检测自己的健康情况打开相机。而大多数用户会有自拍的习惯,面诊技术的特殊性在于它基于面部信息,如何将自拍和健康诊断结合起来,也是一个非常有意思的研究点。

	\item 探索日常诊断的社交功能。在第二章中,本文讨论了面诊信息的分享的特殊性:一方面有的用户考虑个人照片会泄露隐私,另一方面用户希望通过分享和其他人交流健康的知识。同时,在大量相关研究中,添加社交功能也是促进用户持续使用的常用方法之一。那么如何设计一个在不泄露用户隐私信息的情况下,实现用户之间的面部诊断信息分享和交流,还需要进一步地研究。
	
\end{enumerate}