\chapter{致谢}

% 总结这三年的经历
三年飞快就过去了,经历了一波又一波实验室的师兄师姐毕业,转眼就快要到了自己的毕业的时间。 
这几年里,复旦大学优秀的学习环境,浓厚的学术氛围也给我留下了深刻的印象。
现在迎来了自己的毕业时刻,此刻才明白只有切身体会过才知道毕业这个字眼包含了太多太多情感。
这近三年的学院生活:有优秀的学友一起学习生活,探讨交流,提高了自己的眼界,收获到了很多;也有博学的师长的关爱成长。
在学习中懂得团队的重要性,在探讨中懂得方向的重要性,收获了成长,明白了奋斗与珍惜,拓展了眼界与知识。

% 感谢导师
首先感谢我的导师丁向华老师一直以来的包容和工作方面的支持。丁老师的学术功底扎实,科研态度严谨,关心学生。虽然直到学习了扎根理论才对这个领域有所了解。

% 感谢实验室其他老师
同样感谢实验室顾老师和卢老师三年来在学习和生活上的照顾,让我受益匪浅。

% 感谢实验室其他同学
感谢坐在我前面的张健豪师兄。作为实验室的优秀榜样,师兄指导了我们研究生规划和找工作的技巧,并传授了很多公司里学到的方法论。
感谢宋励师兄在我刚来实验室的时候,带领我阅读论文,一起打乒乓球,指导我学习安卓应用开发,一起熟悉实验室的环境。 
感谢刘鹏师兄与蒋特同学,一起完成实验室的项目,在我遇到困难时为我出谋划策。
同时感谢210实验室小团体其他同学三年来的陪伴以及在我经济困难时给予的帮助。

% 感谢家人和公司的同事
最后感谢实习公司的领导和同事,在实习期间我技术上的提升非常大。
作为一名毫无项目经验的小白,在公司里我了解到了规范的开发流程和成熟的业务解决方案,对论文的系统实现以及后续找工作起了很大的作用。

复旦的研究生活必将在我人生中留下浓墨重彩的一笔。