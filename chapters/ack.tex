\chapter*{致谢}
\rhead{致谢}
\addcontentsline{toc}{chapter}{致谢}

% 总结这三年的经历
三年飞快就过去了,经历了一波又一波实验室的师兄师姐毕业,转眼就快要到了自己的毕业的时间。 
这几年里,复旦大学优秀的学习环境,浓厚的学术氛围也给我留下了深刻的印象。
现在迎来了自己的毕业时刻,此刻才明白只有切身体会过才知道毕业这个字眼包含了太多太多情感。

% 感谢导师 todo: 多加一点
首先感谢我的导师丁向华老师一直以来在我犯错时的包容和工作方面的支持。丁老师的学术功底扎实,科研态度严谨,关心学生,。在完成论文的过程中,丁老师给出了很多关键的建议。
同样感谢实验室顾老师和卢老师三年来在学习和生活上的照顾,让我受益匪浅。感谢各位老师,让我有更多的时间和机会接触新的技术。感谢各位老师为实验室做的努力和贡献。

% 感谢实验室其他同学
感谢坐在我前面的张健豪师兄。作为实验室大家学习的优秀榜样,师兄指导了我们研究生规划和找工作的技巧,并传授了很多公司里学到的方法论。
感谢宋励师兄在我刚来实验室的时候,带领我阅读论文,一起打乒乓球,指导我学习安卓应用开发,一起熟悉实验室的环境。同时感谢张莹师姐在校期间给予的帮助。
感谢这几位师兄师姐在我刚来实验室的那一会儿帮助我迅速了解实验室的基本情况以及融入实验室的环境。

感谢刘鹏师兄与蒋特同学,一起完成实验室的项目,在我遇到困难时为我出谋划策。
刘鹏师兄作为实验室项目负责人,项目经验丰富,项目管理经验也令大家信服。项目遇到难点时,师兄对问题分析的非常透彻,对解决问题的思路也非常清晰。
在论文选题时,非常感谢师兄和我多次讨论和分析,最后才确定了大致的方向。

同时感谢210实验室小团体其他同学三年来的陪伴以及在我经济困难时给予的帮助。
在210小实验室这几年,桌子上总是经常出现同学出差或者回家带来的各地的零食,以及在午饭之后大家自费买的水果。实验室的同学在本文的调研过程中也给予了很大的帮助。

% 感谢公司的同事
最后感谢实习公司的领导和同事,特别是在实习期间我的mentor非常照顾我,实习期间感觉工作氛围非常轻松,项目中留给自己发挥的空间很大,收获非常多。
在准备串讲的那几周,不仅在工作之余内心回答我在看代码时遇到的问题,还在百忙之中抽空专门为我讲解项目。
后来准备论文的时候,部门负责人还和我讨论了论文的相关话题,并推荐了公司里几位做算法的同学一起交流,还特意让我留出时间专心写论文。
惭愧的是我最后论文进度还是一直停滞不前,秋招的时候也没有选择留在实习公司。
作为一名刚进去公司毫无项目经验的小白,在公司里我了解到了规范的开发流程和成熟的业务解决方案,技术上的提升非常大,对论文的系统实现以及后续找工作起了很大的作用。


% 感谢家人
这近三年的生活,有优秀的学友一起学习生活,探讨交流,提高了自己的眼界,收获到了很多;有博学的师长的关爱成长,
在学习中懂得团队的重要性,在探讨中懂得方向的重要性,收获了成长,明白了奋斗与珍惜,拓展了眼界与知识。最后统一感谢曾经帮助过关心过我的人,复旦的研究生活必将在我人生中留下浓墨重彩的一笔。