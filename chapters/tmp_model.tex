
\subsection{现有模型概况}

本系统提供了对各类诊断模型的支持,并不只是局限于以下提到的特定模型。
基于在云中医技术探针工作的基础上,目前现有两类已拆分的模型,特征提取模型(面部和舌部两种)和诊断打分模型,分别有对图片进行特征提取和对特征进行打分的能力。
特征提取模型对用户输入的特征(面部或者舌部图片)进行特征提取,而诊断打分模型,由一系列规则组成,定义了每一个特征对最终的打分输出影响的规则。


SVM分类器的训练数据来自各个医院的临床真实数据,诊断规则由上海中医药大学和专业的医生编写,规则的表现形式:把人的体质分为7大体质倾向(阳虚,阴虚,痰湿,瘀滞,脾虚,肾虚,气虚),根据用户的特征,修正体质倾向的数值,最后得到最终的体质和健康得分。
在后续随机抽取患者的对比实验中,准确率达到了90\%以上。

下面本小节将具体介绍拆分后模型的输入输出,其中模型的计算能力如表\ref{tab:face-feature}, 表\ref{tab:tongue-feature},表\ref{tab:diag-feature}所示。



\subsection{兴趣区域提取}
兴趣区域提取在面诊系统中则对应唇部舌部区域提取,本文采用了基于混合高斯模型的方法进行兴趣区域提取,在此做简单介绍。

\subsubsection{高斯分布}
高斯分布也称为正太分布,是自然界大量存在的最为常见的分布模式。

\begin{equation}
    \mathcal{N}(x | \mu, \Sigma) = \frac{1}{(2\pi)^{D/2}}\frac{1}{|\Sigma|^{1/2}}exp\{-\frac{1}{2}(x-\mu)^T\Sigma^{-1}(x-\mu)\}
\end{equation}

% \begin{figure}
% \includegraphics[width=200pt]{Guassian.png}
% \end{figure}
\subsubsection{混合高斯分布}
混合高斯模型是对高斯分布模型的拓展,假设数据的分布为多个高斯分布的线性叠加组成,相比简单高斯分布对复杂数据的刻画能力更强。
\begin{equation}
p(x) = \Sigma^K_{k=1}\pi_k\mathcal{N}(x|\mu_k,\Sigma_k)
\end{equation}
其中
\begin{equation}
\Sigma^K_{k=1}\pi_k = 1
\end{equation}

混合高斯模型在图像处理中经常用来完成图像分割任务:通过对图像中的像素点进行混合高斯模型建模,可以得到图像上每个点属于某个高斯模型的概率,进而通过概率值将图像分割为多个区域。

\subsubsection{基于混合高斯模型的区域提取}

利用混合高斯模型进行兴趣区域提取,可以截取唇部区域,如研究\cite{Hu2016Robust}所述主要步骤如下:
\begin{enumerate}
    \item 使用人脸检测算法检测图片中是否包含人脸,裁剪人脸区域作为下一步输入。

    \item 对人脸上半部分无唇部的区域使用混合高斯模型获取当前图像的肤色概率分布p。

    \item 根据经验值,预先设置好肤色裁剪比例参数列表\textit{it[N]}。

    \item 迭代式地判断\textit{N}次,对于每次迭代\textit{i},在人脸下半部分定位唇部:(1)按照预先设置的参数\textit{it[i]},通过比较每个像素点的肤色概率\textit{p[x]}和参数\textit{t[i]}, 如果\textit{p[x]>t[i]}, 去除皮肤区域。最终得到去除部分肤色区域的图像。
(2)在图像中根据唇部的形状与色彩特征,定位到唇部个数:如果个数为1标记为找到唇部区域,继续迭代;否则标记为未找到唇部区域,直接退出。

\item 在上一步获取到的有唇部区域的图像中,按照图像中的肤色区域大小变化率绘制曲线,取曲线的局部最小值对应的唇部区域作为最终的唇部裁剪区域。

\item 对唇部轮廓进行简化平滑处理,截取出唇部区域。
\end{enumerate}

在上述方法中,\textit{N}为迭代次数,一般设置为14,裁剪比例参数为\{0.6, 0.5, 0.4, 0.3, 0.2, 0.1, 0.05, 0.01, 0.005, 0.001, 0.0005, 0.0001, 0.00005, 0.00001\}。
该方法相比\cite{rahman2014lip, li2010automatic, li2020tcminet}等方法相对简单且准确率相对比较高,鲁棒性更强。





\subsection{脸部特征提取}




\subsection{舌部特征提取}

\begin{table}[h]
    \centering
    \caption{舌部特征提取}
    \begin{tabular}{lll}
        \toprule
        特征 & 特征描述 & 特征内容 \\ 
        \midrule
        tongueDetectRes & 舌体 & 0:未检测出舌像、1:成功检测出舌像 \\
        tongueCrack & 舌裂纹 & 0:未检测到裂纹、1:成功检测到裂纹 \\ 
        tongueFatThin & 舌胖瘦 & 0:正常(瘦)、1:胖舌 \\
        tongueCoatThickness & 舌苔厚薄 & 0:薄、1:厚 \\
        tongueCoatColor & 舌苔颜色 & 0:苔白、1:苔黄 \\
        tongueNatureColor & 舌质颜色 & 0:舌暗红、1:舌淡白、2:舌淡红、3:舌红、4:舌紫\\
        \bottomrule
    \end{tabular}
    \label{tab:tongue-feature}
\end{table}

如表\ref{tab:tongue-feature}所示,舌部特征提取,输入为舌头图片,输出有以下几个维度:
\begin{enumerate}
    \item 是否检测到舌体(tongueDetectRes): 如果没有检测到舌体,则剩余所有维度无效,且取值为0。

    \item 是否检测到舌裂纹(tongueCrack): 舌裂纹是最终特征,剩余特征是否有效和该标志位没有关系。

    \item 舌头特征(tongueFatThin、tongueCoatThickness、tongueCoatColor、tongueNatureColor): 包括舌胖瘦,舌苔的厚薄、颜色和舌质颜色。

\end{enumerate}
