\chapter{绪论}

\section{研究背景}
% 社会背景
改革开放以来,随着我国经济的高速发展,社会竞争的日趋激烈,人们的生活节奏逐渐加快,收入也不断提高。在收入提高的同时,工作压力的增大而导致的亚健康问题日渐普遍。根据《中国人健康大数据》\cite{中国人健康大数据} 调查,目前我国已经有70\%左右,也就是超过9亿的人处于亚健康的状态中。亚健康作为一种临界状态,虽然没有明确的疾病,但却出现精神活力、适应能力和反应能力的下降等问题。在亚健康已成为国民健康大隐患的今天,如果不重视会带来严重后果。人们在享受物质财富的同时,也在追求身心健康,养生理念在社会中逐渐兴起,对于日常健康管理的应用的需求也越来越大;但与之矛盾的是因为生活节奏快,大部分人没有专门去诊所看病或者咨询医生的时间。

% 面部不仅能表达一个人的情绪,也是反映一个人精神面貌的重要部位。中国文化历史悠久,在长期的积累中,中医方面有着丰富的面诊理论。中医有经验充足的面诊理论基础,大量学者也为中医面诊信息化做了实证研究。中医认为,内脏的病变可以反映到体表,相反,通过对外部的诊察,也可以推测内脏的变化,面部是最快表现脏腑病症的部位,通过观察面部的颜色、形状、五官状况等,可以快捷大致地诊断脏腑疾病。

% 人机交互方面

% 理论背景
健康诊断的信息化也取得了一定的进展。传统中医诊断主要包括脉诊、面诊、舌诊和问诊。目前已有很多研究在探索如何用计算机技术来辅助医生完成健康诊断。如市面上和现在中医诊所内已经有的面诊仪来帮助医生采集用户面部信息到电脑中,脉诊仪可以由计算机根据用户的脉象得出脉象的基本特征辅助医生等\cite{Zhang2018Study2}。在应用方面,有些公司推出了个人健康管理应用,常见的有血糖,体脂体重管理等,人机交互领域长期也来也有大量的研究关注于如何帮助用户更好地进行慢性病管理及健康管理。计算机诊断应用有使用远程辅助的方式让医生帮助用户进行远程面诊。

% 技术背景
近年来随着机器学习特别是深度学习的发展,深度神经网络模型在各大竞赛中已经超越了人类水平。人工智能面部读取分析技术越来越成熟且应用越发广泛,如人脸识别可用于签到系统,门禁系统; 表情识别可用于检测疲劳驾驶,制作测谎工具; 脸部特征检测则可以做到检测鼻子大小检测脸型等。

% 场景背景
于此同时,移动设备近几年也在迅速发展,无论是手机拍照的清晰度还是硬件的计算能力都有了很大的提高。由于移动操作系统本身的优越性加上硬件技术的迅速发展,移动设备的计算能力和存储能力不断增强,移动设备可以处理越来越复杂的任务,在很多方面逐渐取代传统PC设备。更重要的是大多数人都有日常自拍的习惯,自拍已经成为很多人日常生活中的一部分。



由此看来,相信不久的将来,手机上用于日常健康管理的成熟的面诊应用不会有很大的阻碍。但是,和诊所环境相比,日常健康追踪场景下主要有以下几个特点需要考虑: 

(1) 应用的长期使用。日常健康追踪对于用户来说,是一个长期而持续的一个过程。 如何吸引用户继续使用、如果提高日常使用场景下交互的便捷性、如何引导用户改变自己的生活方式,提高生活质量是需要考虑的问题。

(2) 设备的多样性。 医院或者诊所环境下,用户是在专门的设备下进行健康诊断。而在日常场景下,所有操作都在用户的移动设备上完成。用户的移动设备具有多样性,从移动操作系统到硬件计算能力和存储空间的都存在很大的差异性。

(3) 用户的专业素养。与医院或者诊所环境下不同,用户在使用过程中是没有专业的医生或者有可询问的技术人员,帮助用户理解诊断过程和诊断结果难度更大。

目前看来从人机交互的角度我们并不知道用户在日常使用场景下使用面诊应用会遇到哪些问题,面诊应用在日常实际应用的时候会有哪些特殊性质,设计应用的时候应该注意哪些方面等,关于日常健康管理方面的面诊应用的交互研究是一个非常新颖的研究课题且有重大的意义。

% \section{国内外研究现状}

% \subsection{基于中医的健康诊断技术}

% 目前基于中医的健康诊断信息技术的研究,很多应用场景都是在局限在医院及诊所场景下。目前的医疗信息系统,主要可以分为一下几个方向:
% 医疗数据采集,医疗数据存储,诊断数据管理如电子病历系统,医护信息系统,电子影像系统等。

% 按照研究性质也可划分为新的算法研究和硬件应用研究,算法研究如基于肤色的肝病判别方法,硬件研究如面诊仪,舌诊仪等。

% 当前的基于中医的健康诊断技术研究的局限性在于,大多数系统或者应用只能应用在诊所环境,不方便用户在日常环境下使用。
% 在日常环境下,需要考虑设备的便携性和计算速度,用户的长期使用等;同时在用户的使用过程中,身边是没有专业的医生进行指导的。

% \subsection{日常健康技术}

% 在人机交互领域,如何更好地设计健康相关的技术和应用,也一直是研究的热点。

% 和诊断技术研究不同,近几年人机交互研究可以大致分类两类: 第一类是在用户没有生病的情况下,鼓励用户进行日常健康的方式, 如更好地锻炼,饮食,睡眠等。健康监测和跟踪系统的主要作用,就是能够让用户得到关于自己健康的反馈。目前这类健康检测系统监测的指标主要是

% 第二类则是关于慢性病的长期管理和追踪技术,提高用户的生活品质。如糖尿病\cite{mamykina2008mahi:}, 痴呆\cite{yasuda2009remote}, 多发性硬化\cite{ayobi2017quantifying},双向情感障碍症\cite{bardram2013designing},以及多种慢性病的管理\cite{nunes2015self-care}。

% 但是目前在人机交互领域的健康技术,还没有涉及到通过面部信息去进行健康管理的研究。
% 很多都是日常的

% 没有生病,鼓励健康生活方法

% 慢性病管理

% 在人机交互领域,更好地设计健康追踪技术和应用,一直是研究的热点。在以往的研究中,大多数研究者探索了糖尿病,高血压等慢性病追踪技术。


% 但是,目前的研究还没有涉及到基于面部舌部特征进行健康评估的技术。

% 健康相关的人机交互的研究,主要集中
% 关于算法以及人工智能的可解释性(XAI的研究现状)


% 用户对系统的期望会影响他们对技术的接受程度。

% 如何让用户能够接受一个不完美的系统呢?

% 有三类调整用户期望的方法: [will you accept imperfect ai? ]
% 1. 给出准确的数值,比如显示目前结果的置信程度
% 2. 给出示例提高用户的理解程度
% 3. 允许用户控制,调整系统的参数进行结果的对比


% ai算法的理解度和透明方法
% 1. 事后进行解释  Most proposed approaches
% also require significant end-user interest and effort in understanding system behavior, which can be inefficient in many
% scenarios

% 缺乏系统的认识度,会导致对系统的信任度下降,主动提供系统的相关信息以及决策的过程,可以提高用户的满意度和接受程度。

\section{本文研究内容与创新}

针对目前的基于面部诊断的技术大多是在诊所环境下使用的问题,本文以日常环境下中医面诊应用的交互技术作为研究问题,
使用定性研究的方法设计交互实验,深入探索了目前基于面部诊断应用在日常使用场景下的一系列的交互问题,希望能为将来成熟的面诊应用提供交互方面的指导。

本文的主要研究工作有以下几点:

(1) 对基于面诊养生应用在日常使用的场景下进行了定性研究,选用云中医作为技术探针设计并开展实验,探索了日常使用环境下面诊应用的特点以及待解决的问题,并提出了三条对应的日常场景下面诊应用的设计原则。

(2) 为了支持后续的深入研究,设计并开发了一套支持后续面诊交互研究的实验平台,提供了版本管理,日志管理,模型管理,问卷关联等功能,方便接入其他新的人脸模型快速开展人机交互的实验,并降低了系统对用户设备类型和性能的要求。

(3) 在第二个工作点的基础上,利用实验平台,设计并实现了两个实验用的原型系统,分别对第一个工作点提出的设计原则进行了验证实验。实验的结果和分析表明,本文提出的提高可用性的设计规则的确能够提高系统使用的便利性,能够满足用户日常使用的需求;而本文提出的增加系统透明性的设计原则,的确能够在目前算法的结果不完美的情况下,提高用户对系统的信赖程度和对结果的接受程度。


% 2、从数据中分析出中医面诊应用潜在的使用场景,提出了实用性敏感性自适应性问题的解决方法。
% 3、实现一个方便易用的在线中医面诊的系统,对界面进行了改进。在面诊舌诊问诊过程中加入了透明性,将内部算法的细节暴露够用户。
% 4、对算法透明性的作用进行了用户调研。

% 支持后续的研究
% 拿出透明性作为一个研究案例,也可以做其他的交互研究的实验

% // 系统设计

% 跨平台,版本管理,日志管理, 流程建模

% 模型分类:打分模型,分类模型
% 模型执行:指定先后关系
% 结果合并:合并任务

% 规则系统:

% 方便后续接入其他的人脸模型,

% 模型在线更新

% 面诊技术设计原则 // 

% 抽象成平台

% 平台可以做什么事情

% 模型系统剥离  

\section{论文结构}
本文按照一共组织了六个章节,具体的组织结构如下:

第一章,介绍了本文工作的研究背景,分析了日常使用场景下的特点,最后介绍了本文的主要工作和创新点。

第二章,介绍了和本文工作相关的研究情况,分人机交互领域的日常健康交互技术和诊断技术信息化两个方面分别进行了相关的介绍。

第三章,就本文的研究问题,采用技术探针的方法进行了定性研究,设计了调研问卷和深度访谈。通过对访谈数据的仔细分析,总结出了日常场景下面诊技术的特点,发现了其中存在的问题,并给出日常场景下面诊技术的设计规则。

第四章,为了方便后续的研究工作,介绍了本文设计并实现的用于面诊交互实验的实验平台,该平台提供了日志管理,问卷关联,模型管理,任务管理等功能,并在客户端通过重新设计交互界面,解决了第三章提到的影响用户使用的交互可用性问题。

第五章,在实验平台的基础上,对定性研究提出的结论进行了两个验证实验,验证了第二章工作的有效性。

第六章,总结了文本的研究内容和不足之处,并对未来工作进行了展望。