\chapter{绪论}

% 移动技术人工智能技术的快速发展也为支持日常健康诊断和管理提供了有利的技术支持。

% 从一般的技术,提出本文主要关注面诊技术?

社会的发展和经济的进步,导致人们的生活节奏越来越快。高强度的工作让处于亚健康状态的人们越来越多,健康养生也愈发成为当前热门的话题。同时,有限及分配不平衡的医疗资源经常不能及时地满足广大群众健康管理的需求,大部分用户去见专业医生时需要提前预约或者经历长时间的排队\cite{雷鹏中国医疗资源配置与服务利用现状评价}。
日常健康管理技术是解决这一问题的重要途径。传统的健康管理技术可通过采集用户的心率,体重,血压等信息\cite{Liu2014Computerized, Wang2004An, Shu2007Developing}(通常需要借助额外的传感器等设备完成数据的采集),帮助用户进行完成日常管理。
在这些可获取的信息中,面部信息除了用于识别身份,表达情绪之外,也包含大量和健康相关的特征,通过面部信息大致判断个人身体状况的研究也有充足的理论基础\cite{杨淑芳2002伤寒六经辨证初探, Clifford2006Shortliffe}。
随着读脸技术的不断深入,利用面部信息进行帮助用户健康管理即面诊技术也在探索中。

% 近些年来随着移动设备的普及和性能的提升,移动设备能方面地获取用户健康信息相关信息,也有大量工作研究如何利用移动设备作为健康管理平台。

然而,现有的面诊技术与研究大多局限于医疗环境、而日常场景下中的交互研究缺乏利用面部信息进行健康管理深入探索。那么日常场景下的面诊技术的应用在交互设计层面会遇到哪些问题?面诊应用在日常使用场景下会有哪些特点?研究这一系列问题,将有助于面诊技术应用到日常场景中,满足用户日常健康管理的需要。

% 随着计算机技术的进步和移动设备的发展,越来越多的日常健康技术也如雨后春笋般出现来满足用户的这一需求,如何设计日常健康技术也成为了研究者关注的重点。
% 本文的关注点是面诊技术在日常健康场景下的问题。面诊技术的发展和读脸技术和医疗信息化密切相关,因此本章介绍了读脸技术和诊断技术的发展,引出了本文的主要工作,最后介绍了论文的组织结构。

\section{研究背景及意义}

% 经济发展,关注健康问题
% 
随着人工智能特别是基于深度学习的计算机视觉技术的发展,智能读脸技术变得日益成熟和广泛。
% 长期以来,人的面部在人与人之间相互识别,情感交流等方面起着重要作用。
% 随着人工智能特别是基于深度学习的计算机视觉技术的发展,深度神经网络能力越来越强,在有些领域中人工智能甚至已经超越了人类水平\cite{he2015delving}。
读脸技术指的是通过读取人的面部图片获取相关的信息,主要应用有人脸检测、人脸识别、情绪识别、性别识别、年龄识别等,基于面部图像分析的读脸技术在日常生活中的应用也越来越广泛。
例如,通过将人脸图像与人脸库中图像进行比较来识别人脸的身份的人脸识别技术\cite{Zhang2016Joint, Schroff2015FaceNet},已广泛应用于考勤管理\cite{surekha2017attendance},访问控制\cite{atick2000continuous}和安防领域\cite{liu2005ibotguard}。
此外,还有相关研究探索了使用读脸技术进行情感识别\cite{corneanu2016survey},识别面相\cite{Li2007Online, Tempark2012Chinese},甚至可以用于智商检测\cite{Kleisner2014Perceived}。
随着读脸技术的不断进步,其在人们日常生活场景下的实际应用也越来越普遍。
% 面诊背景及日常工具化

读脸技术在健康诊断方面有深厚文化基础和理论基础。
中国文化历史悠久,观察人的面部在中国也有深厚的文化基础。在长期的积累中,面部除了在人与人之间相互识别,情感交流等方面起着重要作用,还可以指示一个人的整体心理和情感健康状况,如显示出疲劳、抑郁的状态等,甚至有一系列专门研究面相的理论。面诊作为传统中医四诊(望闻问切)中重要的一步,在中国有深远的影响,大部分人都接受或理解通过观察面部进行健康诊断的方式。
大量研究表明,面部是反映一个人身体健康和精神面貌的重要部位,能够透露出各种健康状况甚至体内疾病的迹象。
在现代医学领域,当医生评估患者的总体体质并得出可能疾病的初步假设时,直接观察面部是医学诊断的一部分\cite{Clifford2006Shortliffe}。
内部器官的病变会导致面部出现对应的症状,通过观察面部对应的症状,如面部的颜色、形状、斑点等,可以大致推测内部器官的情况\cite{杨淑芳2002伤寒六经辨证初探}。
此外,医疗领域不仅有经验充足的面诊理论基础,大量学者也为面诊信息化做了实证研究,如面色和乙肝关系的研究\cite{杨宏志2007慢性乙型肝炎肝硬化中医面部五色诊断与临床病理的相关性研究, Li2008Therapeutic}、基于面部诊断黄疸\cite{艾英1998黄疸病人面部色泽定量实验研究}等。

% 特别的是,面诊是中医四诊(望闻问切)的重要组成部分。

% 

% 近些年来,随着社会的发展,人们的生活节奏不断加快,各种亚健康问题随之出现,越来越多的人注重自己的健康,人们对健康管理技术的需求也越发强烈。
% 当前已开发出多种用于临床目的的健康应用,包括专业的仪器如面诊仪\cite{Liu2014Computerized}以及各类辅助医护人员的医疗系统。
% 直到最近,信息技术的进步和移动设备的便携性增强才将其中一些工具变成了日常用户可使用的设备,健康管理的相关应用正从诊所环境转移到日常场景下。各类可穿戴和便携式的脉冲读取设备,如金姆健康推出的指尖脉搏血氧仪———金姆脉诊仪\footnote{http://www.jinmuhealth.com},已商业化用于日常健康监测。相比当前的基于穿戴式和便携式设备的日常健康诊断方法,基于面诊技术的日常健康管理有明显的优势:(1)不需要额外的设备。当前的移动设备可以方便地获取用户的面部信息,基于面诊技术的日常健康管理应用成本更低,日常使用也更加方便。(2)大部分人在使用手机的过程中都有拍照的习惯,面诊也有悠久的文化基础和扎实的理论基础,基于面诊技术的日常健康管理可以融入到日常场景中。

% 当前基于读脸技术已开发出多种用于临床目的的健康应用,包括专业的仪器如面诊仪\cite{Liu2014Computerized}以及各类辅助医护人员的医疗系统。
% 直到最近,信息技术的进步和移动设备的便携性增强才将其中一些工具变成了日常用户可使用的设备,健康管理的相关应用正从诊所环境转移到日常场景下。各类可穿戴和便携式的脉冲读取设备,如金姆健康推出的指尖脉搏血氧仪———金姆脉诊仪\footnote{http://www.jinmuhealth.com},已商业化用于日常健康监测。


随着技术的发展,将基于读脸技术的计算机面诊技术引入到日常健康场景下在技术上是可行的。相比当前的基于穿戴式和便携式设备的日常健康诊断方法,基于面诊技术的日常健康管理有明显的优势:

\begin{itemize}
    \item 不需要额外的设备。当前的移动设备可以方便地获取用户的面部信息,基于面诊技术的日常健康管理应用成本更低,日常使用也更加方便。
    
    \item 大部分人在日常生活中有使用手机自拍的习惯,面诊也有悠久的文化基础和扎实的理论基础,基于面诊技术的日常健康管理可以融入到日常场景中。
\end{itemize}

在如何设计系统,将读脸技术集成到日常健康场景中,相关的应用与交互研究还远远不够。研究读脸技术在日常场景下的交互问题,了解系统应该如何设计,需要通过实地调研,分析读脸技术在日常场景下的特点,了解用户的需要以及遇到的问题,帮助用户更好地使用基于读脸技术的日常健康管理应用。
深入研究读脸技术在日常健康场景下的交互问题,将有助于将读脸技术从专业的诊所环境应用到日常健康场景,从而在一定程度上缓解当前整体公共医疗资源不足的压力,改善用户的生活品质。


% 由此看来,在医疗资源不足,不同地区资源分配不均匀\cite{雷鹏中国医疗资源配置与服务利用现状评价}的社会背景下,如何设计系统,将面诊技术整合到日常健康实践中有重大的研究意义和挑战。

\section{研究现状与存在的问题}

% 读脸技术和面诊理论的发展使得将计算机面诊技术引入到日常健康场景下在技术上是可行的,然而在如何设计系统,将读脸技术集成到日常健康场景中,相关的人机交互研究还远远不够。
% 简单概括相关工作

\subsection{读脸技术的应用}
% 读脸技术发展背景
读脸技术的应用非常广泛,与本文相关领域主要可分为医疗健康与日常场景的应用。

在医疗健康领域,读脸技术主要帮助提高医护人员工作效率,如快速获取病人信息\cite{nwosu2016mobile}、远程监控\cite{Hossain2015Cloud}、辅助采集面部特征\cite{张红凯2015中医面诊信息采集与识别方法研究进展}等。这类技术特点主要有两点:(1)为专业的医护人员设计。医护人员有丰富的专业知识,有足够的时间学习系统如何使用。(2)在诊所的标准环境下使用。与日常场景相比,通常诊所环境使用的都是价格昂贵的标准的设备来保证系统的可靠性。
那么在日常场景下,这类技术是否能够满足用户使用的需要以及会出现什么问题仍亟待进一步的研究。

在日常场景下,许多研究致力于如何将读脸技术融入到用户的日常生活中,如将读脸技术与冰箱结合起来,提醒用户在日常生活中保持微笑\cite{Tsujita2011Smiling};将读脸技术与镜子结合起来,在用户照镜子的时候检测用户的健康状况\cite{andreu2015mirror}等。这类技术能够将读脸技术融入到用户的日常生活中,但缺点是不够便携而且需要结合额外的硬件设备成本。
% 存在问题

\subsection{日常健康场景的交互研究}
关于日常场景下与健康相关的交互技术,人机交互领域长期以来有大量的相关研究,主要可以分为慢性病管理和健康行为引导两个方面。

以慢性病管理为例,慢性病作为一种无法治愈的疾病,病人通过日常的健康管理能够增加对病情的控制从而提高生活的品质,例如研究如何帮助糖尿病\cite{mamykina2008mahi}、痴呆\cite{yasuda2009remote}以及多发性硬化\cite{ayobi2017quantifying}等病人在日常场景下更好地进行检测以及提高控制意识等。在健康行工作为引导方面,相关研究则关注于通过监控日常行为\cite{purpura2011fit4life,Inagawa2013A,bravata2007using,cordeiro2015barriers,lin2006fish, miller2014stepstream}以及健康指标\cite{kay2012lullaby,gronvall2013beyond,logan2007mobile,walters2010a}、用户行为劝导、吸引用户持续使用等技术,指导处于亚健康状态的用户更加健康地生活。

这类技术对日常场景下的健康管理有非常深入的交互研究,包括各类慢性病以及各类用户行为以及健康指标的监控等,关于如何利用面部信息来支持日常健康管理的研究还非常缺乏。
% 存在问题

\section{本文研究内容与创新}

本文以日常场景下基于读脸技术的面诊应用的交互问题展开研究,通过技术探针的用户设计研究,搭建实验平台,利用平台探索交互设计方案。

本文主要对以下几个方面进行了探索:

\subsection{基于技术探针的用户设计研究}

目前大部分的面诊技术主要应用场景在诊所环境下,我们对基于读脸技术的面诊应用在日常场景下的使用进行了交互研究,选用云中医作为技术探针设计并开展实验,探索了日常健康场景下面诊技术的特点以及待解决的问题,并提出了对应的日常场景下如何设计面诊应用的解决方案。

通过用户调研和数据分析我们发现,和诊所环境相比,日常健康场景下基于面诊技术的健康管理应用主要有以下几点问题亟待解决: 

\begin{itemize}
    
    \item 长期使用。诊所环境下,面诊技术只考虑用户在就诊时的本次使用。日常健康场景中的面诊技术的使用需要系统能够和用户的生活融合起来,是一个长期的过程。在日常场景下,如何吸引用户继续使用、如果提高日常使用场景下交互的便捷性、如何引导用户改变自己的生活方式,提高生活质量等也是需要考虑的问题。

    \item 设备的多样性。在诊所环境下,用户是在专门的设备下进行健康诊断。而在日常场景下,所有操作都在用户的移动设备上完成。用户的移动设备差异性比较大,特别是安卓系统的碎片化问题非常严重。不同用户的设备从移动操作系统到硬件计算能力和存储空间的都存在很大的差异性。
    
    \item 用户的多样性。各类用户的受教育程度,日常空余时间,职业、工作场合是需要考虑的因素。比如在日常使用的时候,与诊所环境下不同,用户在使用过程中可询问的专业技术人员,大部分用户没有医疗相关的专业知识,使用起来会比较困难,在日常场景下帮助用户理解诊断过程和诊断结果难度更大。

\end{itemize}

对于以上的问题,我们提出了增加可变性,系统可解释性,日常可用性的三个设计方案(详细的设计方案在第二章中展开)。

\subsection{实验平台设计与实现}

为了支持后续的深入研究以及解决设备多样性的问题,设计并开发了一套支持后续面诊交互研究的实验平台。该实验平台实现了以下几个方面的功能。

\begin{itemize}

    \item 跨平台实现与模型服务化。本文采用了mui跨平台方案与容器化技术,让模型在服务端运行,屏蔽了用户的设备差异。通过容器化技术将算法模型打包成独立的服务,提高了系统的整体服务并发度和稳定性。

    \item 用户操作记录管理。通过记录并上传用户的所有操作,为后续的实验提供数据分析的基础。

    \item 第三方问卷系统支持。通过关联第三方的问卷系统,每次进行交互实验时能够灵活地设计新的用户调研问卷,同时能方便地使用第三方问卷系统提供的数据分析、质量控制等功能。

\end{itemize}

通过这些功能设计,该平台能够方便接入其他新的人脸模型,快速开展人机交互的实验,并降低了系统对用户设备类型和性能的要求。


\subsection{设计方案实现与验证实验}

% (多加一点)

在第二个工作点的基础上,利用实验平台,具体实现了第一个工作点提出的设计方案,并在实验平台上分别实现了原型系统并进行了验证实验。
在交互设计实现方面,本文探索了如何实现高可用性设计以及如何对面诊的过程、体质倾向的影响权重进行解释。
结果表明,本文提出的提高可用性的设计方案能提高系统使用的便利性,方便用户在日常场景下使用;
而本文提出的增加系统解释性的设计方案,能在目前系统算法结果不完美的情况下,提高用户对系统的信赖程度和对结果的接受程度。


\section{论文结构}
本文按照一共组织了六个章节,具体的论文结构如下:

第一章,绪论。介绍了本文工作的研究背景和研究意义,概括了当前的研究现状和存在的问题,最后介绍了本文的主要工作和创新点。

第二章,相关工作。介绍了和本文工作相关的研究情况,分读脸技术的应用和日常健康场景的交互研究两个方面分别进行了相关的介绍。

第三章,基于技术探针的交互研究。就本文的研究问题,采用技术探针的方法进行了交互研究,设计了调研问卷和深度访谈。通过对访谈数据的仔细分析,总结出了日常场景下面诊技术的特点,发现了其中存在的问题,并给出日常场景下面诊技术的设计方案。

第四章,实验平台设计与实现。介绍了本文设计并实现的用于面诊交互实验的实验平台,该平台提供了用户操作记录管理,问卷关联,模型管理,任务管理等功能。该系统设计提高了后续交互实验的效率,解决了第三章提到的影响用户使用的设备多样性问题。

第五章,设计方案实现及验证实验。在实验平台的基础上,探索如何具体实现可解释性和日常可用性设计,并通过两个验证实验,验证了工作的有效性。

第六章,总结与展望。总结了本文的三个主要工作内容和不足之处,并对后续工作进行了展望。