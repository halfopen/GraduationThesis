\chapter{绪论}

\section{研究背景}
改革开放以来,随着我国经济的高速发展,社会竞争的日趋激烈,人们的生活节奏逐渐加快,收入也不断提高。在收入提高的同时,工作压力的增大而导致的亚健康问题日渐普遍。根据《中国人健康大数据》调查,目前我国已经有70\%左右,也就是超过9亿的人处于亚健康的状态中。亚健康作为一种临界状态,虽然没有明确的疾病,但却出现精神活力、适应能力和反应能力的下降等问题。在亚健康已成为国民健康大隐患的今天,如果不重视会带来严重后果。人们在享受物质财富的同时,也在追求身心健康,养生理念在社会中逐渐兴起,对于日常健康管理的应用的需求也越来越大;但与之矛盾的是因为生活节奏快,大部分人没有专门去诊所看病或者咨询医生的时间。

面部不仅能表达一个人的情绪,也是反映一个人精神面貌的重要部位。中国文化历史悠久,在长期的积累中,中医方面有着丰富的面诊理论。中医有经验充足的面诊理论基础,大量学者也为中医面诊信息化做了实证研究。中医认为,内脏的病变可以反映到体表,相反,通过对外部的诊察,也可以推测内脏的变化,面部是最快表现脏腑病症的部位,通过观察面部的颜色、形状、五官状况等,可以快捷大致地诊断脏腑疾病。

其次,传统中医诊断的信息化也取得了一定的进展。传统中医诊断主要包括脉诊、面诊、舌诊和问诊。目前已有很多研究在探索如何用计算机技术来辅助医生完成健康诊断。如市面上和现在中医诊所内已经有的面诊仪来帮助医生采集用户面部信息到电脑中,脉诊仪可以由计算机根据用户的脉象得出脉象的基本特征辅助医生等\cite{Zhang2018Study2}。在应用方面,有些公司推出了个人健康管理应用,常见的有血糖,体脂体重管理等,人机交互领域长期也来也有大量的研究关注于如何帮助用户更好地进行慢性病管理及健康管理。计算机诊断应用有使用远程辅助的方式让医生帮助用户进行远程面诊。

近年来随着机器学习特别是深度学习的发展,深度神经网络模型在各大竞赛中已经超越了人类水平。人工智能面部读取分析技术越来越成熟且应用越发广泛,如人脸识别可用于签到系统,门禁系统; 表情识别可用于检测疲劳驾驶,制作测谎工具; 脸部特征检测则可以做到检测鼻子大小检测脸型等。

于此同时,移动设备近几年也在迅速发展,无论是手机拍照的清晰度还是硬件的计算能力都有了很大的提高。由于IOS和Android系统本身的优越性加上硬件技术的迅速发展,移动设备的计算能力和存储能力不断增强,移动设备可以处理越来越复杂的任务,在很多方面逐渐取代传统PC设备。更重要的是大多数人都有日常自拍的习惯,自拍已经成为很多人日常生活中的一部分。

由此看来,相信不久的将来,手机上用于日常健康管理的成熟的面诊应用不会有很大的阻碍。但是,就目前看来从人机交互的角度我们并不知道用户在日常使用场景下,面诊应用会遇到哪些问题?面诊应用在日常实际应用的时候会有哪些特殊性质?设计应用的时候应该注意哪些方面?关于日常健康管理方面的面诊应用的交互研究有重大的意义。

\section{国内外研究现状}
健康相关的人机交互的研究,主要集中
关于算法以及人工智能的可解释性(XAI的研究现状)


用户对系统的期望会影响他们对技术的接受程度。

如何让用户能够接受一个不完美的系统呢?

有三类调整用户期望的方法: [will you accept imperfect ai? ]
1. 给出准确的数值,比如显示目前结果的置信程度
2. 给出示例提高用户的理解程度
3. 允许用户控制,调整系统的参数进行结果的对比


ai算法的理解度和透明方法
1. 事后进行解释  Most proposed approaches
also require significant end-user interest and effort in understanding system behavior, which can be inefficient in many
scenarios

缺乏系统的认识度,会导致对系统的信任度下降,主动提供系统的相关信息以及决策的过程,可以提高用户的满意度和接受程度。

\section{本文研究内容与创新}

本文的主要研究工作有以下几点:
1、对中医养生应用在日常使用的场景下进行了定性研究,探索了日用环境下中医面诊APP的特点以及待解决的问题。(面诊应用设计原则)



% 2、从数据中分析出中医面诊应用潜在的使用场景,提出了实用性敏感性自适应性问题的解决方法。
% 3、实现一个方便易用的在线中医面诊的系统,对界面进行了改进。在面诊舌诊问诊过程中加入了透明性,将内部算法的细节暴露够用户。
% 4、对算法透明性的作用进行了用户调研。

支持后续的研究
拿出透明性作为一个研究案例,也可以做其他的交互研究的实验

// 系统设计

跨平台,版本管理,日志管理, 流程建模

模型分类:打分模型,分类模型
模型执行:指定先后关系
结果合并:合并任务

规则系统:

方便后续接入其他的人脸模型,

模型在线更新

面诊技术设计原则 // 

抽象成平台

平台可以做什么事情

模型系统剥离  

\section{论文结构}

相关工作

用户调研

系统实现

诊断算法透明性实验