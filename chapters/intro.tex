\chapter{绪论}

随着社会的发展,人们的生活节奏越来越快。高强度的工作导致处于亚健康状态的人们越来越多,健康养生也渐渐成为当前热门的话题。尽管如此,紧凑的工作时间却让大部分人没有专门的时间去专业的机构去调理自己的身体,因此人们对日常场景下健康应用的需求非常强烈。
后来,随着计算机技术的进步和移动设备的发展,越来越多的日常健康技术也如雨后春笋般出现来满足用户的这一需求。日常健康应用虽然不能代替专业的诊断,但是能给人们对自己身体状态一个大致的判断,如何设计日常健康技术也成为了研究者关注的重点。
本文的关注点是面诊技术在日常健康场景下的问题,面诊技术的发展和读脸技术和医疗信息化密切相关。因此本章介绍了读脸技术和诊断技术的发展,引出了本文的主要工作,最后介绍了论文的组织结构。

\section{研究背景}

% 读脸技术发展背景
长期以来,人脸在人与人之间相互识别,情感交流等方面起着重要作用。
随着人工智能,特别是基于深度学习的计算机视觉技术的发展,深度神经网络模型能力越来越强,在有些中已经超越了人类水平\cite{he2015delving}。
读脸技术指的是通过读取人的面部图片获取相关的信息,主要应用有人脸检测、人脸识别、情绪识别、性别识别、年龄识别等。
在这个背景下,基于面部图像分析的读脸技术在日常生活中的应用越来越广泛。
例如,通过将人脸图像与人脸库中图像进行比较来识别人脸的身份的人脸识别技术 \cite{Zhang2016Joint}\cite{Schroff2015FaceNet},已广泛应用于考勤管理\cite{surekha2017attendance},访问控制\cite{atick2000continuous}和安防领域\cite{liu2005ibotguard}。
此外,还有相关研究探索了使用读脸技术进行情感识别\cite{corneanu2016survey},识别面相 \cite{Li2007Online}\cite{Tempark2012Chinese},甚至可以用于智商检测\cite{Kleisner2014Perceived}。
随着读脸技术的不断进步,其在人们日常生活场景下的实际应用也越来越普遍。

% 面诊背景及日常工具化
面部在日常生活中除了用于识别和表达情绪之外,也是反映一个人身体健康和精神面貌的重要部位,能够透露出各种健康状况甚至体内疾病的迹象。
在现代医学领域,当医生评估患者的总体体质并得出可能疾病的初步假设时,直接观察面部是医学诊断的一部分\cite{Clifford2006Shortliffe}。例如,患有肝炎和其他肝脏问题的人的脸或眼睛可能带有黄色的色调\cite{Li2008Therapeutic}。
此外,中国文化历史悠久,在长期的积累中,中医方面有着丰富的面部特征和身体状况的面诊理论,看面相在中国也有深厚的文化基础。
中医有经验充足的面诊理论基础,大量学者也为中医面诊信息化做了实证研究。
中医认为,内脏的病变可以反映到体表;相反,通过对外部的诊察,也可以推测内脏的变化\cite{杨淑芳2002伤寒六经辨证初探}。面部是最快表现脏腑病症的部位,通过观察面部的颜色、形状、五官状况等,可以快捷大致地诊断脏腑疾病。
在中医领域也有大量关于利用计算机技术探索面部特征和疾病关系的研究,如面色和乙肝关系的研究\cite{杨宏志2007慢性乙型肝炎肝硬化中医面部五色诊断与临床病理的相关性研究}、基于面部诊断黄疸\cite{艾英1998黄疸病人面部色泽定量实验研究}等。
面部还可以指示一个人的整体心理和情感健康状况,显示出疲劳或疲劳的状态。总体而言,通过观察一个人的脸部可以了解很多关于人的健康的信息。
特别的是,面诊是中医四诊(望闻问切)的重要组成部分。基于中医理论,目前已开发出多种用于临床目的的计算机辅助工具完成健康诊断,包括面诊仪\cite{Liu2014Computerized},舌诊仪\cite{Wang2004An}和脉诊仪\cite{Shu2007Developing}。直到最近,信息技术的进步和移动设备的便携性增强才将其中一些工具变成了日常用户可使用的设备。例如,各类可穿戴和便携式的脉冲读取设备,例如金姆健康推出的指尖脉搏血氧仪———金姆脉诊仪\footnote{http://www.jinmuhealth.com},已商业化用于日常健康监测。

% 日常健康场景下的交互研究

读脸技术和面诊理论的发展使得将计算机面诊技术引入到日常健康场景下在技术上是可行的,然而在如何设计系统,将系统集成到日常健康场景中,相关的研究还远远不够。
目前大部分的面诊技术主要应用场景在诊所环境下,和诊所环境相比,日常健康场景下主要有以下几个方面的问题需要考虑\cite{Clawson2015No} \cite{Epstein2016Beyond} \cite{lupton2017self-tracking}: 

(1) 长期使用。日常健康场景中的面诊技术的使用需要系统能够和用户的生活融合起来,是一个长期的过程。 在日常场景下,如何吸引用户继续使用、如果提高日常使用场景下交互的便捷性、如何引导用户改变自己的生活方式,提高生活质量等也可能是需要考虑的问题。

(2) 设备的多样性。 医院或者诊所环境下,用户是在专门的设备下进行健康诊断。而在日常场景下,所有操作都在用户的移动设备上完成。用户的移动设备差异性比较大,特别是安卓系统的碎片化问题非常严重。不同用户的设备从移动操作系统到硬件计算能力和存储空间的都存在很大的差异性。

(3) 用户的多样性。日常场景下用户的教育程度,日常空余时间,工作内容可能大不相同。比如在日常使用的时候,与医院或者诊所环境下不同,用户在使用过程中是没有专业的医生或者有可询问的技术人员,大部分用户可能没有医疗相关的专业知识,使用起来会比较困难。在日常场景下帮助用户理解诊断过程和诊断结果难度更大。

% 由此看来,相信在不久的将来移动设备上用于日常健康管理的成熟的面诊应用将成为可能。

由此看来,如何设计系统将面诊技术整合到日常健康实践中方面有重大的研究意义和挑战。



\section{本文研究内容与创新}

本文以日常环境下中医面诊应用的交互技术作为研究问题,
通过设计交互实验,深入探索了目前基于面部诊断应用在日常使用场景下的一系列的交互问题,提出了对应的设计上的解决方案并通过实验进行了验证。

本文的研究工作主要对以下几个方面进行了探索:

(1) 对基于面诊养生应用在日常使用的场景下进行了定性研究,选用云中医作为技术探针设计并开展实验,探索了日常健康场景下面诊技术的特点以及待解决的问题,并提出了对应的日常场景下如何设计面诊应用的解决方案。

(2) 为了支持后续的深入研究,设计并开发了一套支持后续面诊交互研究的实验平台。该实验平台提供了版本管理,日志管理,模型管理,问卷关联等功能,方便接入其他新的人脸模型,快速开展人机交互的实验,并降低了系统对用户设备类型和性能的要求。

(3) 在第二个工作点的基础上,利用实验平台,设计并实现了两个实验用的原型系统,分别对第一个工作点提出的设计原则进行了验证实验。结果表明,本文提出的提高可用性的设计方案的确能够提高系统使用的便利性,能够方便用户在日常场景下使用;而本文提出的增加系统透明性的设计原则,的确能够在目前系统算法结果不完美的情况下,提高用户对系统的信赖程度和对结果的接受程度。


\section{论文结构}
本文按照一共组织了六个章节,具体的组织结构如下:

第一章,介绍了本文工作的研究背景,分析了日常使用场景下的特点,最后介绍了本文的主要工作和创新点。

第二章,介绍了和本文工作相关的研究情况,分人机交互领域的日常健康交互技术和诊断技术信息化两个方面分别进行了相关的介绍。

第三章,就本文的研究问题,采用技术探针的方法进行了定性研究,设计了调研问卷和深度访谈。通过对访谈数据的仔细分析,总结出了日常场景下面诊技术的特点,发现了其中存在的问题,并给出日常场景下面诊技术的设计方案。

第四章,为了方便后续的研究工作,介绍了本文设计并实现的用于面诊交互实验的实验平台,该平台提供了日志管理,问卷关联,模型管理,任务管理等功能,并在客户端通过重新设计交互界面,解决了第三章提到的影响用户使用的交互可用性问题。

第五章,在实验平台的基础上,对定性研究提出的结论进行了两个验证实验,验证了第二章工作的有效性。

第六章,总结了本文的三个主要研究工作和不足之处,并对后续工作进行了展望。