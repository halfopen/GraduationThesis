\chapter{绪论}

\section{研究背景}

长期以来,人脸在人与人之间相互识别,情感交流等方面起着重要作用。
随着人工智能,特别是基于深度学习的计算机视觉技术的发展,深度神经网络模型在各大竞赛中已经超越了人类水平。
在这个背景下,基于面部图像分析的读脸技术(Face Reading Technologies)在日常生活中的应用越来越广泛。
例如,通过将人脸图像与人脸库中图像进行比较来识别人脸的身份的人脸识别技术 \cite{Zhang2016Joint}\cite{Schroff2015FaceNet},已广泛应用于考勤管理,访问控制和安全性的所有领域(Apple Face ID)。
此外,还有相关研究探索了使用读脸技术进行情感识别\cite{corneanu2016survey},算命 \cite{Li2007Online}\cite{Tempark2012Chinese},甚至IQ检测\cite{Kleisner2014Perceived}。
随着读脸技术不断进步和日益普及,其在日常生活中的应用领域也越来越广泛。

面部除了用于识别和表达情绪之外,也是反映一个人身体健康和精神面貌的重要部位,能够透露出各种健康状况甚至体内疾病的迹象。
在现代医学领域,当医生评估患者的总体体质并得出可能疾病的初步假设时,直接观察面部是医学诊断的一部分\cite{Clifford2006Shortliffe}。例如,患有肝炎和其他肝脏问题的人的脸或眼睛可能带有黄色调\cite{Li2008Therapeutic}。
此外,中国文化历史悠久,在长期的积累中,中医方面有着丰富的面诊理论。
中医有经验充足的面诊理论基础,大量学者也为中医面诊信息化做了实证研究。
中医认为,内脏的病变可以反映到体表;相反,通过对外部的诊察,也可以推测内脏的变化。面部是最快表现脏腑病症的部位,通过观察面部的颜色、形状、五官状况等,可以快捷大致地诊断脏腑疾病。
在中医领域也有大量关于面部特征和疾病关系的研究,如面色和乙肝关系的研究\cite{杨宏志2007慢性乙型肝炎肝硬化中医面部五色诊断与临床病理的相关性研究}、基于面部诊断黄疸\cite{艾英1998黄疸病人面部色泽定量实验研究}等。
面部还可以指示一个人的整体心理和情感健康状况,显示出疲劳或疲劳的状态。总体而言,通过观察一个人的脸部可以了解很多关于人的健康的信息。
特别的是,面诊是中医四诊(望闻问切)的重要组成部分。基于中医理论,目前已开发出多种用于临床目的的计算机辅助工具完成健康诊断,包括面诊仪\cite{Liu2014Computerized},舌诊仪\cite{Wang2004An}和脉诊仪\cite{Shu2007Developing}。然而,直到最近,信息技术的进步和移动设备的便携性增强才将其中一些工具变成了日常用户可使用的设备。例如,各类可穿戴和便携式的脉冲读取设备,例如金姆健康推出的指尖脉搏血氧仪———金姆脉诊仪\footnote{http://www.jinmuhealth.com},已商业化用于日常健康监测。

于此同时,移动设备近几年也在迅速发展,无论是手机拍照的清晰度还是硬件的计算能力都有了很大的提高。由于移动操作系统本身的优越性加上硬件技术的迅速发展,移动设备的计算能力和存储能力不断增强,移动设备可以处理越来越复杂的任务,在很多方面逐渐取代传统PC设备。
由此看来,相信不久的将来,手机上用于日常健康管理的成熟的面诊应用将成为可能。

但是,和诊所环境相比,日常健康场景下主要有以下几个特点需要考虑: 

(1) 长期使用。日常健康追踪对于用户来说,是一个长期而持续的一个过程。 如何吸引用户继续使用、如果提高日常使用场景下交互的便捷性、如何引导用户改变自己的生活方式,提高生活质量是需要考虑的问题。

(2) 设备的多样性。 医院或者诊所环境下,用户是在专门的设备下进行健康诊断。而在日常场景下,所有操作都在用户的移动设备上完成。用户的移动设备具有多样性,从移动操作系统到硬件计算能力和存储空间的都存在很大的差异性。

(3) 用户的多样性。与医院或者诊所环境下不同,用户在使用过程中是没有专业的医生或者有可询问的技术人员,帮助用户理解诊断过程和诊断结果难度更大。

目前关于面诊技术的研究应用场景主要还是集中在诊所环境下,如何设计系统将面诊技术整合到日常健康实践中方面,它仍然没有得到充分的研究。



\section{本文研究内容与创新}

本文以日常环境下中医面诊应用的交互技术作为研究问题,
使用定性研究的方法设计交互实验,深入探索了目前基于面部诊断应用在日常使用场景下的一系列的交互问题,希望能为将来成熟的面诊应用提供交互方面的指导。

本文的主要研究工作有以下几点:

(1) 对基于面诊养生应用在日常使用的场景下进行了定性研究,选用云中医作为技术探针设计并开展实验,探索了日常使用环境下面诊应用的特点以及待解决的问题,并提出了三条对应的日常场景下面诊应用的设计原则。

(2) 为了支持后续的深入研究,设计并开发了一套支持后续面诊交互研究的实验平台,提供了版本管理,日志管理,模型管理,问卷关联等功能,方便接入其他新的人脸模型快速开展人机交互的实验,并降低了系统对用户设备类型和性能的要求。

(3) 在第二个工作点的基础上,利用实验平台,设计并实现了两个实验用的原型系统,分别对第一个工作点提出的设计原则进行了验证实验。实验的结果和分析表明,本文提出的提高可用性的设计规则的确能够提高系统使用的便利性,能够满足用户日常使用的需求;而本文提出的增加系统透明性的设计原则,的确能够在目前算法的结果不完美的情况下,提高用户对系统的信赖程度和对结果的接受程度。


% 2、从数据中分析出中医面诊应用潜在的使用场景,提出了实用性敏感性自适应性问题的解决方法。
% 3、实现一个方便易用的在线中医面诊的系统,对界面进行了改进。在面诊舌诊问诊过程中加入了透明性,将内部算法的细节暴露够用户。
% 4、对算法透明性的作用进行了用户调研。

% 支持后续的研究
% 拿出透明性作为一个研究案例,也可以做其他的交互研究的实验

% // 系统设计

% 跨平台,版本管理,日志管理, 流程建模

% 模型分类:打分模型,分类模型
% 模型执行:指定先后关系
% 结果合并:合并任务

% 规则系统:

% 方便后续接入其他的人脸模型,

% 模型在线更新

% 面诊技术设计原则 // 

% 抽象成平台

% 平台可以做什么事情

% 模型系统剥离  

\section{论文结构}
本文按照一共组织了六个章节,具体的组织结构如下:

第一章,介绍了本文工作的研究背景,分析了日常使用场景下的特点,最后介绍了本文的主要工作和创新点。

第二章,介绍了和本文工作相关的研究情况,分人机交互领域的日常健康交互技术和诊断技术信息化两个方面分别进行了相关的介绍。

第三章,就本文的研究问题,采用技术探针的方法进行了定性研究,设计了调研问卷和深度访谈。通过对访谈数据的仔细分析,总结出了日常场景下面诊技术的特点,发现了其中存在的问题,并给出日常场景下面诊技术的设计规则。

第四章,为了方便后续的研究工作,介绍了本文设计并实现的用于面诊交互实验的实验平台,该平台提供了日志管理,问卷关联,模型管理,任务管理等功能,并在客户端通过重新设计交互界面,解决了第三章提到的影响用户使用的交互可用性问题。

第五章,在实验平台的基础上,对定性研究提出的结论进行了两个验证实验,验证了第二章工作的有效性。

第六章,总结了文本的研究内容和不足之处,并对未来工作进行了展望。