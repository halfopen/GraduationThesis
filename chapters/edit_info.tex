
\chapter{修改说明}

\section{问题1}
\textbf{文章探索了日常场景下面诊应用新的潜在使用场景,这一条根本就不是创新点。}

正如您所说,日常场景下面诊应用可作为健康诊断工具与医患沟通平台,这一点应属于在用户研究讨论的部分内容。

目前已在论文1.3.4重新论述本文的创新点。

\section{问题2}
\textbf{根据拟定的论文题目,本文在面诊交互技术研究方面讨论不够深入。}

根据您的反馈,对于交互研究方面在论文中补充了相关的背景和所用技术的介绍,同时作为专业硕士论文补充了相关系统设计方面的工作,目前论文题目已修改为《日常场景下面诊系统的设计与实现》。

本文从人机交互的角度研究面诊技术在日常生活场景的使用和设计的问题。
人机交互是一个跨学科领域,研究解决计算机技术在设计和使用过程中的实际问题,使得技术能够更容易和更有效的为用户和组织所使用。
人机交互领域一个重要的认识是,上下文或者场景对我们的行为和技术的使用起着关键性作用\cite{1987Plans}。
关于面诊技术,目前的现状是主要集中在诊所环境下使用,虽然面诊技术在技术层面已经满足日常使用的情况(例如,手机上就已经可以安装使用了),
但对在日常场景下面诊技术应该怎么设计和会如何使用的问题还缺乏理解。针对这些问题,本论文基于已有的面诊技术基础上用技术探针方法,研究了日常场景中面诊技术的设计和使用问题,
以帮助面诊技术更多的应用到日常场景中作为日常健康管理的工具,让用户能大致了解自己的身体情况、增强健康意识、培养健康的生活习惯等,而不是为了替代专业的医生诊断。

关于难点,与相对可控的实验和专业场景不同,日常场景下做用户研究难点在于日常场景复杂多样,并且包括很多家庭空间中的私密空间,因此很难深入到这些场所用实际跟踪和观察的方式去收集用户使用技术的数据。
此外,本文研究的是一个还没有被普遍使用的技术,因此用户还没有实际的使用经验,即无法仅通过用户访谈获得他们的使用数据,如何获得真实可靠的使用数据则更加困难。
面对这些问题,我们采用了人机交互领域的技术探针的方法去收集真实可靠的用户技术使用数据。
通过开发相关技术探针,让用户带到他们的日常生活场景中去使用,经过一段时间之后,收集相关的使用数据进行分析,以了解该技术的使用情况以及设计问题。
技术探针的方法帮我们发现了用户使用存在的问题,也帮助我们启发了相关设计解决方案。该研究成果已经发表在了人机交互CCF A类会议上。

工作量方面,主要包括两部分:一部分是技术探针的实现以及用户调研和数据分析,另一部分是根据前面对用户使用和设计上的了解,进行了系统的实现和验证。
本文将技术探针发现的诸多问题,如持续使用问题、设备多样性问题、系统稳定性与性能问题、系统可解释性问题、算法更新迭代快等在系统设计中解决,也是存在挑战的。
此外,除了要改进交互方面,但目前技术发展迅速,算法更新迭代迅速,目前还没有一个通用的面诊系统设计。为当前诸多面诊系统提供一个通用的解决方案,能及时灵活地接入与迭代新算法也是非常重要且存在挑战的。


\section{问题3}

\textbf{论文结构逻辑混乱。}

感谢您指出,目前论文已按通过技术探针分析发现问题、提出设计策略、设计并实现系统的逻辑安排主要章节,其中所用技术探针的具体细节统一放到附录中,交互设计与系统设计放入同一章。

\section{问题4}
\textbf{参考文献格式不规范,出版地不详、出版者不祥的文献不能作为参考文献。}

感谢您指出错误,此类问题已修改。

\section{问题5}
\textbf{文中标点符号、语言表达不规范。}

正文中标点符号使用中文,语言表达已做大量修改,如去除第一人称描述如\myfont{我}、\myfont{我们}等,改为客观描述等。

\section{问题6}
\textbf{文中近五年内的相关文献引用不足,需要补充并重新撰写绪论及相关部分。}

论文前两章已经重写并补充近五年内的相关文献(部分经典论文除外)。




