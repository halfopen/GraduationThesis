\chapter{诊断算法解释性实验}

\section{系统可解释性}

已有大量的研究表明,增加系统的可解释性,是可以提高用户对系统的信任程度的。
考虑到中医应用的特殊性,普通用户需要提高对应用的理解。因此我们在系统中加入算法的解释性,提升用户的体验。而根据之前的用户调研来看,由于当前诊断和打分模型存在改进的空间,部分的用户也对结果产生了怀疑或者对结果不理解。
在这个基础上,我们希望通过讲模型的判决过程透明化,并对结果进行解释。

在具体代码实现的时候,系统对于是否过程透明、对结果进行解释是通过用户的类型来进行判断的。

对于每一个用户,我们通过哈希算法,对用户名计算哈希值,按照哈希值的奇偶性,将奇数用户归类为不透明用户,偶数用户归类为透明用户。
两类用户在进行面诊舌诊时的流程一样,但是透明用户能够看到背后特征提取算法和诊断算法的中间数据,同时系统为给出的诊断结果进行了解释。


解释在呈现的时候,大致可以分为两种:

1. 结果中文字结果是有提示可以进行点击。如果用户点击了健康报告的分数,会通过弹窗进行详细的解释。

2. 通过雷达图的方式直接显示对结果的影响,交互性比较强。

\subsection{面诊舌诊过程的解释性}

\begin{figure}
    \centering
    \subfigure[不解释]{\includegraphics[width=4.5cm]{images/face_tongue.png}}
    \subfigure[解释]{\includegraphics[width=4.5cm]{images/exp_face_tongue.png}}
    \subfigure[解释舌诊中间结果]{\includegraphics[width=4.5cm]{images/exp_tongue.png}}
    \caption{面诊舌诊的解释}
    \label{fig:face_diags}
\end{figure}

面诊和舌诊流程比较类似,添加解释主要体现在结果的解释,包括中间结果,和对各种体质倾向的影响。
这样用户查看解释之后,能够大致了解本次拍照是否成功,并且知道目前面诊的结果,会对最终的健康报告造成哪些影响。

如图 \ref{fig:face_diags} 所示:

a) 普通用户只能看到结果。

b) 透明类型的用户,可以通过点击结果,查看对结果的解释。

c) 透明类型的用户,不仅可以看到当面诊断的中间结果,也能看到这次诊断的体质倾向得分。



\subsection{问诊的解释}

\begin{figure}
    \centering
    \subfigure[不解释]{\includegraphics[height=10cm]{images/questions.png}}
    \subfigure[解释]{\includegraphics[height=10cm]{images/questions2.png}}
    \subfigure[解释体质术语]{
        \includegraphics[height=10cm]{images/exp_phy.png}
    }
    \caption{问诊}
    \label{fig:questions}
\end{figure}

如图 \ref{fig:questions} 所示:

a) 普通用户,回答问诊问题之后,没有任何提示或者解释。

b) 透明类型的用户,在进行问诊过程中,可以立即看到每个答案对结果的印象,通过下方的雷达图显示了影响的体质倾向类型和具体的数值。
雷达图的更新是实时根据用户的选择进行更新的,提高了交互性。

c) 对中医术语中,各种体质的解释,对于体质内容的解释文字引用自 《中医体质分类研究》标准。



\subsection{诊断结果的解释}
\begin{figure}[ht]
    \centering
    \subfigure[解释]{
        \includegraphics[height=10cm]{images/report3.png}
    }
    \subfigure[不解释]{
        \includegraphics[height=10cm]{images/report.png}
    }
    \caption{诊断报告}
    \label{fig:my_label}
\end{figure}

普通用户在诊断结果页面,可以看到自己的健康分数和体质结果;透明用户可以点击诊断分数,了解这个分数是根据哪些指标,通过哪一个算法计算过来的。

可以点击查看的结果的解释有:面诊舌诊结果的解释,健康分数的解释,体质的解释。

\subsubsection{面诊舌诊结果的解释}

\subsubsection{健康分数的解释}
问诊结果的解释主要是对用户透明诊断结果是如何计算出来的,以及那些问诊的问题对结果有影响,影响程度多少。

\begin{figure}[h]
    \centering
    \subfigure[相关问题]{
        \includegraphics[height=7cm]{images/report7.png}
    }
    \subfigure[雷达图]{
        \includegraphics[height=7cm]{images/report8.png}
    }
    \subfigure[计算公式]{
        \includegraphics[height=7cm]{images/report9.png}
    }
    \caption{分数的解释}
    \label{fig:report_expalin_score}
\end{figure}

如图\ref{fig:report_expalin_score}所示,用户点击诊断页面的分数之后,弹窗里会显示分数相关的问题、雷达图和分数计算公式。

分数相关问题,展示了面诊舌诊对体质分数的影响和问诊对体质分数的影响,无影响的问题则不会显示。其中体质分数的变化分两种,一个是分数的累加,另一种是体质分数的清空。

雷达图对体质分数进行了汇总,给用户展示最终个人的体质倾向的结果。

根据诊断打分模型的内部算法,解释页面的计算公式一共有5种类别,我们使用选项卡的方式,将所有的打分计算公式全部透明给用户,并且默认打开当前计算公式的选项卡。

点击面诊结果,可以看到自己的面部舌部的对于整个诊断的影响。

点击体质分数,可以看到当次诊断中,面诊舌诊和用户自己回答的问题,哪些影响到了最后体质的判断。

\subsubsection{体质的解释}

\begin{figure}[ht]
    \centering
    \subfigure[面诊的解释]{
        \includegraphics[height=7cm]{images/report4.png}
    }
    \subfigure[概念的解释]{
        \includegraphics[height=7cm]{images/report5.png}
    }
    \subfigure[体质相关问题]{
        \includegraphics[height=7cm]{images/report6.png}
    }
    \caption{诊断结果的解释}
    \label{fig:report_explain_phy_1}
\end{figure}


% \section{反馈调研}
% 在新系统完成之后,我们采访了16名用户。
% 在本次调研过程中,用户试用的是同时有新旧界面的版本,除了第一次介绍使用的时候,我们会让用户两个版本都是使用一下,后续不做限时。用户具体使用的时候可以根据自己的喜好选择。
% 经过回访,新版的界面达到了预期,大部分用户觉得使用起来更加地方便。
% 不过值得注意的是,也有少部分的用户喜欢旧版的将面诊,舌诊,问诊分开为三步进行的方式,因为这样比较符合日常生活的习惯。


\section{实验设计}
我们通过在各大社交平台发布海报招募,如图\ref{fig:poster},以及使用问卷星的样本服务, 经过筛选之后,一共招募了100位左右的用户。
\begin{figure}[htb]
    \centering
    \includegraphics[height=8cm]{images/poster.png}
    \caption{招募海报}
    \label{fig:poster}
\end{figure}
\section{实验流程}

每个用户的实验流程如下:

1. 用户通过扫描二维码,或者通过我们给定的链接,进入问卷星调查问卷。

2. 完成调查问卷之后,自动进入云中医在线app。通过调用问卷星提供的企业用户接口, 同时把问卷星的问卷id通过ssojump传给云中医在线应用。通过ssojump中的问卷id, 完成自动登陆, 登陆的用户id为wjx-{问卷星id}, 透明类别为通过问卷星id哈希得到。

3. 在用户完成一次面诊之后,会在健康诊断页面下,看到一个跳转链接,可以选择填写用后问卷。

这样就把调查问卷信息,云中医应用使用日志和用后问卷数据关联起来了。


\section{实验数据获取}
每个用户在参与实验之后,我们可以得到调查问卷的数据,云中医的使用日志,已经用后问卷的数据。

调查问卷: 序号,提交答卷时间,所用时间,来源,来源详情,IP,个人信息	我相信中医养生,了解中医养生,平时注重养生,经常去看中医,希望自己的生活方式更健康,对学习相关中医养生知识感兴趣,认为中医面诊可以了解健康情况,认为中医舌诊可以了解健康情况,认为智能系统可以自动评估健康情况,随机顺序的中医知识问题

用后问卷: 序号,提交答卷时间,所用时间,来源,来源详情,IP,云中医的诊断结果,结果的信任程度,对结果的理解程度,是否愿意使用类似应用,随机顺序的中医知识问题

用户使用日志: 序号,用户名,设备信息,操作名,操作信息,日期 

个人信息包括性别,年龄段,受教育水平,职业,城市,健康状态等。

\subsubsection{关联规则}
对于同一个用户在一次实验过程中,若调查问卷的序号为ID, 则此次用户使用日志的用户名为wjx-ID, 用户问卷的来源详情为wjx-ID。 
通过这个对应关系,我们就可以利用实验平台提供的问卷关联的功能,把三个表的信息,合并到一个表中,通过django-admin导出下载,最终得到特征如表\ref{tab:exp_data}所示。


\begin{table}[h]
    \centering
    \begin{tabular}{ll}
        \toprule
        字段 & 描述 \\ 
        \midrule
        序号 & 调查问卷序号 \\
        性别 & 男/女/未知 \\
        教育程度 & 本科以下/本科/本科以上 \\
        工作类型 & 计算机相关/计算机不相关 \\
        用户类型 & 透明/不透明 \\
        健康得分 & 云中医应用给出的分数 0-100 \\
        信任 & 对结果的信任程度 1-5 \\
        理解 & 对结果的理解程度 1-5 \\
        前健康知识得分 & 使用云中医之前的健康知识得分 \\
        后健康知识得分 & 使用云中医之后的健康知识得分 \\
        查看了哪些解释 & 使用过程中查看的解释类型 \\
        \bottomrule
    \end{tabular}
    \caption{实验数据汇总}
    \label{tab:exp_data}
\end{table}

表格 \ref{tab:exp_data}的字段说明如下:

工作类型: 根据调查问卷的数据,用户自由填写的职业类型有很多,为了便于分析,我们把 IT经理,IT软件设计, hr, it, 互联网, 技术研发, 技术研发人员, 技术经理, 电脑工程师, 研发, 科研, 程序员, 计算机, 软件工程师, 软件开发工程师, 通信, 通讯等归为it相关。

信任、理解: 这两个字段来自用后问卷调查表,取值范围为1-5的整数。

健康知识得分: 使用云中医应用前后的健康知识得分,使用的是用户回答正确中医知识的个数。

用户类型:透明类型的用户才能看到对结果以及术语的解释。

查看了哪些解释: 通过检索日志中关键字获取,包括面诊过程,舌诊过程,体质术语介绍,诊断报告的:面诊结果,舌诊结果,健康分数,体质,分数计算公式等。

\section{实验结果}

\subsection{用户是否愿意继续使用?}

\subsection{}

同时实验分析,我们可以得出,增加解释可以提高用户对应中医知识的理解。

和看解释有关, 用 二元逻辑回归

把用户进行筛选,看了解释和没有看解释的

有解释的,看了没没看的(为什么看了,为什么没看), 

相关性分析,用于分析自变量

挑选相关性比较低的,放到后续的模型中 , 不加*,是很弱的 0.4-0.6比较强

了解中医养生,平时注重养生,了解中医养生
相信中医,了解中医

有没有看解释和it相关性不高

看了解释的影响:
独立样本t检验,差异性分析,没有显著差异

找最相似的不看解释的用户和看了解释的用户对比: 
对结果的信任和理解程度,两个样本没有显著差异



量化结果

定性实验

\section{结构性访谈}


